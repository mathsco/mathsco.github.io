\documentclass[12pt,fleqn]{article}

\newif\ifcomplete
\completetrue
%\completefalse
\def\topic{Integration in Spherical Coordinates}
\def\lecdate{Tues., May. 1}

\oddsidemargin=0in
\textwidth=6.5in
\topmargin=-0.5in
\textheight=9.0in

\input mathMacros.tex
\usepackage{amsmath}
\usepackage{ascii}
%\usepackage{wasysym}
\usepackage{fancyhdr}
\usepackage{/Users/scofield/Library/texmf/latex/timestamp}
\usepackage[pdftex]{hyperref}
\hypersetup{pdfpagemode=None,colorlinks=true}
\usepackage{graphicx}
\usepackage[usenames,dvipsnames]{color}
\def\mbunsc{\blank{0.07in}}
\def\pref#1{(\ref{#1})}
%\def\RE{$\mathrm{Re}\,$}
%\def\IM{$\mathrm{Im}\,$}
\def\RE{$\mbox{Re}\,$}
\def\IM{$\mbox{Im}\,$}
\def\fb#1{\framebox{\parbox[t]{6.5in}{#1}}}
\def\sfb#1{\framebox{\parbox[t]{6.0in}{#1}}}
\def\defn#1{\fb{{\bf Definition}: #1}}
\def\sdefn#1{\sfb{{\bf Definition}: #1}}
\def\thm#1{\fb{{\bf Theorem}: #1}}
\def\sthm#1{\sfb{{\bf Theorem}: #1}}
\def\eg#1{{\bf Example}: #1}
\def\egs#1{{\bf Examples}: #1}
\def\egsc#1{{\bf Examples #1:}}
\def\newt{\vspace{0.2in}\ni}
\setlength{\fboxsep}{5pt}
%\def\vdotprod{{\ascii\BEL}}
\def\vdotprod{\,\mbox{{\Large $\cdot$}}\,}
\def\vcrossprod{\times}


\begin{document}

\pagestyle{fancy}
\fancyhf{}
\lhead{Framework for \lecdate}
\chead{}
\rhead{\topic}
\lfoot{}
\cfoot{\thepage}
\rfoot{}
\renewcommand{\headrulewidth}{0.4pt}
\renewcommand{\footrulewidth}{0.4pt}

\thispagestyle{empty}

\begin{center}
  \Large{MATH 162: Calculus II} \\
  \large{Framework for \lecdate} \\
  \large{\topic}
\end{center}

\vs{0.2in}
\ni
{\bf Today's Goal}:
To learn to set up and evaluate triple integrals in
spherical coordinates.

\vspace{0.15in}
\ni
{\bf Important Note}: In conjunction with this framework,
you should look over Section 13.7 of your text.

\vspace{0.15in}
\subsection*{Simple Equations in Spherical Coordinates and Their Graphs}
\bi
\item
  $\rho = \rho_0$ (a constant) corresponds to a sphere of radius
  $\rho_0$.
\item
  $\phi = \phi_0$ corresponds to a cone with vertex at the origin
  and the $z$-axis as axis of symmetry.
\item
  $\theta = \theta_0$ corresponds to a half-plane with $z$-axis
  as the terminal edge.
\ei

\vspace{0.2in}
\subsection*{Changing $(x,y,z)$ to $(\rho,\phi,\theta)$}

\begin{minipage}[t]{4.5in}
  Recall that we have the following relationships:
  $$ \begin{array}{l}
	x \;=\; \rho\sin\phi\cos\theta, \\
	y \;=\; \rho\sin\phi\sin\theta, \\
	z \;=\; \rho\cos\phi. \end{array} $$
  Thus, the equation (in rectangular coordinates)
  $$ (x-2)^2 + y^2 + z^2 \;=\; 4 $$
\end{minipage}\hspace{0.3in}
\begin{minipage}[t]{2in}
  \mbox{}

  \vspace{-0.5in}
  \includegraphics[width=2in]{figs/sphCylCoords.png}
\end{minipage}
for a sphere of radius 2 centered at the point $(x,y,z) = (2, 0, 0)$
may be rewritten as
$$ \rho \;=\; 2\left(\sin\phi\cos\theta + \sqrt{\sin^2\phi\cos^2\theta + 1}
	\right). $$
(Try verifying this.)

\vspace{0.2in}
\subsection*{Volume Element $dV$ in Spherical Coordinates}
\begin{minipage}[t]{4.5in}
  Pictured at right is a typical ``volume element'' $\Delta V$
  at a spherical point $(\rho, \phi, \theta)$ corresponding to
  small changes $\Delta\rho$, $\Delta\phi$ and $\Delta\theta$
  in the spherical variables.  Its sides, as can be verified
  using trigonometry, have approximate measures $\Delta\rho$,
  $(\rho\Delta\phi)$ and $(\rho\sin\phi\Delta\theta)$.  Thus
  $$ dV \;=\; \rho^2\sin\phi\,d\rho\,d\phi\,d\theta. $$
  As a result
\end{minipage}\hspace{0.2in}
\begin{minipage}[t]{2.0in}
  \mbox{}

  \vspace{-0.6in}
  \includegraphics[width=2.4in]{figs/volElemSphCoords.jpg}
\end{minipage}\\
$$ \iiint\limits_D f(x,y,z)\,dV \;=\; \iiint\limits_D
	\rho^2\sin\phi\,f(\rho\sin\phi\cos\theta, \rho\sin\phi\sin\theta,
	\rho\cos\phi) \,d\rho\,d\phi\,d\theta, $$\\[16pt]
\egs{}
\be
\item
  Evaluate $\iiint_D 16z\,dV$, where $D$ is the upper half of
  the sphere $x^2 + y^2 + z^2 = 1$.
\item
  Find the volume of the smaller section cut from a solid ball
  of radius $a$ by the plane $z=1$.
\ee

\end{document}

