\documentclass[12pt,fleqn]{article}

\newif\ifcomplete
\completetrue
%\completefalse
\def\topic{Functions of Multiple Variables}
\def\lecdate{Tues., Feb.~27}

\oddsidemargin=0in
\textwidth=6.5in
\topmargin=-0.5in
\textheight=9.0in

\input mathMacros.tex
\usepackage{amsmath}
\usepackage{fancyhdr}
\usepackage{/Users/scofield/Library/texmf/latex/timestamp}
\usepackage[pdftex]{hyperref}
\hypersetup{pdfpagemode=None,colorlinks=true}
\usepackage{graphicx}
\usepackage[usenames,dvipsnames]{color}
\def\unsc{\blank{0.07in}}
%\def\RE{$\mathrm{Re}\,$}
%\def\IM{$\mathrm{Im}\,$}
\def\RE{$\mbox{Re}\,$}
\def\IM{$\mbox{Im}\,$}
\def\fb#1{\framebox{\parbox[t]{6.5in}{#1}}}
\def\sfb#1{\framebox{\parbox[t]{6.0in}{#1}}}
\def\defn#1{\fb{{\bf Definition}: #1}}
\def\sdefn#1{\sfb{{\bf Definition}: #1}}
\def\thm#1{\fb{{\bf Theorem}: #1}}
\def\eg#1{{\bf Example}: #1}
\def\egs#1{{\bf Examples}: #1}
\setlength{\fboxsep}{5pt}


\begin{document}

\pagestyle{fancy}
\fancyhf{}
\lhead{MATH 162---Framework for \lecdate}
\chead{}
\rhead{\topic}
\lfoot{}
\cfoot{\thepage}
\rfoot{}
\renewcommand{\headrulewidth}{0.4pt}
\renewcommand{\footrulewidth}{0.4pt}

\thispagestyle{empty}

\begin{center}
  \Large{MATH 162: Calculus II} \\
  \large{Framework for \lecdate} \\
  \large{\topic}
\end{center}

\vs{0.2in}
\ni
\defn{
A {\em function} (or {\em function of n variables}) $f$ is a
rule that assigns to each ordered $n$-tuple of real numbers
$(x_1, x_2, \ldots, x_n)$ in a certain set $D$ a real number
$f(x_1, x_2, \ldots, x_n)$.  The set $D$ is called the
{\em domain} of the function.
}

\vspace{0.2in}
\ni
\eg{
Most real-life functions are, in fact, functions of multiple
variables.  Here are some:
\be
\item $v(r,h) = \pi r^2 h$
\item $d(x,y,z) = \sqrt{x^2 + y^2 + z^2}$
\item $g(m_1, m_2, R) = G m_1 m_2 / R^2 \quad$ ($G$ is a constant)
\item $P(n,T,V) = nRT/V \quad$ ($R$ is a constant)
\ee
}

\ni
Of course, one can hold fixed the values of all but one
of the input variables and thereby create a function of
a single variable.  For instance, the way the volume of
a right-circular cylinder whose height is 3 varies with
its radius is given by the formula
$$ V(r) \;=\; 3\pi r^2, \quad r \ge 0. $$

\section*{Graphing}
\subsection*{Functions of a single variable}

To graph a function of a single variable requires two coordinate
axes.  When we write $y = f(x)$, it is implied that $x$ is a
possible input and the $y$-value is the corresponding output.  We
think of the domain (the set of all possible inputs) of $f$ as
consisting of some part of the real line, the graph of $f$ (often
called a ``curve'') as having a point at location $(x,f(x))$ for
each $x$ in the domain of $f$.  Keep in mind that, given an
arbitrary equation involving $x$ and $y$, it is not always
the case that
\be
\item[(i)] we {\em want} to make $y$ be the dependent variable, and
\item[(ii)] if we {\em do} solve for $y$, the result is a {\em function}.
\ee
\eg{$x^2 + y^2 = 4$}

\subsection*{Functions of multiple variables}
We have the following analogies for functions of multiple variables:
\bi
\item
  When nothing explicit is said about the inputs to a function of
  multiple variables, we take the domain to be as inclusive as
  possible. \\[4pt]
  \egs{}
  \bi
  \item[] $f(x,y) = \sqrt{xy}$
  \vspace{0.2in}
  \item[] $f(x,y) = xy(x^2 + y)^{-1}$
  \ei
\vspace{0.1in}
\item
  The graphs of functions of $n$ variables are $n$-dimensional
  objects drawn in a coordinate frame involving $(n+1)$
  mutually-perpendicular coordinate axes.  (Think of a curve
  which is the graph of $y = f(x)$ as a 1-dimensional object
  weaving through 2-dimensional space.)

  \vspace{0.1in}
  As a corollary: {\em It is not possible to produce the graph of
  a function of 3 or more variables.}  A possible work-around:
  level sets.

  \vspace{0.1in}
  \sdefn{Let $f$ be a function of $n$ variables, and $c$ be a
  real number.  The set of all $n$-tuples $(x_1, \ldots, x_n)$
  for which $f(x_1, \ldots, x_n) = c$ is called the {\em c level
  set of f}.}

  \vspace{0.1in}
  \egs{}
  \bi
  \item[] $f(x,y) = y^2 - x^2$
  \vspace{0.2in}
  \item[] $f(x,y,z) = z - x^2 - 2y^2$
  \ei
\vspace{0.1in}
\item
  Not every equation involving $x$, $y$ and $z$ yields $z$
  as a single function of $x$ and $y$. \\[4pt]
  \egs{}
  \bi
  \item[] $x - y^2 - z^2 = 0$
  \vspace{0.2in}
  \item[] $x^2 + y^2 + z^2 = 4$
  \ei
\vspace{0.1in}
\item
  One may assume a missing variable is implied and
  takes on all real values.

  \vspace{0.1in}
  \eg{}The meaning of $x=1$ in 1, 2 and 3 dimensions.

  \vspace{0.1in}
  One may need more than one equation/inequality to
  describe certain regions of space.

  \vspace{0.1in}
  \eg{}
  $x^2 + (y-1)^2 \le 1$, \; $z = -1$
\ei

\iffalse
\ni

\vspace{0.2in}
\ni
Comparisons with functions of 1 variable\\
\begin{tabular}{|c||c|c|}
  \hline
  & Fns.~of 1 var. & Fns.~of multiple vars. \\ \hline
  \# of input vars. & 1 & $n$ (with $n \ge 2$) \\
  \# of output vars. & 1 & 1 \\
  \# of axes & 2 & $n+1$ \\
  domain is region of & real line & $n$-dimensional space \\
\end{tabular}
\fi

\end{document}

