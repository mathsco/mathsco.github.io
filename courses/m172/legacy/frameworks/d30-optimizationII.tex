\documentclass[12pt,fleqn]{article}

\newif\ifcomplete
\completetrue
%\completefalse
\def\topic{Constrained Optimization of Functions of 2 Variables}
\def\lecdate{Mon., Apr.~9}

\oddsidemargin=0in
\textwidth=6.5in
\topmargin=-0.5in
\textheight=9.0in

\input mathMacros.tex
\usepackage{amsmath}
\usepackage{ascii}
%\usepackage{wasysym}
\usepackage{fancyhdr}
\usepackage{/Users/scofield/Library/texmf/latex/timestamp}
\usepackage[pdftex]{hyperref}
\hypersetup{pdfpagemode=None,colorlinks=true}
\usepackage{graphicx}
\usepackage[usenames,dvipsnames]{color}
\def\mbunsc{\blank{0.07in}}
\def\pref#1{(\ref{#1})}
%\def\RE{$\mathrm{Re}\,$}
%\def\IM{$\mathrm{Im}\,$}
\def\RE{$\mbox{Re}\,$}
\def\IM{$\mbox{Im}\,$}
\def\fb#1{\framebox{\parbox[t]{6.5in}{#1}}}
\def\sfb#1{\framebox{\parbox[t]{6.0in}{#1}}}
\def\defn#1{\fb{{\bf Definition}: #1}}
\def\sdefn#1{\sfb{{\bf Definition}: #1}}
\def\thm#1{\fb{{\bf Theorem}: #1}}
\def\sthm#1{\sfb{{\bf Theorem}: #1}}
\def\eg#1{{\bf Example}: #1}
\def\egs#1{{\bf Examples}: #1}
\def\newt{\vspace{0.2in}\ni}
\setlength{\fboxsep}{5pt}
%\def\vdotprod{{\ascii\BEL}}
\def\vdotprod{\,\mbox{{\Large $\cdot$}}\,}
\def\vcrossprod{\times}


\begin{document}

\pagestyle{fancy}
\fancyhf{}
\lhead{Framework for \lecdate}
\chead{}
\rhead{\topic}
\lfoot{}
\cfoot{\thepage}
\rfoot{}
\renewcommand{\headrulewidth}{0.4pt}
\renewcommand{\footrulewidth}{0.4pt}

\thispagestyle{empty}

\begin{center}
  \Large{MATH 162: Calculus II} \\
  \large{Framework for \lecdate} \\
  \large{\topic}
\end{center}

\vs{0.2in}
\ni
{\bf Today's Goal}:
To be able to find absolute extrema for functions
of two variables on closed and bounded domains.

\vspace{0.3in}
\ni
\defn{A function $f$ of two variables is said to have
\be
\item
  a {\em global maximum} (or {\em absolute maximum}) at the
  point $(a,b)\in \dom(f)$ if $f(a,b) \ge f(x,y)$ for all
  points $(x,y)$ in $\dom(f)$.
\item
  a {\em global minimum} (or {\em absolute minimum}) at the
  point $(a,b)\in \dom(f)$ if $f(a,b) \ge f(x,y)$ for all
  points $(x,y)$ in $\dom(f)$.
\ee
The value $f(a,b)$ is correspondingly called the {\em global}
(or {\em absolute}) {\em maximum or minimum} value of $f$.
}\\[18pt]
Before we embark on a process of looking for absolute extrema,
it would be nice to know that what we are looking for is out
there to be found.  The following theorem supplies such an assurance
in the special case that the domain of $f$ is closed and bounded.\\[18pt]
\thm{
({\em Extreme Value Theorem})\;
Suppose $f(x,y)$ is a continuous function on a closed and bounded
region $R$ of the $xy$-plane.  Then there exist points
$(a,b)$ and $(c,d)$ in $R$ for which
$$ f(a,b) \ge f(x,y) \qquad\mbox\qquad f(c,d) \le f(x,y) $$
for all $(x,y)$ in $R$.
}\\[18pt]

%% TLS pick it up here

\vs{0.1in}
\ni
Remarks:
\bi
\item
  In some simple cases, it is even possible that $(a,b)$ and
  $(c,d)$ from the theorem are the same point.

\item
  The theorem says that ``continuity of $f$ over a closed, bounded
  region $R$'' provides {\em sufficient} conditions to guarantee
  that $f$ reaches maximum and minimum values in $R$.  They are not
  {\em necessary}, however, meaning that it is possible for $f$ to
  attain such extrema even if one or both of these conditions (the
  ``continuity'' or the ``closed and boundedness of $R$'') is
  not in place.
\ei

\vspace{0.1in}
\ni
\egs{}
\be
\item
  Find the maximum value of $\;f(x,y) = 49 - x^2 - y^2\;$ along the
  line $\; x + 3y = 10$.

\item
  Find the absolute extrema of
  $\;f(x,y) = x^2 - y^2 - 2x + 4y\;$ on the region of the $xy$-plane
  bounded below by the $x$-axis, above by the line $y = x+2$,
  and on the right by the line $x=2$.

\item
  Find the absolute extrema of
  $\;f(x,y) = x^2 + 2y^2\;$ on the closed disk $\;x^2 + y^2 \le 1$.

\item
  Find the maximum of $\;f(x,y,z) = 36 - x^2 - y^2 - z^2\;$ subject
  to the constraint $\;x + 4y - z = 21$.

\item
  Find the point on the graph of $\; z = x^2 + y^2 + 10\;$ closest
  to the plane $\; x + 2y - z = 0$.
\ee
\end{document}

