\documentclass[12pt,fleqn]{article}

\newif\ifcomplete
\completetrue
%\completefalse
\def\topic{Partial Derivatives}
\def\lecdate{Thurs., Mar.~1}

\oddsidemargin=0in
\textwidth=6.5in
\topmargin=-0.5in
\textheight=9.0in

\input mathMacros.tex
\usepackage{amsmath}
\usepackage{fancyhdr}
\usepackage{/Users/scofield/Library/texmf/latex/timestamp}
\usepackage[pdftex]{hyperref}
\hypersetup{pdfpagemode=None,colorlinks=true}
\usepackage{graphicx}
\usepackage[usenames,dvipsnames]{color}
\def\unsc{\blank{0.07in}}
%\def\RE{$\mathrm{Re}\,$}
%\def\IM{$\mathrm{Im}\,$}
\def\RE{$\mbox{Re}\,$}
\def\IM{$\mbox{Im}\,$}
\def\fb#1{\framebox{\parbox[t]{6.5in}{#1}}}
\def\sfb#1{\framebox{\parbox[t]{6.0in}{#1}}}
\def\defn#1{\fb{{\bf Definition}: #1}}
\def\sdefn#1{\sfb{{\bf Definition}: #1}}
\def\thm#1{\fb{{\bf Theorem}: #1}}
\def\sthm#1{\sfb{{\bf Theorem}: #1}}
\def\eg#1{{\bf Example}: #1}
\def\egs#1{{\bf Examples}: #1}
\setlength{\fboxsep}{5pt}


\begin{document}

\pagestyle{fancy}
\fancyhf{}
\lhead{MATH 162---Framework for \lecdate}
\chead{}
\rhead{\topic}
\lfoot{}
\cfoot{\thepage}
\rfoot{}
\renewcommand{\headrulewidth}{0.4pt}
\renewcommand{\footrulewidth}{0.4pt}

\thispagestyle{empty}

\begin{center}
  \Large{MATH 162: Calculus II} \\
  \large{Framework for \lecdate} \\
  \large{\topic}
\end{center}

\vs{0.2in}
\ni
{\bf Today's Goal}: To understand what is meant by a partial derivative.

\vspace{0.2in}
\ni
\begin{minipage}[t]{3in}
In the \href{http://www.calvin.edu/~scofield/courses/m162/S07/frameworks/d16-fnsOfMultVarsIntro.pdf}{framework that introduced multivariate functions},
we indicated that one can always turn a function of $n$ variables (like
the one depicted at top right) into
a function of one particular variable by holding the others constant.
Suppose $(x_0,y_0)$ is an interior point to the $\dom(f)$ (the black
dot, for example).  Holding $y=y_0$ fixed and letting
$x$ take on values near $x_0$ traces out a curve on this surface.
This curve is the intersection of our surface $z = f(x,y)$ and
the plane $y = y_0$.  (See the bottom figure.)

\vspace{0.2in}
\ni
One might ask what the slope of this curve (the one where
the plane and surface intersect) is above the point $(x_0,y_0)$---that
is, at the location of the point $(x_0, y_0, f(x_0,y_0))$ (the yellow
dot).  The answer would be found by taking a derivative of the
function of $x$ that results by holding $y = y_0$ fixed:
\end{minipage}\hspace{0.3in}
\begin{minipage}[t]{3in}
  \mbox{}

  \vspace{-1.2in}
  \includegraphics[width=3in]{figs/surface.pdf}

  \vspace{-1.5in}
  \includegraphics[width=3in]{figs/surfaceConst_ySlice.pdf}
\end{minipage}

\vspace{-0.75in}
$$ \frac{\partial f}{\partial x}\Big|_{(x_0,y_0)} \;:=\;
	\lim_{h\rightarrow 0} \frac{f(x_0+h,y_0) - f(x_0,y_0)}{h}, $$
called the {\em partial derivative of f with respect to x at} $(x_0,y_0)$.
This partial derivative has various other notations:
$$ \frac{\partial f}{\partial x}(x_0, y_0), \quad f_x(x_0, y_0), \quad
	\frac{\partial z}{\partial x}\Big|_{(x_0,y_0)}. $$
Of course, the partial derivative is a function of $x$ and $y$ in its
own right.  When we think of it that way, we write
$$ \frac{\partial f}{\partial x} \quad\mbox{or}\quad f_x. $$

\np
\ni
We can also take partial derivatives with respect to $y$ (or other
variables, if $f$ is a function of more than 2 variables).  The resulting
partial derivatives may be differentiated again:
\bq
  \begin{tabular}{ll}
	$\ds{\frac{\partial^2 f}{\partial y^2}}$, \; or \; $f_{yy}$ &
	  get this by differentiating $f_y$ with respect to $y$. \\[16pt]
	$\ds{\frac{\partial^2 f}{\partial x \partial y}}$, \; or \; $f_{yx}$ &
	  get this by differentiating $f_y$ with respect to $x$. \\[16pt]
	$\ds{\frac{\partial^2 f}{\partial y^2 \partial x}}$, \; or \; $f_{xyy}$
	  \phantom{XX} & get this by differentiating $f_x$
	  twice with respect to $y$.
  \end{tabular}
\eq

\vspace{0.2in}
\ni
The usual theorems that provide shortcuts to taking derivative may
be applied, keeping in mind which variable(s) is being held constant.\\[8pt]
\egs{}
\bi
%\item[] $\ds{ f(x,y) = x^5 - 6x^2 y + 2xy^3 - x }$
\item[] $\ds{ \frac{2x^2 - y}{3x - xy^2} }$\\[24pt]
\item[] $\ds{ \exp(x/y^2) }$\\[24pt]
\item[] $\ds{ \ln(xy^2) }$
\ei

\end{document}

