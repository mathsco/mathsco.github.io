\documentclass[12pt,fleqn]{article}

\newif\ifcomplete
\completetrue
%\completefalse
\def\topic{Double Integrals in Polar Coordinates}
\def\lecdate{Mon., Apr.~23}

\oddsidemargin=0in
\textwidth=6.5in
\topmargin=-0.5in
\textheight=9.0in

\input mathMacros.tex
\usepackage{amsmath}
\usepackage{ascii}
%\usepackage{wasysym}
\usepackage{fancyhdr}
\usepackage{/Users/scofield/Library/texmf/latex/timestamp}
\usepackage[pdftex]{hyperref}
\hypersetup{pdfpagemode=None,colorlinks=true}
\usepackage{graphicx}
\usepackage[usenames,dvipsnames]{color}
\def\mbunsc{\blank{0.07in}}
\def\pref#1{(\ref{#1})}
%\def\RE{$\mathrm{Re}\,$}
%\def\IM{$\mathrm{Im}\,$}
\def\RE{$\mbox{Re}\,$}
\def\IM{$\mbox{Im}\,$}
\def\fb#1{\framebox{\parbox[t]{6.5in}{#1}}}
\def\sfb#1{\framebox{\parbox[t]{6.0in}{#1}}}
\def\defn#1{\fb{{\bf Definition}: #1}}
\def\sdefn#1{\sfb{{\bf Definition}: #1}}
\def\thm#1{\fb{{\bf Theorem}: #1}}
\def\sthm#1{\sfb{{\bf Theorem}: #1}}
\def\eg#1{{\bf Example}: #1}
\def\egs#1{{\bf Examples}: #1}
\def\egsc#1{{\bf Examples #1:}}
\def\newt{\vspace{0.2in}\ni}
\setlength{\fboxsep}{5pt}
%\def\vdotprod{{\ascii\BEL}}
\def\vdotprod{\,\mbox{{\Large $\cdot$}}\,}
\def\vcrossprod{\times}


\begin{document}

\pagestyle{fancy}
\fancyhf{}
\lhead{Framework for \lecdate}
\chead{}
\rhead{\topic}
\lfoot{}
\cfoot{\thepage}
\rfoot{}
\renewcommand{\headrulewidth}{0.4pt}
\renewcommand{\footrulewidth}{0.4pt}

\thispagestyle{empty}

\begin{center}
  \Large{MATH 162: Calculus II} \\
  \large{Framework for \lecdate} \\
  \large{\topic}
\end{center}

\vs{0.2in}
\ni
{\bf Today's Goal}:
To learn the mechanics of setting up double integrals
in polar coordinates, and learn to recognize situations
in double integrals that may be easier in polar form.

\vspace{0.15in}
\ni
{\bf Important Note}: In conjunction with this framework,
you should look over Section 13.4 of your text.

\vspace{0.15in}
\subsection*{Polar Rectangles}
While the specific integrand $f(x,y)$ plays a large role in
how difficult it is to {\em evaluate} a double integral
$\iint_R f(x,y) \,dA$, it is the region $R$ {\em alone} that
determines what limits of integration one uses in formulating
an iterated integral.  When $R$ is the rectangular region
$R: \; a \le x \le b, \; c \le y \le d$, setting up an
iterated integral is quite easy (the limits are simply the
bounding $x$ and $y$-values for the rectangle).

Correspondingly, if our region $R$ is a {\em polar rectangle}
$$ a \le r \le b, \; \alpha \le \theta \le \beta, $$
then it will be easy to find limits of integration for an
iterated integral in polar coordinates.\\[10pt]
{\bf Some examples of polar rectangles}:
\bi
\item $0 \le r \le 1, \; 0 \le \theta \le 2\pi$
\item $1 \le r \le 2, \; 0 \le \theta \le 2\pi$
\item $\ds{ 1 \le r \le 2, \; \frac{\pi}{4} \le \theta \le \pi}$
\ei

\subsection*{Double Integrals in Polar Coordinates over a
Polar Rectangle $R: \; r_1\le r\le r_2, \; \alpha\le\theta\le\beta$}
\bi
\item
  If the integrand is $f(x,y)$ (i.e., if it is given in terms of
  rectangular coordinates), one must find the appropriate expression
  in polar coordinates by substituting $r\cos\theta$ for $x$,
  $r\sin\theta$ for $y$.
\item
  The $dA$ in $\iint_R f(x,y)\, dA$ becomes $dx\,dy$ or $dy\,dx$
  when written as an iterated integral in rectangular form.

  In polar form, $dA = r \,dr\,d\theta$ because of the need for the
  \href{http://www.math.umn.edu/~rogness/multivar/polartransformation.html}{{\em area expansion factor}} $r$.
\ei
\eg{}
Compute the volume under the hemisphere $z = \sqrt{1 - x^2 - y^2}$
above the polar rectangle $R: \; 0 \le r \le 1/2,
\; 0 \le \theta \le \pi$ in the plane.

\subsection*{Bounded Regions}
Our regions of integration for double integrals are not always
rectangles (neither in the usual sense, nor in the polar sense).
Often the region $R$ of integration for a double integral
$\iint_R f(x,y)\,dA$ is described as ``the region bounded
by the curves \ldots.''  In such instances, one step in
setting up an iterated integral involves finding points
where the curves intersect.\\[8pt]
\eg{}
Find the area of the region outside the circle
$r = 2$ and inside the circle $r = 4\sin\theta$.

\subsection*{More Examples}
\be
\item
  Find the volume of the region bounded by the paraboloid
  $z = 10 - 3x^2 - 3y^2$ and the plane $z = 4$.
\item
  Evaluate the iterated integral $\ds{\int_0^2
  \int_{-\sqrt{4-y^2}}^{\sqrt{4-y^2}} x^2 y^2 \,dx\,dy}$.
\ee

\end{document}

