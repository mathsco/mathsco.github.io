\documentclass[12pt,fleqn]{article}

\newif\ifcomplete
\completetrue
%\completefalse
\def\topic{Vector Functions and Integral Calculus}
\def\lecdate{Fri., Mar.~16}

\oddsidemargin=0in
\textwidth=6.5in
\topmargin=-0.5in
\textheight=9.0in

\input mathMacros.tex
\usepackage{amsmath}
\usepackage{ascii}
%\usepackage{wasysym}
\usepackage{fancyhdr}
\usepackage{/Users/scofield/Library/texmf/latex/timestamp}
\usepackage[pdftex]{hyperref}
\hypersetup{pdfpagemode=None,colorlinks=true}
\usepackage{graphicx}
\usepackage[usenames,dvipsnames]{color}
\def\mbunsc{\blank{0.07in}}
%\def\RE{$\mathrm{Re}\,$}
%\def\IM{$\mathrm{Im}\,$}
\def\RE{$\mbox{Re}\,$}
\def\IM{$\mbox{Im}\,$}
\def\fb#1{\framebox{\parbox[t]{6.5in}{#1}}}
\def\sfb#1{\framebox{\parbox[t]{6.0in}{#1}}}
\def\defn#1{\fb{{\bf Definition}: #1}}
\def\sdefn#1{\sfb{{\bf Definition}: #1}}
\def\thm#1{\fb{{\bf Theorem}: #1}}
\def\sthm#1{\sfb{{\bf Theorem}: #1}}
\def\eg#1{{\bf Example}: #1}
\def\egs#1{{\bf Examples}: #1}
\def\newt{\vspace{0.2in}\ni}
\setlength{\fboxsep}{5pt}
%\def\vdotprod{{\ascii\BEL}}
\def\vdotprod{\,\mbox{{\Large $\cdot$}}\,}
\def\vcrossprod{\times}


\begin{document}

\pagestyle{fancy}
\fancyhf{}
\lhead{MATH 162---Framework for \lecdate}
\chead{}
\rhead{\topic}
\lfoot{}
\cfoot{\thepage}
\rfoot{}
\renewcommand{\headrulewidth}{0.4pt}
\renewcommand{\footrulewidth}{0.4pt}

\thispagestyle{empty}

\begin{center}
  \Large{MATH 162: Calculus II} \\
  \large{Framework for \lecdate} \\
  \large{\topic}
\end{center}

\vs{0.2in}
\ni
{\bf Today's Goal}: To understand how to integrate vector functions.

\vspace{0.2in}
\subsection*{Indefinite integrals}
Recall: For scalar functions $f(t)$,
\bi
\item
  A function $F$ is called an antiderivative of $f$ on the interval
  $I$ if $F'(t) = f(t)$ at each point $t\in I$.
\item
  Given any antiderivative $F$ of $f$ and any constant $C$,
  $F(t) + C$ is also an antiderivative of $f$.
\item
  The indefinite integral
  $$ \int f(t)\, dt $$
  stands for the set of all antiderivatives of $f$.
\ei
{\bf For a vector function} $\mbr(t) = f(t)\mbi + g(t)\mbj + h(t)\mbk$,
\bi
\item
  An antiderivative of $\mbr(t)$ on an interval $I$ is another
  vector function $\mbR(t)$ for which $\mbR'(t) = \mbr(t)$
  at each $t\in I$.
\item
  Finding an antiderivative of $\mbr(t)$ comes down to finding
  antiderivatives for its component functions.  That is, if
  $F$, $G$ and $H$ are antiderivatives of $f$, $g$ and $h$
  on the interval $I$, then
  $$ \mbR(t) \;=\; F(t)\mbi + G(t)\mbj + H(t)\mbk $$
  is an antiderivative of $\mbr(t)$.
\item
  Given any antiderivative $\mbR(t)$ of $\mbr(t)$ and any constant
  vector $\mbC$, $\mbR(t) + \mbC$ is also an antiderivative of $\mbr(t)$.
\item
  The symbol
  $$ \int \mbR(t)\,dt $$
  stands for the set of all antiderivatives of $\mbr$.
\ei

\newt
\eg{}
$\ds{\int \left[\left(\frac{1}{1 + t^2}\right)\mbi + (\sin t\cos t)\,\mbj
  + \left(\frac{t}{\sqrt{1 + 3t^2}}\right)\mbk\right]\,dt}$
% \;=\; (\arctan t)\,\mbi + \left(\frac{\sin^2 t}{2}\right)\mbj
% + \left(\frac{2\sqrt{1 + 3t^2}}{3}\right)\mbk + \mbC.} $

\vspace{0.1in}
\subsection*{Definite Integrals}

\vspace{0.1in}
\defn{Suppose that the component functions of
$\mbr(t) = f(t)\mbi + g(t)\mbj + h(t)\mbk$ are all integrable
over the interval $[a,b]$ (true, say, if each of $f$, $g$ and
$h$ are continuous over that interval).  Then we say the vector
function $\mbr(t)$ is {\em integrable over} $[a,b]$, and define
its integral to be
$$ \int_a^b \mbr(t)\,dt \;:=\; \left(\int_a^b f(t)\,dt\right)\mbi
  + \left(\int_a^b g(t)\,dt\right)\mbj + \left(\int_a^b h(t)\,dt\right)\mbk. $$
}

\newt
Since the fundamental theorem of calculus holds for the components
of $\mbr(t)$, it holds for $\mbr(t)$ as well.  Here we state just
part II.

\newt
\thm{If $\mbr(t)$ is continuous at each point of the interval
$[a,b]$ and if $\mbR$ is any antiderivative of $\mbr$ on $[a,b]$, then
$$ \int_a^b \mbr(t)\, dt \;=\; \mbR(b) - \mbR(a). $$
}\\[8pt]

\newt
{\bf Application: Projectile motion}\\[10pt]
For projectiles near enough to sea level, we think of them as having
constant acceleration $\mba(t) = -g\mbk$, where $g$ has the value
9.8 $m/s^2$ or 32 $ft/s^2$.  Since velocity is an antiderivative of
acceleration, we may write
$$ \int_0^t \mba \,d\tau \;=\; -gt\mbk \;=\; \mbv(t) - \mbv(0), $$
or, abbreviating $\mbv(0)$ by $\mbv_0$,
$$ \mbv(t) \;=\; \mbv_0 - gt\mbk. $$
The position $\mbr(t)$ is an antiderivative of velocity, so
$$ \int_0^t \mbv(\tau)\,d\tau \;=\; t\mbv_0 - \frac{1}{2}gt^2\mbk
  \;=\; \mbr(t) - \mbr(0), $$
or
$$ \mbr(t) \;=\; \mbr_0 + t\mbv_0 - \frac{1}{2}gt^2\mbk, $$
where $\mbr_0 := \mbr(0)$ is the initial position.

\end{document}

