\documentclass[12pt,fleqn]{article}

\newif\ifcomplete
\completetrue
%\completefalse
\def\topic{Vector Functions and Differential Calculus}
\def\lecdate{Thurs., Mar.~15}

\oddsidemargin=0in
\textwidth=6.5in
\topmargin=-0.5in
\textheight=9.0in

\input mathMacros.tex
\usepackage{amsmath}
\usepackage{ascii}
%\usepackage{wasysym}
\usepackage{fancyhdr}
\usepackage{/Users/scofield/Library/texmf/latex/timestamp}
\usepackage[pdftex]{hyperref}
\hypersetup{pdfpagemode=None,colorlinks=true}
\usepackage{graphicx}
\usepackage[usenames,dvipsnames]{color}
\def\mbunsc{\blank{0.07in}}
%\def\RE{$\mathrm{Re}\,$}
%\def\IM{$\mathrm{Im}\,$}
\def\RE{$\mbox{Re}\,$}
\def\IM{$\mbox{Im}\,$}
\def\fb#1{\framebox{\parbox[t]{6.5in}{#1}}}
\def\sfb#1{\framebox{\parbox[t]{6.0in}{#1}}}
\def\defn#1{\fb{{\bf Definition}: #1}}
\def\sdefn#1{\sfb{{\bf Definition}: #1}}
\def\thm#1{\fb{{\bf Theorem}: #1}}
\def\sthm#1{\sfb{{\bf Theorem}: #1}}
\def\eg#1{{\bf Example}: #1}
\def\egs#1{{\bf Examples}: #1}
\def\newt{\vspace{0.2in}\ni}
\setlength{\fboxsep}{5pt}
%\def\vdotprod{{\ascii\BEL}}
\def\vdotprod{\,\mbox{{\Large $\cdot$}}\,}
\def\vcrossprod{\times}


\begin{document}

\pagestyle{fancy}
\fancyhf{}
\lhead{MATH 162---Framework for \lecdate}
\chead{}
\rhead{\topic}
\lfoot{}
\cfoot{\thepage}
\rfoot{}
\renewcommand{\headrulewidth}{0.4pt}
\renewcommand{\footrulewidth}{0.4pt}

\thispagestyle{empty}

\begin{center}
  \Large{MATH 162: Calculus II} \\
  \large{Framework for \lecdate} \\
  \large{\topic}
\end{center}

\vs{0.2in}
\ni
{\bf Today's Goal}: To understand parametrized curves and
their derivatives.

\vspace{0.3in}
\ni
In yesterday's lab, we called a set of continuous functions
over a common interval $I$
\begin{equation}
  \begin{array}{l} x = x(t), \\ y = y(t), \\ z = z(t), \end{array}
	\quad t \in I, \label{paramEqns}
\end{equation}
a {\em parametrized curve}.
($I$ may be a finite interval, like $I = [a,b]$, or one of infinite
length.)  Another name for a parametrized
curve is {\em path}, as one may think of tracing out the location
of a particle $(x(t), y(t), z(t))$ at various $t$-values in $I$.

\vspace{0.1in}
\ni
\subsection*{Equations of lines in space}

Whereas lines in the $xy$-plane may be characterized by
a slope, lines in $xyz$-space are most
easily characterized by a vector that is parallel to the
line in question.  Since there are infinitely many parallel
vectors, there are infinitely many ways to describe a given
line.  Say we want the line passing through the point
$P = (x_0,y_0,z_0)$ parallel to the vector
$\mbv = v_1 \mbi + v_2 \mbj + v_3 \mbk$.
We describe it parametrically.  We might arbitrarily decide
to associate the point $P$ with the parameter value $t=0$,
and integer values of $t$ correspond to integer leaps of
length $|\mbv|$:
$$ \begin{array}{l} x = x_0 + v_1 t, \\ y = y_0 + v_2 t, \\ z = z_0 + v_3 t,
  \end{array} \quad -\infty < t < \infty. $$

\newt
\eg{}
Find 3 possible parametrizations of the line through
$(2,-1,4)$ in the direction of $\mbv = -\mbi + 2\mbj -2\mbk$.
Make one of these parametrizations be by arc length.

\subsection*{The position vector}

One might take the functions (\ref{paramEqns}) and create from
them a {\em vector function}, with $x(t)$, $y(t)$, and
$z(t)$ as {\em component functions}:
$$ \mbr(t) \;=\; x(t)\mbi + y(t)\mbj + z(t)\mbk. $$
Following the idea that the path (\ref{paramEqns}) describes the
locations of a moving particle, $\mbr(t)$ is often called
a {\em position vector}---that is, when drawn in {\em standard
position} (i.e., with its initial point at the origin), the
terminal point of $\mbr(t)$ moves so as to trace out the curve.

\newt
\eg{}
{\bf Equation of a line in space, vector form}.
For the line passing through the point $P = (x_0,y_0,z_0)$
parallel to the vector $\mbv = v_1 \mbi + v_2 \mbj + v_3 \mbk$,
we have the vector form
$$ \mbr(t) \;=\; (x_0 + v_1 t)\mbi + (y_0 + v_2 t)\mbj
  + (z_0 + v_3 t) \mbk. $$

\newt
\subsection*{Limits and continuity of vector functions}

While the following definition is not identical to the one
given in the text, the two are logically equivalent.\\[8pt]
\defn{
Let $\mbr(t) = x(t)\mbi + y(t)\mbj + z(t)\mbk$ (so the component
functions of $\mbr(t)$ are $x(t)$, $y(t)$ and $z(t)$).  We say
that
$$ \lim_{t\rightarrow t_0} \mbr(t) \;=\; \mbL \;=\;
	L_1 \mbi + L_2 \mbj + L_3 \mbk $$
precisely when each corresponding limit of the component functions
$$ \lim_{t\rightarrow t_0} x(t) \;=\; L_1, \qquad
	\lim_{t\rightarrow t_0} y(t) \;=\; L_2, \qquad\mbox{and}\qquad
	\lim_{t\rightarrow t_0} z(t) \;=\; L_3 $$
holds. \\[10pt]
We say that $\mbr(t)$ is continuous at $t=t_0$ precisely when
each of the component functions $x(t)$, $y(t)$ and $z(t)$ are
continuous at $t=t_0$.}

\newt
\eg{}
The vector function $\mbr(t) = t/(t-1)^2 \mbi + (\ln t)\mbj$
is continuous at all points $t$ where its component functions
$x(t) = t/(t-1)^2$ and $y(t) = \ln t$ are continuous---that is
for $t > 0$.  Thus, $\lim_{t\rightarrow t_0} \mbr(t)$ exists
whenever $t_0 > 0$.

\newt
\subsection*{Derivatives of vector functions}
\defn{A vector function $\mbr(t) = x(t)\mbi + y(t)\mbj + z(t)\mbk$
is {\em differentiable at} $t$ if the limit
$$ \mbr'(t) \;:=\; \lim_{\Delta t\rightarrow 0}
  \frac{\mbr(t+\Delta t) - \mbr(t)}{\Delta t}$$
exists.}

\newt
Notes:
\bi
\item
  An equivalent definition to the one above would be that
  $\mbr(t)$ is differentiable at a given $t$-value precisely when
  each of its component functions $x(t)$, $y(t)$ and $z(t)$ are
  differentiable there.  When this is so, we have
  $$ \frac{d\mbr}{dt} \;=\; x'(t)\mbi + y'(t)\mbj + z'(t)\mbk. $$
\item
  If a vector function $\mbr(t)$ is differentiable, then the
  derivative $\mbr'(t)$ is itself another vector function, which
  may be differentiable as well.  When this is so, we have
  $$ \frac{d^2 \mbr}{d^2 t} \;=\; x''(t)\mbi + y''(t)\mbj
	+ z''(t)\mbk. $$
\item
  If the position vector function $\mbr(t)$ is differentiable,
  then $d\mbr/dt$ is the corresponding {\em velocity} vector
  function.  What we call {\em speed} is actually the length
  $|d\mbr/dt|$ of the velocity function.\\[10pt]
  If $d\mbr/dt$ is differentiable, then we call $d^2\mbr/dt^2$
  the {\em acceleration} vector function.
\item
  One check that we have defined dot and cross products between
  vectors in a useful fashion is whether they obey ``product
  rules.''  In fact, all of the rules for differentiation that
  hold for scalar functions, and are appropriate to apply to
  vector functions, still hold:
  \be
  \item
	$\ds{\frac{d}{dt}\,\mbC \;=\; \0} \qquad$
	(constant function rule)\\[10pt]
  \item
	$\ds{\frac{d}{dt}[c \mbu(t)] \;=\; c\mbu'(t)} \qquad$
	(constant multiple rule)\\[10pt]
  \item
	$\ds{\frac{d}{dt}[f(t) \mbu(t)] \;=\; f'(t)\mbu(t) + f(t)\mbu'(t)} \qquad$
	(product of scalar and vector fn.)\\[10pt]
  \item
	$\ds{\frac{d}{dt}\left[\frac{\mbu(t)}{f(t)}\right] \;=\;
	  \frac{f'(t)\mbu(t) - f(t)\mbu'(t)}{[f(t)]^2}} \qquad$
	(quotient of vector and scalar fn.)\\[10pt]
  \item
	$\ds{\frac{d}{dt}[\mbu(t) \pm \mbv(t)] \;=\; \mbu'(t) \pm \mbv'(t)} \qquad$
	(sum and difference rules)\\[10pt]
  \item
	$\ds{\frac{d}{dt}[\mbu\vdotprod\mbv] \;=\; \mbu'(t)\vdotprod\mbv(t)
	  + \mbu(t)\vdotprod\mbv'(t)} \qquad$ (dot product rule)\\[10pt]
  \item
	$\ds{\frac{d}{dt}[\mbu\vcrossprod\mbv] \;=\; \mbu'(t)\vcrossprod\mbv(t)
	  + \mbu(t)\vcrossprod\mbv'(t)} \qquad$ (cross product rule)\\[10pt]
  \item
	$\ds{\frac{d}{dt}[\mbu(f(t))] \;=\; f'(t)\mbu'(f(t))} \qquad$
	(chain rule)
  \ee
\ei

\end{document}

