\documentclass[12pt,fleqn]{article}

\newif\ifcomplete
\completetrue
%\completefalse
\def\topic{Polar Coordinates}
\def\lecdate{Thurs., Apr.~19}

\oddsidemargin=0in
\textwidth=6.5in
\topmargin=-0.5in
\textheight=9.0in

\input mathMacros.tex
\usepackage{amsmath}
\usepackage{ascii}
%\usepackage{wasysym}
\usepackage{fancyhdr}
\usepackage{/Users/scofield/Library/texmf/latex/timestamp}
\usepackage[pdftex]{hyperref}
\hypersetup{pdfpagemode=None,colorlinks=true}
\usepackage{graphicx}
\usepackage[usenames,dvipsnames]{color}
\def\mbunsc{\blank{0.07in}}
\def\pref#1{(\ref{#1})}
%\def\RE{$\mathrm{Re}\,$}
%\def\IM{$\mathrm{Im}\,$}
\def\RE{$\mbox{Re}\,$}
\def\IM{$\mbox{Im}\,$}
\def\fb#1{\framebox{\parbox[t]{6.5in}{#1}}}
\def\sfb#1{\framebox{\parbox[t]{6.0in}{#1}}}
\def\defn#1{\fb{{\bf Definition}: #1}}
\def\sdefn#1{\sfb{{\bf Definition}: #1}}
\def\thm#1{\fb{{\bf Theorem}: #1}}
\def\sthm#1{\sfb{{\bf Theorem}: #1}}
\def\eg#1{{\bf Example}: #1}
\def\egs#1{{\bf Examples}: #1}
\def\egsc#1{{\bf Examples #1:}}
\def\newt{\vspace{0.2in}\ni}
\setlength{\fboxsep}{5pt}
%\def\vdotprod{{\ascii\BEL}}
\def\vdotprod{\,\mbox{{\Large $\cdot$}}\,}
\def\vcrossprod{\times}


\begin{document}

\pagestyle{fancy}
\fancyhf{}
\lhead{Framework for \lecdate}
\chead{}
\rhead{\topic}
\lfoot{}
\cfoot{\thepage}
\rfoot{}
\renewcommand{\headrulewidth}{0.4pt}
\renewcommand{\footrulewidth}{0.4pt}

\thispagestyle{empty}

\begin{center}
  \Large{MATH 162: Calculus II} \\
  \large{Framework for \lecdate} \\
  \large{\topic}
\end{center}

\vs{0.2in}
\ni
{\bf Today's Goal}:
To understand the use of polar coordinates for specifying
locations on the plane.

\vspace{0.15in}
\ni
{\bf Important Note}: In conjunction with this framework,
you should look over Section 9.1 of your text.

\vspace{0.15in}
\subsection*{Coordinate Systems for the Plane}
\be
\item {\bf Rectangular coordinates}
  Coordinate pair $(x,y)$ indicates how far one must travel
  in two perpendicular directions to arrive at point.

\item {\bf Polar coordinates}
  Coordinate pair $(r,\theta)$ indicates {\em signed
  distance} and {\em bearing}
  \bi
  \item
	The $r$ value (signed distance) is written first, followed
	by $\theta$.
  \item
	The {\em bearing} is an angle with the positive horizontal axis,
	with positive angles taken in the counterclockwise direction.
  \item
	Any polar coordinate pair $(r,\theta)$ with $r = 0$
	specifies the origin.
  \item
	Each point has infinitely many specifications.  For instance,
	$$ \dots = (-1,-\pi) = (1,0) = (-1,\pi) = (1,2\pi) = \ldots. $$
	In particular, for any point other than the origin, there
	are infinitely-many representations $(r,\theta)$ in polar
	coordinates with $r > 0$, and infinitely-many with $r < 0$.
  \ei
\ee

\subsection*{Relationships between Rectangular/Polar Coordinates}
\bi
\item {\bf Rectangular to polar}: $\;$
  If a point $(x,y)$ is specified in rectangular coordinates,
  it has a corresponding polar representation $(r,\theta)$
  determined by
  $$ r \;=\; \sqrt{x^2 + y^2}, \qquad
	\sin\theta \;=\; \frac{y}{\sqrt{x^2 + y^2}} \quad\mbox{and}\quad
	\cos\theta \;=\; \frac{x}{\sqrt{x^2 + y^2}}. $$
\item {\bf Polar to rectangular}: $\;$
  If a point $(r,\theta)$ is specified in rectangular
  coordinates, it has a corresponding polar representation
  $(x,y)$ determined by
  $$ x \;=\; r\cos\theta \qquad\mbox{and}\qquad y \;=\; r\sin\theta. $$
\ei

\subsection*{When Polar Coordinates Are Useful}
Using the ``rectangular to polar'' conversion above, equations
in $x$ and $y$ (and the curves that correspond to them) may be
expressed as polar equations (equations involving polar coordinates).
Often (though not always), what was a simple equation in rectangular
coordinates is uglier in polar coordinates.  Nevertheless, even
when this is the case, integration of particular functions over
particular regions is sometimes more easily carried out in polar
form than in rectangular.  (See examples of this in the framework
for Apr.~23.) \\[10pt]
\egsc{of curves in both forms}
\be
\item {\bf Circles centered at the origin}.\\[8pt]
  \mbox{}\hspace{0.1in}
  Rectangular form:
  $\ds{ x^2 + y^2 \;=\; a^2}$
  \hspace{0.5in}
  Polar form:
  $\ds{ r \;=\; \pm a}$

\item
  {\bf Circles centered on coordinate axis with point
  of tangency at the origin}.\\[8pt]
  \mbox{}\hspace{0.1in}
  Rectangular:
  $\ds{ x^2 + (y-a)^2 \;=\; a^2 }$
  \hspace{0.5in}
  Polar:
  $\ds{ r \;=\; \pm 2a\sin\theta}$

\item {\bf Certain ellipses}.\\[8pt]
  \mbox{}\hspace{0.1in}
  Rectangular:
  $\ds{ \frac{x^2}{a^2} + \frac{y^2}{b^2} \;=\; 1 }$
  \hspace{0.5in}
  Polar:
  $\ds{ r = \frac{b}{\sqrt{1 - \epsilon^2 \cos^2\theta}}, \quad\mbox{where}\;
	\epsilon = \sqrt{1 - \frac{b^2}{a^2}} }$

\item {\bf Lines through the origin}.\\[8pt]
  \mbox{}\hspace{0.1in}
  Rectangular:
  $\ds{ y \;=\; mx }$
  \hspace{0.5in}
  Polar:
  $\ds{ \theta = c, \quad\mbox{where} \; c = \arctan m }$

\item {\bf Horizontal and vertical lines}.\\[8pt]
  \mbox{}\hspace{0.1in}
  Rectangular:
  $\ds{ y \;=\; b }$
  \hspace{0.5in}
  Polar:
  $\ds{ r \;=\; b\csc\theta }$
\ee

\subsection*{Polar Functions}
For equations in rectangular coordinates, we often express
$y$ as a function of $x$ (i.e., treat $x$ as independent)
when this is possible.
Similarly, when it is possible to make $\theta$ the
independent variable (i.e., when a polar equation may be
solved for $r$), we tend to prefer doing so, writing
$r = f(\theta)$.  The polar forms in examples 1--3 and 5
above were expressed this way.\\[10pt]
\egsc{of families of polar curves of some interest}\\[3pt]
Note: If you place your graphing calculator in {\em polar}
mode, you should be able to plot any polar curve written
in the form $r = f(\theta)$.  See these and other polar
curves at
\href{http://mathdemos.gcsu.edu/mathdemos/family_of_functions/polar_gallery.html}{this link}.
\be
\item {\bf Cardioids}: $\;$
  $\ds{r = a(1 \pm \cos\theta)}
	\quad\mbox{or}\quad
	\ds{r = a(1 \pm \sin\theta)}$
\item {\bf Lemniscates}: $\;$
  $\ds{r^2 = a^2\cos(2\theta)},
  \quad \ds{r^2 = a^2\sin(2\theta)}, \quad$ etc.
\item {\bf Lima\c{c}ons}: $\;$
  $\ds{r = a + b\cos\theta}
	\quad\mbox{or}\quad
	\ds{r = a + b \sin\theta}$
\item {\bf Rose curves}: $\;$
  $\ds{r = a\cos(b\theta)}
	\quad\mbox{or}\quad
	\ds{r = a\sin(b\theta)}$
\item {\bf Spirals}: $\;$
  $\ds{r = a \theta},
  \quad \ds{r = e^{a \theta}}, \quad$ etc.
\ee

\end{document}

