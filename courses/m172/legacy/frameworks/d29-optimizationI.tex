\documentclass[12pt,fleqn]{article}

\newif\ifcomplete
\completetrue
%\completefalse
\def\topic{Unconstrained Optimization of Functions of 2 Variables}
\def\lecdate{Wed., Apr.~4}

\oddsidemargin=0in
\textwidth=6.5in
\topmargin=-0.5in
\textheight=9.0in

\input mathMacros.tex
\usepackage{amsmath}
\usepackage{ascii}
%\usepackage{wasysym}
\usepackage{fancyhdr}
\usepackage{/Users/scofield/Library/texmf/latex/timestamp}
\usepackage[pdftex]{hyperref}
\hypersetup{pdfpagemode=None,colorlinks=true}
\usepackage{graphicx}
\usepackage[usenames,dvipsnames]{color}
\def\mbunsc{\blank{0.07in}}
\def\pref#1{(\ref{#1})}
%\def\RE{$\mathrm{Re}\,$}
%\def\IM{$\mathrm{Im}\,$}
\def\RE{$\mbox{Re}\,$}
\def\IM{$\mbox{Im}\,$}
\def\fb#1{\framebox{\parbox[t]{6.5in}{#1}}}
\def\sfb#1{\framebox{\parbox[t]{6.0in}{#1}}}
\def\defn#1{\fb{{\bf Definition}: #1}}
\def\sdefn#1{\sfb{{\bf Definition}: #1}}
\def\thm#1{\fb{{\bf Theorem}: #1}}
\def\sthm#1{\sfb{{\bf Theorem}: #1}}
\def\eg#1{{\bf Example}: #1}
\def\egs#1{{\bf Examples}: #1}
\def\newt{\vspace{0.2in}\ni}
\setlength{\fboxsep}{5pt}
%\def\vdotprod{{\ascii\BEL}}
\def\vdotprod{\,\mbox{{\Large $\cdot$}}\,}
\def\vcrossprod{\times}


\begin{document}

\pagestyle{fancy}
\fancyhf{}
\lhead{Fmwk.~for \lecdate}
\chead{}
\rhead{\topic}
\lfoot{}
\cfoot{\thepage}
\rfoot{}
\renewcommand{\headrulewidth}{0.4pt}
\renewcommand{\footrulewidth}{0.4pt}

\thispagestyle{empty}

\begin{center}
  \Large{MATH 162: Calculus II} \\
  \large{Framework for \lecdate} \\
  \large{\topic}
\end{center}

\vs{0.2in}
\ni
{\bf Today's Goal}:
To be able to locate and classify local extrema for functions
of two variables.

\vspace{0.3in}
\ni
\defn{Suppose the domain of $f(x,y)$ includes the point $(a,b)$.
\be
\item
  $f(a,b)$ is called a {\em local maximum} (or {\em relative maximum})
  value of $f$ if $f(a,b) \ge f(x,y)$ for all points from $\dom(f)$
  contained in some open disk (an open disk of some positive, though
  perhaps quite small, radius) centered at $(a,b)$.
\item
  $f(a,b)$ is called a {\em local minimum} (or {\em relative miminum})
  value of $f$ if $f(a,b) \le f(x,y)$ for all points from $\dom(f)$
  contained in some open disk centered at $(a,b)$.
\ee
}

\vs{0.1in}
\ni
Remarks:
\bi
\item
  As with functions of a single variable (think of the absolute value
  function), local extrema (maxima or minima) of functions $f$ of two
  variables may occur at points where $f$ is not differentiable.

\item
  When an extremum occurs at an interior point $(a,b)$ of $\dom(f)$
  where $f$ is differentiable,
  one would expect $f$ to have a horizontal tangent plane there.
  The equation for the tangent plane to $z = f(x,y)$ at a point $(x_0,y_0)$
  where $f$ is differentiable is
  $$ f_x(x_0,y_0) (x-x_0) + f_y(x_0,y_0) (y-y_0) - (z-z_0) \;=\; 0, $$ or
  $$ z \;=\; z_0 + f_x(x_0,y_0) (x-x_0) + f_y(x_0,y_0) (y-y_0), $$
  while the equation of a horizontal plane (one parallel to the $xy$-plane
  is) $z =$ constant.  We may, therefore, conclude:\\[14pt]
  \sthm{
  If $f(x,y)$ has a local extremum at an interior point $(a,b)$ of
  $\dom(f)$, and if the partial derivatives of $f$ exist there, then
  $$ f_x(a,b) = 0 \qquad\mbox{and}\qquad f_y(a,b) = 0. $$
% or, equivalently
% $$ \vec\nabla f(a,b) = \0. $$
  }\\[14pt]
% Said another way, when an extremum occurs at an interior point of
% the domain, there should be no (single) direction of maximum increase
% (nor maximum decrease).
  This motivates the following definition.\\[14pt]
  \sdefn{Let $f$ be a function of two variables.
  An interior point of $\dom(f)$ where
  \be
  \item[(i)]
	both $f_x$ and $f_y$ are zero, or
  \item[(ii)]
	at least one of $f_x$, $f_y$ does not exist
  \ee
  is called a {\em critical point} of $f$.
  }
\ei

\newpage
\subsection*{Classifying Critical Points}

Just as with functions of one variable, not all critical points
of $f(x,y)$ correspond to a local extremum.  On pp.~757--759 of
the text, Figures 12.37 and 12.41 depict situations in which $(0,0)$
is a critical point corresponding to an extremum; Figure 12.40 depicts
situations in which $(0,0)$ is the location of a saddle point.\\[18pt]
\defn{Suppose $f(x,y)$ is a differentiable function with critical
point $(a,b)$.  If every open disk centered at $(a,b)$ contains
both domain points $(x,y)$ for which $f(x,y) > f(a,b)$
and domain points $(x,y)$ for which $f(x,y) < f(a,b)$, then
$f$ is said to have a {\em saddle point} at $(a,b)$.
}\\[18pt]
With functions of a single variable, we have several tests (the
First Derivative Test and the Second Derivative Test) for determining
when a critical point corresponds to a local extremum.  The following
theorem provides a test for those critical points of type (i) for
which $f$ is twice continuously differentiable throughout a disk
surrounding the critical point.\\[18pt]
\thm{
Suppose that $f(x,y)$ and its first and 2nd partial derivatives
are continuous throughout a disk centered at $(a,b)$, and that
$\vec\nabla f(a,b) = \0$.  Let $D$ be given by the following
two-by-two determinant:
$$ D(x,y) \;:=\; \left|\begin{array}{cc}
	f_{xx}(x,y) & f_{xy}(x,y) \\ f_{xy}(x,y) & f_{yy}(x,y)
  \end{array}\right| \;=\; f_{xx}f_{yy} - f_{xy}^2. $$
Then
\be
\item[(i)]
  $f$ has a local maximum at $(a,b)$ if $f_{xx}(a,b) < 0$
  and $D(a,b) > 0$.
\item[(ii)]
  $f$ has a local minimum at $(a,b)$ if $f_{xx}(a,b) > 0$
  and $D(a,b) > 0$.
\item[(iii)]
  $f$ has a saddle point at $(a,b)$ if $D(a,b) < 0$.
\ee
If $D(a,b) = 0$, or if $D(a,b) > 0$ and $f_{xx}(a,b) = 0$,
then this test fails to classify the critical point $(a,b)$.
}\\[18pt]
\egs{}
\bi
\item[]
% $\ds{f(x,y) \;=\; 9 - 2x + 4y - x^2 - 4y^2}$
  $\ds{f(x,y) \;=\; x^3 y + 12x^2 - 8y}$
\vs{0.2in}
\item[]
% $\ds{f(x,y) \;=\; e^{4y-x^2-y^2}}$
  $\ds{f(x,y) \;=\; \frac{x^2 y^2 - 8x + y}{xy}}$
\vs{0.2in}
\item[]
  $\ds{f(x,y) \;=\; xy(1-x-y)}$
\ei

\end{document}

