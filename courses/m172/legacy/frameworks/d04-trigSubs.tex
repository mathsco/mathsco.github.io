\documentclass[12pt,fleqn]{article}

\newif\ifcomplete
\completetrue
%\completefalse
\def\topic{Trigonometric Substitutions}
\def\lecdate{Fri., Feb.~2}

\oddsidemargin=0in
\textwidth=6.5in
\topmargin=-0.5in
\textheight=9.0in

\input mathMacros.tex
\usepackage{amsmath}
\usepackage{fancyhdr}
\usepackage{/Users/scofield/Library/texmf/latex/timestamp}
\usepackage[pdftex]{hyperref}
\hypersetup{pdfpagemode=None,colorlinks=true}
\usepackage{graphicx}
\usepackage[usenames,dvipsnames]{color}
\def\unsc{\blank{0.07in}}
%\def\RE{$\mathrm{Re}\,$}
%\def\IM{$\mathrm{Im}\,$}
\def\RE{$\mbox{Re}\,$}
\def\IM{$\mbox{Im}\,$}


\begin{document}

\pagestyle{fancy}
\fancyhf{}
\lhead{MATH 162---Framework for \lecdate}
\chead{}
\rhead{\topic}
\lfoot{}
\cfoot{\thepage}
\rfoot{}
\renewcommand{\headrulewidth}{0.4pt}
\renewcommand{\footrulewidth}{0.4pt}

\thispagestyle{empty}

\begin{center}
  \Large{MATH 162: Calculus II} \\
  \large{Framework for \lecdate} \\
  \large{\topic}
\end{center}

\vs{0.2in}
\bi
\item a technique used in integration
\item
  a \textit{pullback} substitution: we set $x = g(\theta)$, replacing
  the variable of integration (here assumed to be $x$) with a more
  complicated expression.  Prior to this, our substitutions have been
  of \textit{push-forward} type, where $u = h(x)$ generally ``simplifies"
  the integrand, wrapping some of the complexities of the expression into
  the new variable $u$.
\item
  useful most often when integrands involve $(a^2 + u^2)^m$,
  $(a^2 - u^2)^m$ or $(u^2 - a^2)^m$, where $a$ and $m$ are
  constants.  $m$ is often, but not exclusively, equal to $1/2$. \\
  Note: To get one of these forms, often completion of a square is required.
\item
  The usual substitutions \\
  \includegraphics[width=6in]{figs/trigSubsAndTriangles.png}
\ei

\ni
Some examples:
\be
\item $\ds{\int \frac{dx}{x^2 \sqrt{9 - x^2}}}$
\vs{1in}
\item $\ds{\int\frac{dx}{(x^2+1)^{3/2}}}$
\vs{1in}
\item $\ds{\int\frac{x}{\sqrt{x^2-3}}\,dx}$
\ee

\end{document}

