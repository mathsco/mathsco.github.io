\documentclass[12pt,fleqn]{article}

\newif\ifcomplete
\completetrue
%\completefalse
\def\topic{Triple Integrals, Rectangular Coordinates}
\def\lecdate{Tues., Apr.~17}

\oddsidemargin=0in
\textwidth=6.5in
\topmargin=-0.5in
\textheight=9.0in

\input mathMacros.tex
\usepackage{amsmath}
\usepackage{ascii}
%\usepackage{wasysym}
\usepackage{fancyhdr}
\usepackage{/Users/scofield/Library/texmf/latex/timestamp}
\usepackage[pdftex]{hyperref}
\hypersetup{pdfpagemode=None,colorlinks=true}
\usepackage{graphicx}
\usepackage[usenames,dvipsnames]{color}
\def\mbunsc{\blank{0.07in}}
\def\pref#1{(\ref{#1})}
%\def\RE{$\mathrm{Re}\,$}
%\def\IM{$\mathrm{Im}\,$}
\def\RE{$\mbox{Re}\,$}
\def\IM{$\mbox{Im}\,$}
\def\fb#1{\framebox{\parbox[t]{6.5in}{#1}}}
\def\sfb#1{\framebox{\parbox[t]{6.0in}{#1}}}
\def\defn#1{\fb{{\bf Definition}: #1}}
\def\sdefn#1{\sfb{{\bf Definition}: #1}}
\def\thm#1{\fb{{\bf Theorem}: #1}}
\def\sthm#1{\sfb{{\bf Theorem}: #1}}
\def\eg#1{{\bf Example}: #1}
\def\egs#1{{\bf Examples}: #1}
\def\newt{\vspace{0.2in}\ni}
\setlength{\fboxsep}{5pt}
%\def\vdotprod{{\ascii\BEL}}
\def\vdotprod{\,\mbox{{\Large $\cdot$}}\,}
\def\vcrossprod{\times}


\begin{document}

\pagestyle{fancy}
\fancyhf{}
\lhead{Framework for \lecdate}
\chead{}
\rhead{\topic}
\lfoot{}
\cfoot{\thepage}
\rfoot{}
\renewcommand{\headrulewidth}{0.4pt}
\renewcommand{\footrulewidth}{0.4pt}

\thispagestyle{empty}

\begin{center}
  \Large{MATH 162: Calculus II} \\
  \large{Framework for \lecdate} \\
  \large{\topic}
\end{center}

\vs{0.2in}
\ni
{\bf Today's Goal}:
To be able to set up and evaluate triple integrals.

\vspace{0.15in}
\ni
{\bf Important Note}: In conjunction with this framework,
you should look over Section 13.5 of your text.

\vspace{0.15in}
\subsection*{Defining Triple Integrals}
Suppose
\bi
\item
  $D$ is a ``nice'' bounded region in $3$-dimensional space.
\item
  We subdivide $D$, creating a partition $P$ of $D$, where
  $P$ consists of $n$ ``boxes'' wholly contained in $D$.
\item
  In the $k$th box ($1 \le k \le n$), we choose a point $(x_k, y_k, z_k)$.
\ei
We then look at sums of the form
$$ \sum_{k=1}^n f(x_k,y_k,z_k)\Delta V_k, $$
where $\Delta V_k$ denotes the volume of the $k$th box.\\[10pt]
As with Riemann sums over partitions of regions of the plane, there
are many functions and regions $D$ for which the limit
$$ \lim_{\|P\|\rightarrow 0} \sum_{k=1}^n f(x_k,y_k,z_k)\Delta V_k $$
exists, in which case we say that $f$ is {\em integrable} over $D$.
This limit is denoted by
$$ \iiint\limits_D f(x,y,z)\,dV, $$
read as the {\em triple integral} of $f$ over $D$.

\newpage
\section*{Comparisons to Double Integrals}
\be
\item {\bf Evaluation}.\\[8pt]
  {\bf Double integrals:}
  \bq
  Here our principle tool is Fubini's Theorem.  We have two cases.\\[8pt]
  {\bf Case:} $\ds{\iint\limits_R g(x,y)\,dA = \int_a^b \int_{g_1(x)}^{g_2(x)}
	g(x,y)\,dy \, dx}$.
  \bq
  \begin{minipage}[t]{5.5in}
	Here, $a$ and $b$ reflect the lowest and highest of a continuum of
	$x$-values encountered in the region $R$, while the lower and upper
	boundaries of $R$ may be identified as functions of $x$.
  \end{minipage}
  \eq
  {\bf Case:} $\ds{\iint\limits_R g(x,y)\,dA = \int_c^d \int_{h_1(x)}^{h_2(x)}
	g(x,y)\,dx \, dy}$.
  \bq
  \begin{minipage}[t]{5.5in}
	Here, $c$ and $d$ reflect the lowest and highest of a continuum of
	$y$-values encountered in the region $R$, while the left and right
	boundaries of $R$ may be identified as functions of $y$.
  \end{minipage}
  \eq
  \begin{minipage}[t]{5.9in}
  For most regions $R$, either case is applicable (sometimes one of
  the options requires a sum of integrals instead of a single one),
  meaning that the double integral may be written in either of
  {\em two orders} (i.e., either with $y$ as the inner integral, as
  in $dy\,dx$, or in the order $dx\,dy$).
  \end{minipage}
  \eq

  {\bf Triple integrals:}
  \bq
  Given a bounded region $D$ of 3-dimensional space, the triple integral
  $\iiint_D f(x,y,z)\,dV$ may be written as an iterated integral
  in {\em six different orders}.  Here are {\em two of the possibilities}:
  \bi
  \item
	$\ds{\iiint\limits_D f(x,y,z)\,dV = \int_c^d \int_{g_1(y)}^{g_2(y)}
	  \int_{h_1(y,z)}^{h_2(y,z)} f(x,y,z) \,dx \,dz \,dy}$.
	\bq
	  \begin{minipage}[t]{5.5in}
	  Here, $c$ and $d$ are the lowest and highest in a
	  continuum of $y$-values encountered as one passes
	  through the region $D$.  For any fixed $y \in [c,d]$,
	  we imagine a 2-dimensional (planar) region that results
	  from slicing through $D$ with a plane parallel to the
	  $xz$-plane.  This planar region has a starting and
	  ending $z$-value, given by $g_1(y)$ and and $g_2(y)$
	  respectively.  At the inner-most level (the inner-most
	  integral, which is in $x$), $y$ and $z$ are held fixed
	  while $x$ is allowed to vary.  The interval of possible
	  $x$-values starts at $h_1(y,z)$ and ends at $h_2(y,z)$.
	  \end{minipage}
	\eq
  \item
	$\ds{\iiint\limits_D f(x,y,z)\,dV = \int_r^s \int_{g_1(z)}^{g_2(z)}
	  \int_{h_1(x,z)}^{h_2(x,z)} f(x,y,z) \,dy \,dx \,dz}$.
	\bq
	  \begin{minipage}[t]{5.5in}
	  The explanation of our region is similar to the above,
	  but this time $r$ and $s$ represent lowest and highest
	  $z$-values encountered in $D$; for a fixed $z$,
	  $g_1(z)$ and $g_2(z)$ give lowest and highest $x$-values;
	  for both $x$ and $z$ fixed, $h_1(x,z)$ and $h_2(x,z)$
	  give lowest and highest $y$-values.
	  \end{minipage}
	\eq
  \ei
  \eq

\item {\bf Interpretations}.
  \be
  \item {\bf Areas, volumes and higher}.\\[10pt]
	{\bf Double integrals:}
	\bq
	\begin{minipage}[t]{5.5in}
	When $g(x,y)$ is nonnegative, the double integral
	$\iint_R f(x,y)\,dA$ gives the volume under the surface
	$z = g(x,y)$ over the region $R$ of the $xy$-plane.
	If $g$ changes sign in the region $R$, then $\iint_R
	g(x,y)\,dA$ represents a {\em difference of volumes}.\\[10pt]
	A special case is when $g(x,y) \equiv 1$.
	As we have seen $\iint_R g(x,y)\,dA = \iint dA$ gives
	the {\em area} of $R$ (numerically equal to the volume
	under a surface over $R$ whose height is uniformly 1).
	\end{minipage}
	\eq

	{\bf Triple integrals:}
	\bq
	\begin{minipage}[t]{5.5in}
	When $f(x,y,z)$ is nonnegative, we can be sure that
	$\iiint_D f(x,y,z)\,dV$ is nonnegative as well.  But
	since the graph of $w = f(x,y,z)$ is 4-dimensional,
	we would have to think of this value as a type of
	4-dimensional volume (or difference of volumes, if
	$f$ changes sign in $D$).\\[10pt]
	When $f(x,y) \equiv 1$, then $\iiint_R f(x,y,z)\,dV
	= \iiint dV$ gives the {\em volume} of the 3-dimensional
	region $D$.
	\end{minipage}
	\eq

  \item {\bf Average values}.\\
	The average value of $g(x,y)$ over a region $R$ of the
	$xy$-plane was defined to be
	$\iint_R g(x,y)\,dA/\iint_R dA$.
	Similarly, we define the average value of
	$f(x,y,z)$ over a region $D$ of 3-dimensional space
	to be $\iiint_D f(x,y,z)\,dV/\iiint_D dV$.

  \item {\bf Density integrals}.
	When $g(x,y)$ gives the amount of a substance per unit
	area, then $\iint_R g(x,y)\,dA$ tallies the amount of
	that substance found in a region $R$ of the $xy$-plane.
	Similarly, when $f(x,y,z)$ gives the amount of a substance
	per unit volume, then $\iiint_D f(x,y,z)\,dV$ tallies the
	amount of that substance found in a region $D$ of 3D space.
  \ee
\ee


\vspace{0.2in}
\ni
\egs{}
\be
\item
  Evaluate $\iiint_D z\,dV$ over the region enclosed by the
  three coordinate planes and the plane $x + y + z = 1$.
%\item
% Write two other integrals whose value should be the same
% as above.
\item
  Find the average $z$-value in the region from problem 1.
\item
  Find limits of integration for $\iiint_D \sqrt{x^2 + z^2}
  \,dy \,dz \,dx$ where $D$ is the region bounded by the paraboloid
  $y = x^2 + z^2$ and the plane $y = 4$.
\item
  Write a triple integral for $f(x,y,z)$ over the region bounded
  by the ellipsoid $9x^2 + 4y^2 + z^2 = 1$.
\item
  What solid is it for which the iterated triple integral
  $\int_0^2 \int_0^{2-y} \int_0^{4-y^2} dz \,dx \,dy$
  gives its volume.  What do other iterated triple
  integrals for the same expression look like?
\ee

\end{document}

