\documentclass[12pt,fleqn]{article}

\newif\ifcomplete
\completetrue
%\completefalse
\def\topic{Cross Products}
\def\lecdate{Tues., Mar.~13}

\oddsidemargin=0in
\textwidth=6.5in
\topmargin=-0.5in
\textheight=9.0in

\input mathMacros.tex
\usepackage{amsmath}
\usepackage{ascii}
%\usepackage{wasysym}
\usepackage{fancyhdr}
\usepackage{/Users/scofield/Library/texmf/latex/timestamp}
\usepackage[pdftex]{hyperref}
\hypersetup{pdfpagemode=None,colorlinks=true}
\usepackage{graphicx}
\usepackage[usenames,dvipsnames]{color}
\def\mbunsc{\blank{0.07in}}
%\def\RE{$\mathrm{Re}\,$}
%\def\IM{$\mathrm{Im}\,$}
\def\RE{$\mbox{Re}\,$}
\def\IM{$\mbox{Im}\,$}
\def\fb#1{\framebox{\parbox[t]{6.5in}{#1}}}
\def\sfb#1{\framebox{\parbox[t]{6.0in}{#1}}}
\def\defn#1{\fb{{\bf Definition}: #1}}
\def\sdefn#1{\sfb{{\bf Definition}: #1}}
\def\thm#1{\fb{{\bf Theorem}: #1}}
\def\sthm#1{\sfb{{\bf Theorem}: #1}}
\def\eg#1{{\bf Example}: #1}
\def\egs#1{{\bf Examples}: #1}
\def\newt{\vspace{0.2in}\ni}
\setlength{\fboxsep}{5pt}
%\def\vdotprod{{\ascii\BEL}}
\def\vdotprod{\,\mbox{{\Large $\cdot$}}\,}
\def\vcrossprod{\times}


\begin{document}

\pagestyle{fancy}
\fancyhf{}
\lhead{MATH 162---Framework for \lecdate}
\chead{}
\rhead{\topic}
\lfoot{}
\cfoot{\thepage}
\rfoot{}
\renewcommand{\headrulewidth}{0.4pt}
\renewcommand{\footrulewidth}{0.4pt}

\thispagestyle{empty}

\begin{center}
  \Large{MATH 162: Calculus II} \\
  \large{Framework for \lecdate} \\
  \large{\topic}
\end{center}

\vs{0.2in}
\ni
{\bf Today's Goal}: To define the cross product and learn of some of its
properties and uses

\vspace{0.1in}
\subsection*{The Cross Product}

\defn{For nonzero, non-parallel 3D vectors $\mbu$ and $\mbv$, we define the
{\em cross product} of $\mbu$ and $\mbv$ to be
$$ \mbu \vcrossprod \mbv \;:=\; (|\mbu||\mbv|\sin\theta)\mbn, $$
where $\theta$ is the angle between $\mbu$ and $\mbv$, and
$\mbn$ is a unit vector perpendicular to both $\mbu$ and $\mbv$,
and in the direction determined by the ``right-hand rule.''\\[10pt]
If either $\mbu = \0$ or $\mbv = \0$, we define $\mbu\vcrossprod\mbv = \0$.
Similarly, if $\mbu$ and $\mbv$ are parallel, we take
$\mbu\vcrossprod\mbv = \0$.}

\newt
Notes:
\bi
\item
  There is no corresponding concept for 2D vectors.
\item
  The dot product between two vectors produces a scalar.
  The cross product of two vectors yields another vector.
\item
  Properties
  \be
  \item $\mbu\vcrossprod\mbv = -\mbv\vcrossprod\mbu$
  \item
	$\mbi\vcrossprod\mbj = \mbk$, $\;\;\mbj\vcrossprod\mbk = \mbi$,
	$\;\;\mbk\vcrossprod\mbi = \mbj$
  \item $(r\mbu)\vcrossprod (s\mbv) \;=\; (rs) (\mbu\vcrossprod\mbv)$
  \item
	The cross product is not associative!  This means that, in general,
	it is {\em not} the case that
	$$ (\mbu \vcrossprod \mbv) \vcrossprod \mbw \qquad\mbox{and}\qquad
		\mbu \vcrossprod (\mbv \vcrossprod \mbw) $$
	are equal.
  \ee
\item
  The cross product $\mbu\vcrossprod\mbv$ may be computed from
  the following symbolic determinant:
  $$ \mbu\vcrossprod\mbv \;=\; \left|\begin{array}{ccc}
	  \mbi & \mbj & \mbk \\ u_1 & u_2 & u_3 \\ v_1 & v_2 & v_3
	\end{array}\right| \;:=\;
	\left|\begin{array}{cc} u_2 & u_3 \\ v_2 & v_3 \end{array}\right|\mbi
	- \left|\begin{array}{cc} u_1 & u_3 \\ v_1 & v_3 \end{array}\right|\mbj
	+ \left|\begin{array}{cc} u_1 & u_2 \\ v_1 & v_2 \end{array}\right|\mbk, $$
  where $\mbu = \left<u_1, u_2, u_3\right>$ and
  $\mbv = \left<v_1, v_2, v_3\right>$.
\ei

\np
\ni
\subsection*{Applications}

\bi
\item
  $\mbr\vcrossprod\mbF$ is the {\em torque} vector resulting
  from a force $\mbF$ applied at the end of a lever arm $\mbr$.
\item
  $|\mbu\vcrossprod\mbv|$ (the length of the cross product
  $\mbu\vcrossprod\mbv$) is the area of a parallelogram
  determined by $\mbu$ and $\mbv$.
\item
  $|(\mbu\vcrossprod\mbv)\vdotprod\mbw|$ (the absolute value
  of the scalar $(\mbu\vcrossprod\mbv)\vdotprod\mbw$) is the
  volume of the parallelepiped determined by $\mbu$, $\mbv$
  and $\mbw$.
\item
  Finding normal vectors to planes. \\[14pt]
  \eg{}
  The vectors $\mbu = \left<1, 2, -1\right>$ and
  $\mbv = \left<-2,3,1\right>$
  \bi
  \item are not parallel,
  \item so they determine a family of parallel planes.
  \ei
  Find a vector that is normal to these planes.  Then determine
  an equation for the particular one of these planes passing
  through the point $(1,1,1)$.
\ei

\end{document}

