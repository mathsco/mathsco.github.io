\documentclass[12pt,fleqn]{article}

\newif\ifcomplete
\completetrue
%\completefalse
\def\topic{Numerical Integration}
\def\lecdate{Wed., Feb.~7}

\oddsidemargin=0in
\textwidth=6.5in
\topmargin=-0.5in
\textheight=9.0in

\input mathMacros.tex
\usepackage{amsmath}
\usepackage{fancyhdr}
\usepackage{/Users/scofield/Library/texmf/latex/timestamp}
\usepackage[pdftex]{hyperref}
\hypersetup{pdfpagemode=None,colorlinks=true}
\usepackage{graphicx}
\usepackage[usenames,dvipsnames]{color}
\def\unsc{\blank{0.07in}}
%\def\RE{$\mathrm{Re}\,$}
%\def\IM{$\mathrm{Im}\,$}
\def\RE{$\mbox{Re}\,$}
\def\IM{$\mbox{Im}\,$}


\begin{document}

\pagestyle{fancy}
\fancyhf{}
\lhead{MATH 162---Framework for \lecdate}
\chead{}
\rhead{\topic}
\lfoot{}
\cfoot{\thepage}
\rfoot{}
\renewcommand{\headrulewidth}{0.4pt}
\renewcommand{\footrulewidth}{0.4pt}

\thispagestyle{empty}

\begin{center}
  \Large{MATH 162: Calculus II} \\
  \large{Framework for \lecdate} \\
  \large{\topic}
\end{center}

\vs{0.2in}
\ni
\section*{Numerical approximations to definite integrals}
\bi
\item
  Riemann (rectangle) sums already give us approximations
  \bi
  \item[] Main types: left-hand, right-hand and midpoint rules
  \ei
\item
  Question: Why rectangles?
  \bi
  \item
	trapezoids
	\bi
	\item
	  \begin{minipage}[t]{3.5in}
	    Area of a trapezoid with bases $b_1$, $b_2$, height $h$
	  \end{minipage}
	  \begin{minipage}[t]{2in}
		\mbox{}

		\vspace{-0.41in}
		\includegraphics[width=1.5in]{figs/trapezoid.png}
	  \end{minipage}
	\item
	  Approximation to $\ds{\int_a^b f(x)\,dx}$ using $n$
	  steps all of width $\Delta x = (b-a)/n$ (Trapezoid Rule)
	\vspace{0.7in}
	\item
	  Remarkable fact: Trapezoid rule does not improve over
	  midpoint rule.
	\ei
  \item
	parabolic arcs
	\bi
	\item
	  \begin{minipage}[t]{3.0in}
		$\ds{\int_{-h}^h g(x)\,dx}$, when $g(x) = Ax^2 + Bx + C$
		is chosen to pass through $(-h, y_0)$, $(0,y_1)$ and $(h, y_2)$
	  \end{minipage}
	  \begin{minipage}[t]{2.5in}
		\mbox{}

		\vspace{-0.61in}
		\includegraphics[width=3.5in]{figs/simpson.png}
	  \end{minipage}
	\vspace{0.2in}
	\item
	  Approximation to $\ds{\int_a^b f(x)\,dx}$ using $n$ (even)
	  steps all of width $\Delta x$ (Simpson's Rule)
	\ei
  \ei
\vspace{0.5in}
\item Error bounds
  \bi
  \item No such thing available for a general integrand $f$
  \item Formulas (available when $f$ is sufficiently differentiable)
	\bi
	\item
	  {\bf Trapezoid Rule}.
	  Suppose $f''$ is continuous throughout $[a,b]$, and $|f''(x)| \le M$
	  for all $x\in [a,b]$.  Then the error $E_T$ in using the Trapezoid
	  rule with $n$ steps to approximate $\ds{\int_a^b f(x)\,dx}$ satisfies
	  $$ |E_T| \;\le\; \frac{M (b-a)^3}{12n^2}. $$
	\item
	  {\bf Simpson's Rule}.
	  Suppose $f^{(4)}$ is continuous throughout $[a,b]$, and
	  $|f^{(4)}(x)| \le M$ for all $x\in [a,b]$.  Then the error $E_S$
	  in using Simpson's rule with $n$ steps to approximate
	  $\ds{\int_a^b f(x)\,dx}$ satisfies
	  $$ |E_S| \;\le\; \frac{M (b-a)^5}{180n^4}. $$
	\ei
  \item Use
	\bi
	\item
	  For a given $n$, gives an upper bound on your error
	\item
	  If a desired upper bound on error is sought, may be used
	  to determine {\em a priori} how many steps to use
	\ei
  \ei
\ei

\end{document}

