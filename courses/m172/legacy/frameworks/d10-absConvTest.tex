\documentclass[12pt,fleqn]{article}

\newif\ifcomplete
\completetrue
%\completefalse
\def\topic{Absolute and Conditional Convergence}
\def\lecdate{Fri., Feb.~16}

\oddsidemargin=0in
\textwidth=6.5in
\topmargin=-0.5in
\textheight=9.0in

\input mathMacros.tex
\usepackage{amsmath}
\usepackage{fancyhdr}
\usepackage{/Users/scofield/Library/texmf/latex/timestamp}
\usepackage[pdftex]{hyperref}
\hypersetup{pdfpagemode=None,colorlinks=true}
\usepackage{graphicx}
\usepackage[usenames,dvipsnames]{color}
\def\unsc{\blank{0.07in}}
%\def\RE{$\mathrm{Re}\,$}
%\def\IM{$\mathrm{Im}\,$}
\def\RE{$\mbox{Re}\,$}
\def\IM{$\mbox{Im}\,$}


\begin{document}

\pagestyle{fancy}
\fancyhf{}
\lhead{MATH 162---Framework for \lecdate}
\chead{}
\rhead{\topic}
\lfoot{}
\cfoot{\thepage}
\rfoot{}
\renewcommand{\headrulewidth}{0.4pt}
\renewcommand{\footrulewidth}{0.4pt}

\thispagestyle{empty}

\begin{center}
  \Large{MATH 162: Calculus II} \\
  \large{Framework for \lecdate} \\
  \large{\topic}
\end{center}

\vs{0.2in}
\ni
\section*{$p$-Series Results Revisited}

\bi
\item
  Results we have shown: The series whose terms are all positive
  \begin{equation}
	\sum_{n=1}^\infty n^{-p} \;=\; 1 + 2^{-p} + 3^{-p} + 4^{-p} + \cdots
	\label{p-series}
  \end{equation}
  converges for $p > 1$, and diverges for $p \le 1$.  The series with
  alternating signs
  \begin{equation}
	\sum_{n=1}^\infty (-1)^{n-1}n^{-p} \;=\; 1 - 2^{-p} + 3^{-p} - 4^{-p}
	+ \cdots \label{alternatingP-series}
  \end{equation}
  converges for $p > 0$, and diverges for $p \le 0$.

\item
  The above results apply narrowly---only to series in the forms
  (\ref{p-series}) and (\ref{alternatingP-series}) respectively.
  Thus, nothing we have learned tells us whether
  $$ 1 + \frac{1}{2} - \frac{1}{3} + \frac{1}{4} - \frac{1}{5} - \frac{1}{6}
	+ \frac{1}{7} + \frac{1}{8} - \frac{1}{9} + \cdots $$
  converges.

\item
  The ``borderline'' case of (\ref{p-series}), the one with $p=1$,
  \begin{equation}
	\sum_{n=1}^\infty n^{-1} \;=\; 1 + \frac{1}{2} + \frac{1}{3}
	+ \cdots + \frac{1}{n} + \cdots \label{harmonicSeries}
  \end{equation}
  is divergent, and has been named the {\em harmonic series}.

\item
  When the convergence/divergence of a series $\sum a_n$ is known,
  then the convergence/divergence of certain modified forms of that
  series can be known as well.  In particular,
  \bi
  \item
	Any nonzero multiple of a series that converges (resp.~diverges)
	will also converge (resp.~diverge).  Thus,
	$$ \frac{1}{3} + \frac{1}{6} + \frac{1}{9} + \frac{1}{12} + \cdots
	  \;=\; \frac{1}{3}\,\sum_{n=1}^\infty n^{-1}, $$
	diverges, being a multiple of the harmonic series (\ref{harmonicSeries}).
  \item
	Suppose $\sum a_n$ is a series whose convergence/divergence is known.
	Any series which has the same ``tail'' as that of $\sum a_n$
	will converge (resp.~diverge) based on what $\sum a_n$ does.  For
	instance, since we know
	$$ \sum_{n=1}^\infty (-1)^{n-1} n^{-1/2} \;=\; 1 - \frac{1}{\sqrt 2}
	  + \frac{1}{\sqrt 3} - \frac{1}{2} + \frac{1}{\sqrt 5}
	  - \frac{1}{\sqrt 6} + \cdots $$
	converges, we can conclude
	$$ \sum_{n=4}^\infty (-1)^{n-1} n^{-1/2} \;=\; - \frac{1}{2}
	  + \frac{1}{\sqrt 5} - \frac{1}{\sqrt 6} + \cdots $$
	and
	$$ b_1 + b_2 + \cdots + b_{50} + \frac{1}{\sqrt 3} - \frac{1}{2}
	  + \frac{1}{\sqrt 5} - \frac{1}{\sqrt 6} + \cdots $$
	converge as well.  Here $b_1$, \ldots, $b_{50}$ is any arbitrary
	list of 50 numbers.  The important thing is not the values of
	these $b_j$'s, but that there are only finitely many (in this
	case, 50) of them.
  \ei
\ei

\section*{Absolute and Conditional Convergence}

\ni
{\bf Definition}: \; Let $\sum a_n$ be a convergent series.  If the
corresponding series $\sum |a_n|$ in which every term has been made
positive diverges, then the original series $\sum a_n$ is said to be
{\em conditionally convergent}.

\vspace{0.15in}
\ni
{\bf Example}: \; The series
$\ds{\sum_{n=1}^\infty (-1)^n n^{-1}} = 1 - 1/2 + 1/3 - 1/4 + \cdots$
is conditionally convergent.

\vspace{0.15in}
\ni
{\bf Definition}: Let $\sum a_n$ be a given series (i.e., one for which
the values of the terms $a_j$ are known).  If the corresponding
series $\sum |a_n|$ with all positive terms converges, then the original
series $\sum a_n$ is said to be {\em absolutely convergent}.

\vspace{0.15in}
\ni
{\bf Theorem (Absolute Convergence Test)}: All absolutely convergent
series are convergent.

\vspace{0.15in}
\ni
{\bf Example}: \; The series
$$ \frac{11}{3} + \frac{11}{6} - \frac{11}{12} + \frac{11}{24} - \frac{11}{48}
  - \frac{11}{96} - \frac{11}{192} + \cdots $$
is absolutely convergent, since
$$ \frac{11}{3} + \frac{11}{6} + \frac{11}{12} + \frac{11}{24} + \frac{11}{48}
  + \cdots \;=\; \sum_{n=0}^\infty \left(\frac{11}{3}\right)
  \left(\frac{1}{2}\right)^n $$
converges (being geometric, with $r = 1/2$).  By the absolute convergence
test, the original series converges as well.

\end{document}

