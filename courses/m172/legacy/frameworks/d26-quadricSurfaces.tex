\documentclass[12pt,fleqn]{article}

\newif\ifcomplete
\completetrue
%\completefalse
\def\topic{Quadric Surfaces}
\def\lecdate{Mon., Mar.~26}

\oddsidemargin=0in
\textwidth=6.5in
\topmargin=-0.5in
\textheight=9.0in

\input mathMacros.tex
\usepackage{amsmath}
\usepackage{ascii}
%\usepackage{wasysym}
\usepackage{fancyhdr}
\usepackage{/Users/scofield/Library/texmf/latex/timestamp}
\usepackage[pdftex]{hyperref}
\hypersetup{pdfpagemode=None,colorlinks=true}
\usepackage{graphicx}
\usepackage[usenames,dvipsnames]{color}
\def\mbunsc{\blank{0.07in}}
\def\pref#1{(\ref{#1})}
%\def\RE{$\mathrm{Re}\,$}
%\def\IM{$\mathrm{Im}\,$}
\def\RE{$\mbox{Re}\,$}
\def\IM{$\mbox{Im}\,$}
\def\fb#1{\framebox{\parbox[t]{6.5in}{#1}}}
\def\sfb#1{\framebox{\parbox[t]{6.0in}{#1}}}
\def\defn#1{\fb{{\bf Definition}: #1}}
\def\sdefn#1{\sfb{{\bf Definition}: #1}}
\def\thm#1{\fb{{\bf Theorem}: #1}}
\def\sthm#1{\sfb{{\bf Theorem}: #1}}
\def\eg#1{{\bf Example}: #1}
\def\egs#1{{\bf Examples}: #1}
\def\newt{\vspace{0.2in}\ni}
\setlength{\fboxsep}{5pt}
%\def\vdotprod{{\ascii\BEL}}
\def\vdotprod{\,\mbox{{\Large $\cdot$}}\,}
\def\vcrossprod{\times}


\begin{document}

\pagestyle{fancy}
\fancyhf{}
\lhead{MATH 162---Framework for \lecdate}
\chead{}
\rhead{\topic}
\lfoot{}
\cfoot{\thepage}
\rfoot{}
\renewcommand{\headrulewidth}{0.4pt}
\renewcommand{\footrulewidth}{0.4pt}

\thispagestyle{empty}

\begin{center}
  \Large{MATH 162: Calculus II} \\
  \large{Framework for \lecdate} \\
  \large{\topic}
\end{center}

\vs{0.2in}
\ni
{\bf Today's Goal}: To review how equations in three variables
are graphed, and to identify special graphs known as quadric
surfaces.

\vspace{0.3in}
\ni
{\bf What we already know}:
A 2nd-order polynomial in $x$ and $y$ takes the general form
$$ p(x,y) \;=\; Ax^2 + Bxy + Cy^2 + Dx + Ey + F. $$
Such a polynomial is called a {\em quadratic polynomial} (in $x$
and $y$).\\[20pt]
{\bf Some special cases}, usually treated in high school classes:
\bi
\item
  {\bf Case} $B = C = 0,\; A\ne 0, \; E\ne 0$:
  $\quad\ds{ p(x, y) \;=\; Ax^2 + Dx + Ey + F }$
  \bi
  \item The level sets of $p$ are parabolas.
  \item By symmetry of argument, the level sets are parabolas
	opening sideways when $A=B=0$, $C\ne 0$ and $D\ne 0$.
  \ei
\item
  {\bf Case} $B^2 < 4AC$:
% $\quad\ds{ p(x, y) \;=\; Ax^2 + Bxy + Cy^2 + Dx + Ey + F }$
  \bi
  \item Some level sets of $p$ are ellipses.
  \item When $B=0$ and $A=C$, the ellipses are actually circles.
  \ei
\item
  {\bf Case} $AC < 0$:
% $\quad\ds{ p(x, y) \;=\; Ax^2 + Bxy + Cy^2 + Dx + Ey + F }$\\[8pt]
  Almost all level sets of $p$ are hyperbolas.
\ei

\subsection*{Quadric Surfaces}
Similar to the above, a quadratic polynomial in $x$, $y$ and $z$ is
a 2nd-order polynomial having general form
\begin{equation}
  p(x,y,z) \;=\; Ax^2 + Bxy + Cy^2 + Dxz + Eyz + Fz^2 + Gx + Hy + Iz + J.
  \label{quadraticxyz}
\end{equation}
\bi
\item
  The graph of $p$ would require 4 dimensions, but the level sets
  of $p$ are {\em surfaces} in 3D.
\item
  The solutions of the equation $p(x,y,z) = k\;$ ($k$ a fixed number)
  coincide with the $k$-level surface for the quadratic function $p$.\\[8pt]
% For a fixed number $k$, the level surface equation is $p(x,y,z) = k$.\\[8pt]
  \sdefn{For a quadratic polynomial $p$ in the form \pref{quadraticxyz},
  the set of points $(x,y,z)$ which satisfy the level surface equation
  $p(x,y,z) = k$ is called a {\em quadric surface}.}
\item
  When the coefficient $F = 0$ and at least one of $D$, $E$ or $I$ is nonzero,
  the level surface equation may be manipulated algebraically to solve
  for $z$ as a function of $x$ and $y$.  In these cases, what we learned
  about graphing functions of 2 variables still applies.\\[18pt]
  \eg{}
% $\ds{9x^2 + y^2 - 4z = 1} \quad$ (elliptic paraboloid)
  $\ds{9x^2 - y^2 - 4z = 0} \quad$ (hyperbolic paraboloid)\\[18pt]
  By symmetry of argument, the level surface equation $p(x,y,z) = k$
  can be written as a function if
  \bi
  \item
	$A = 0$ and at least one of $B$, $D$ or $G$ is nonzero, in which
	case $x$ may be written as a function of $y$ and $z$, or
  \item
	$C = 0$ and any one of $B$, $E$ or $H$ is nonzero, in which case
	$y$ may be written as a function of $x$ and $z$.
  \ei
\item
  Even when the equation $p(x,y,z) = k$ cannot be re-written as
  a function of two variables, a good way to get an idea of the
  graph of the level surface is to consider cross-sections:
  \bi
  \item
	slices by planes parallel to the $xy$-plane are the result
	of setting $z = z_0$.
  \item
	slices by planes parallel to the $xz$-plane are the result
	of setting $y = y_0$.
  \item
	slices by planes parallel to the $yz$-plane are the result
	of setting $x = x_0$.
  \ei
  \egs{}\\[10pt]
  $\ds{9x^2 + y^2 - 4z^2 = 1} \quad$ (an hyperboloid in one sheet)\\[70pt]
  $\ds{9x^2 + y^2 - 4z^2 = 0} \quad$ (an elliptic cone)\\[70pt]
  $\ds{9x^2 - y^2 - 4z^2 = 1} \quad$ (an hyperboloid in two sheets)\\[70pt]
  $\ds{9x^2 + y^2 + 4z^2 = 1} \quad$ (an ellipsoid)
\ei

\end{document}

