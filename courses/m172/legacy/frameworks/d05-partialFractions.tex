\documentclass[12pt,fleqn]{article}

\newif\ifcomplete
\completetrue
%\completefalse
\def\topic{Integrals using Partial Fraction Expansion}
\def\lecdate{Mon., Feb.~5}

\oddsidemargin=0in
\textwidth=6.5in
\topmargin=-0.5in
\textheight=9.0in

\input mathMacros.tex
\usepackage{amsmath}
\usepackage{fancyhdr}
\usepackage{/Users/scofield/Library/texmf/latex/timestamp}
\usepackage[pdftex]{hyperref}
\hypersetup{pdfpagemode=None,colorlinks=true}
\usepackage{graphicx}
\usepackage[usenames,dvipsnames]{color}
\def\unsc{\blank{0.07in}}
%\def\RE{$\mathrm{Re}\,$}
%\def\IM{$\mathrm{Im}\,$}
\def\RE{$\mbox{Re}\,$}
\def\IM{$\mbox{Im}\,$}


\begin{document}

\pagestyle{fancy}
\fancyhf{}
\lhead{MATH 162---Framework for \lecdate}
\chead{}
\rhead{\topic}
\lfoot{}
\cfoot{\thepage}
\rfoot{}
\renewcommand{\headrulewidth}{0.4pt}
\renewcommand{\footrulewidth}{0.4pt}

\thispagestyle{empty}

\begin{center}
  \Large{MATH 162: Calculus II} \\
  \large{Framework for \lecdate} \\
  \large{\topic}
\end{center}

\vs{0.2in}
\ni
{\bf Definition}: A {\em rational function} is a function that is
the ratio of polynomials. \\[10pt]
Examples:
$ \ds{\qquad \frac{2x}{x^2 + 6} \qquad\qquad \frac{x^2 - 1}{(x^2+3x+1)(x-2)^2}
  \qquad\qquad \frac{1}{\sqrt{x + 7}}} $ \\[10pt]
{\bf Definition}: A quadratic (2nd-degree) polynomial function with
real coefficients is said to be {\em irreducible} (over the reals) if
it has no real roots. \\[10pt]
A quadratic polynomial is reducible if and only if it may be written as the
product of linear (1st-degree polynomial) factors with real coefficients \\[6pt]
Examples:
\bi
\item[] $\ds{x^2 + 4x + 3}$
\item[] $\ds{x^2 + 4x + 5}$
\ei

\section*{Partial fraction expansion}
\bi
\item
  Reverses process of
  ``combining rational fns.~into one''
  \bi
  \item
	Input: a rational fn.  \ Output: simpler rational
	fns.~that sum to input fn.
  \item
	degree of numerator in input fn.~must be less than or equal to degree
	of denominator (You may have to use long division to make this so.)
  \item
	denominator of input fn.~must be factored completely
	(i.e., into linear and quadratic polynomials)
  \ei
\item
  Why a ``technique of integration''?
\item
  leaves you with integrals that you must be able to evaluate by
  other means.  Some examples:
  $$ \int\frac{5}{2(x - 5)}\,dx \qquad\qquad
	\int\frac{2}{(3x + 1)^3}\,dx \qquad\qquad
	\int\frac{x+1}{(x^2+4)^2}\,dx $$
\ei

\end{document}

