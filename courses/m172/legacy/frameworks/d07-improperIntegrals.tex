\documentclass[12pt,fleqn]{article}

\newif\ifcomplete
\completetrue
%\completefalse
\def\topic{Improper Integrals}
\def\lecdate{Fri., Feb.~9}

\oddsidemargin=0in
\textwidth=6.5in
\topmargin=-0.5in
\textheight=9.0in

\input mathMacros.tex
\usepackage{amsmath}
\usepackage{fancyhdr}
\usepackage{/Users/scofield/Library/texmf/latex/timestamp}
\usepackage[pdftex]{hyperref}
\hypersetup{pdfpagemode=None,colorlinks=true}
\usepackage{graphicx}
\usepackage[usenames,dvipsnames]{color}
\def\unsc{\blank{0.07in}}
%\def\RE{$\mathrm{Re}\,$}
%\def\IM{$\mathrm{Im}\,$}
\def\RE{$\mbox{Re}\,$}
\def\IM{$\mbox{Im}\,$}


\begin{document}

\pagestyle{fancy}
\fancyhf{}
\lhead{MATH 162---Framework for \lecdate}
\chead{}
\rhead{\topic}
\lfoot{}
\cfoot{\thepage}
\rfoot{}
\renewcommand{\headrulewidth}{0.4pt}
\renewcommand{\footrulewidth}{0.4pt}

\thispagestyle{empty}

\begin{center}
  \Large{MATH 162: Calculus II} \\
  \large{Framework for \lecdate} \\
  \large{\topic}
\end{center}

\vs{0.2in}
\ni
Thus far in MATH 161/162, definite integrals $\ds{\int_a^b f(x)\,dx}$ have:
\bi
\item
  been over regions of integration which were finite in length
  (i.e., $a \ne -\infty$ and $b\ne\infty$)
\item
  involved integrands $f$ which are finite througout the region of integration
\ei

\vs{0.1in}
\ni
Q: How would we make sense of definite ({\em improper}, as they are
called) integrals that violate one or both of these assumptions?

\vspace{0.1in}
\ni
A: As limits (or sums of limits), when they exist, of definite integrals.
\bi
\item
  When all of the limits involved exist, the integral is said to {\em converge}.
\item
  When even one of the limits involved does not exist, the integral
  is said to {\em diverge}.
\ei

\vs{0.1in}
\ni
Examples: \\

\begin{minipage}[t]{1in}
  $\ds{\int_3^\infty \frac{dx}{x^3}}$

  \vspace{1in}
  $\ds{\int_0^1 \frac{dx}{\sqrt x}}$

  \vspace{1in}
  $\ds{\int_0^2 \frac{dx}{(x-1)^{2/3}}}$

  \vspace{1.1in}
  $\ds{\int_1^\infty \frac{dx}{x \sqrt{x^2-1}}}$
\end{minipage} \hspace{0.3in}
\begin{minipage}[t]{4in}
  \mbox{}

  \vspace{-0.3in}
  \includegraphics[width=2in]{figs/improperIntPict1.png}

  \vspace{0.1in}
  \includegraphics[width=2in]{figs/improperIntPict2.png}

  \vspace{0.1in}
  \includegraphics[width=2in]{figs/improperIntPict3.png}
\end{minipage}

\newpage
Evaluating them:
\bi
\item[]
  $\ds{\int_{-\infty}^0 e^x \, dx}$
\vspace{0.5in}
\item[]
  $\ds{\int_0^1 \ln x \, dx}$
\vspace{0.5in}
\item[]
  $\ds{\int_1^\infty \frac{dx}{x^p}}$
\vspace{2.0in}
\item[]
  $\ds{\int_{-\infty}^\infty \frac{dx}{1+x^2}}$
\ei

\vspace{1in}
\ni
Even when an improper integral cannot be evaluated exactly, one might
be able to determine if it converges or not.  One of several possible
theorems which address this issue:

\vspace{0.2in}
\ni
{\bf Theorem (Direct Comparison Test)}: \; Suppose $f$, $g$ satisfy
$0 \le f(x) \le g(x)$ for all $x \ge a$.  Then
\be
\item[(i)]
  $\ds{\int_a^\infty f(x)\, dx}$ \; converges if \;
  $\ds{\int_a^\infty g(x)\,dx}$ \; converges.
\item[(ii)]
  $\ds{\int_a^\infty g(x)\, dx}$ \; diverges if \;
  $\ds{\int_a^\infty f(x)\,dx}$ \; diverges.
\ee


\end{document}

