\documentclass[12pt,fleqn]{article}

\newif\ifcomplete
\completetrue
%\completefalse
\def\topic{Convergence of Taylor Series}
\def\lecdate{Mon., Feb.~26}

\oddsidemargin=0in
\textwidth=6.5in
\topmargin=-0.5in
\textheight=9.0in

\input mathMacros.tex
\usepackage{amsmath}
\usepackage{fancyhdr}
\usepackage{/Users/scofield/Library/texmf/latex/timestamp}
\usepackage[pdftex]{hyperref}
\hypersetup{pdfpagemode=None,colorlinks=true}
\usepackage{graphicx}
\usepackage[usenames,dvipsnames]{color}
\def\unsc{\blank{0.07in}}
%\def\RE{$\mathrm{Re}\,$}
%\def\IM{$\mathrm{Im}\,$}
\def\RE{$\mbox{Re}\,$}
\def\IM{$\mbox{Im}\,$}


\begin{document}

\pagestyle{fancy}
\fancyhf{}
\lhead{MATH 162---Framework for \lecdate}
\chead{}
\rhead{\topic}
\lfoot{}
\cfoot{\thepage}
\rfoot{}
\renewcommand{\headrulewidth}{0.4pt}
\renewcommand{\footrulewidth}{0.4pt}

\thispagestyle{empty}

\begin{center}
  \Large{MATH 162: Calculus II} \\
  \large{Framework for \lecdate} \\
  \large{\topic}
\end{center}

\vs{0.2in}
\ni
{\bf Today's Goal}: To determine if a function equals its power series.

\vs{0.2in}
\ni
In a remark from the last class, it was stated that, while
a certain function $f$ may allow the construction of a Taylor
series about $x=a$ with positive radius of convergence, one may
not assume this Taylor series converges to $f$.  In our
``favorite Taylor series'' (see the framework for that day),
however, the convergence of the MacLaurin series for $(1-x)^{-1}$,
$\arctan x$ and $\ln(1+x)$ to their respective functions throughout
their intervals of convergence has already been established.
What has yet to be established is whether the MacLaurin series
for $e^x$, $\cos x$ and $\sin x$ converge to their respective
functions.

\section*{The Remainder}

Suppose $f$ has $(n+1)$ derivatives throughout an interval
$I$ around $x=a$.  Under these conditions, we can write down
the $n^{\mbox{th}}$-order Taylor polynomial for $f$ about $x=a$:
$$ P_{n,a}(x) \;=\; f(a) + f'(a)(x-a) + \frac{f''(a)}{2!}(x-a)^2
	+ \frac{f'''(a)}{3!}(x-a)^3 + \cdots + \frac{f^{(n)}(a)}{n!}
	(x-a)^n, $$
Here the subscript $a$ has been added to indicate that this
polynomial is about $x=a$.  The discrepancy between the function
and its Taylor polynomial is called the {\em remainder} term:
$$ R_{n,a}(x) \;:=\; f(x) - P_{n,a}(x). $$

\ni
\framebox{\parbox[t]{6.5in}{
{\bf Theorem} (Lagrange): Suppose $f$, $P_{n,a}$ and $R_{n,a}$
are as described above, and that $x$ (fixed) is a number in the
interval $I$.  Then there is a number $t$ between $a$ and $x$
such that
$$ R_{n,a}(x) \;=\; \frac{f^{(n+1)}(t)}{(n+1)!}\,(x-a)^{n+1}. $$
}}

\vs{0.2in}
\ni
{\bf Example}: We can use Lagrange's theorem to show that $\sin x$
is equal to its MacLaurin series for every real number $x$.
For any (fixed) $x$, the theorem guarantees the existence of
a number $t$ between 0 and $x$ such that
$$ \left|R_{n,0}(x)\right|
	\;=\; \frac{|\sin^{(n+1)}(t)|}{(n+1)!}\,|x|^{n+1}
	\;\le\; \frac{|x|^{n+1}}{(n+1)!}
	\;\rightarrow\; 0 \quad\mbox{as}\quad n\rightarrow\infty. $$
Thus,
$$ \lim_{n\rightarrow\infty} P_{n,0}(x)
	\;=\; \lim_{n\rightarrow\infty} \left[\sin x - R_{n,0}(x)\right]
	\;=\; \sin x - \lim_{n\rightarrow\infty} R_{n,0}(x)
	\;=\; \sin x, $$
which says that the sequence of partial sums of the MacLaurin
series for the sine function converges to sine at $x$.
Since we did not assume anything special about the $x$ involved
in this calculation, the result holds for any real $x$.

\np
\ni
A similar type of argument may be used to establish that the
MacLaurin series for $e^x$ converges to $e^x$ for all real
$x$, and that the MacLaurin series for $\cos x$ converges to
$\cos x$ for all real $x$.

\vspace{0.3in}
\ni
{\bf Example} (a weird function):  Let $f$ be defined by the formula
$$ f(x) \;:=\; \left\{\begin{array}{ll}
	e^{-1/x^2}, & x \ne 0, \\[6pt]
	0, & x = 0.
  \end{array}\right. $$
It can be shown that $f^{(n)}(0) = 0$ for all integer $n \ge 0$.
The MacLaurin series for $f$ is thus
$$ \sum_{n=0}^\infty \frac{0}{n!} x^n \;=\; 0, $$
the zero function (not even an infinite series, so of course
it converges for all $x$).  A graph of $f$ appears in
Figure 8.14 on p.~558 of the text.  It may not be obvious from
the picture, but while $f(0) = 0$, for all other choices of $x$,
$f(x) > 0$.  Hence, the only place its MacLaurin series equals
$f$ is at $x=0$.

\end{document}

