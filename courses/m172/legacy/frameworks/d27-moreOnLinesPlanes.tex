\documentclass[12pt,fleqn]{article}

\newif\ifcomplete
\completetrue
%\completefalse
\def\topic{More about Lines and Planes in Space}
\def\lecdate{Tues., Mar.~27}

\oddsidemargin=0in
\textwidth=6.5in
\topmargin=-0.5in
\textheight=9.0in

\input mathMacros.tex
\usepackage{amsmath}
\usepackage{ascii}
%\usepackage{wasysym}
\usepackage{fancyhdr}
\usepackage{/Users/scofield/Library/texmf/latex/timestamp}
\usepackage[pdftex]{hyperref}
\hypersetup{pdfpagemode=None,colorlinks=true}
\usepackage{graphicx}
\usepackage[usenames,dvipsnames]{color}
\def\mbunsc{\blank{0.07in}}
\def\pref#1{(\ref{#1})}
%\def\RE{$\mathrm{Re}\,$}
%\def\IM{$\mathrm{Im}\,$}
\def\RE{$\mbox{Re}\,$}
\def\IM{$\mbox{Im}\,$}
\def\fb#1{\framebox{\parbox[t]{6.5in}{#1}}}
\def\sfb#1{\framebox{\parbox[t]{6.0in}{#1}}}
\def\defn#1{\fb{{\bf Definition}: #1}}
\def\sdefn#1{\sfb{{\bf Definition}: #1}}
\def\thm#1{\fb{{\bf Theorem}: #1}}
\def\sthm#1{\sfb{{\bf Theorem}: #1}}
\def\eg#1{{\bf Example}: #1}
\def\egs#1{{\bf Examples}: #1}
\def\newt{\vspace{0.2in}\ni}
\setlength{\fboxsep}{5pt}
%\def\vdotprod{{\ascii\BEL}}
\def\vdotprod{\,\mbox{{\Large $\cdot$}}\,}
\def\vcrossprod{\times}


\begin{document}

\pagestyle{fancy}
\fancyhf{}
\lhead{MATH 162---Framework for \lecdate}
\chead{}
\rhead{\topic}
\lfoot{}
\cfoot{\thepage}
\rfoot{}
\renewcommand{\headrulewidth}{0.4pt}
\renewcommand{\footrulewidth}{0.4pt}

\thispagestyle{empty}

\begin{center}
  \Large{MATH 162: Calculus II} \\
  \large{Framework for \lecdate} \\
  \large{\topic}
\end{center}

\vs{0.2in}
\ni
{\bf Today's Goal}:
To review how lines and planes in space are represented,
and use these notions to derive some useful formulas
and algorithms involving points, lines and planes.

\vspace{0.2in}
\subsection*{Lines and Planes}
We have derived the following representations.
\bi
\item
  {\bf Lines}.  The line trough point $P = (x_0,y_0,z_0)$
  parallel to $\mbv = v_1 \mbi + v_2 \mbj + v_3 \mbk$ \\[8pt]
  \begin{tabular}{ll}
	{\bf component form}: & $x = x_0 + v_1 t,$ \\
	& $y = y_0 + v_2 t,$ $\qquad -\infty < t < \infty$, \\
	& $z = z_0 + v_3 t,$ \\[8pt]
	{\bf vector form}: & $\mbr(t) = (x_0 + v_1t)\mbi + (y_0 + v_2t)\mbj
	  (z_0 + v_3t)\mbk,$ $\qquad -\infty < t < \infty$. \\
  \end{tabular}

\item
  {\bf Planes}.  The plane trough point $P = (x_0,y_0,z_0)$
  perpendicular to $\mbn = a \mbi + b \mbj + c\mbk$
  $$ \mbn \vdotprod [(x - x_0)\mbi + (y - y_0)\mbj + (z - z_0)\mbk] = 0, 
	\qquad\mbox{or}\qquad ax + by + cz = d, $$
  where $d = ax_0 + by_0 + cz_0$.
\ei

\vspace{0.2in}
\subsection*{Formulas and Algorithms for Lines and Planes}
\bi
\item
  {\bf Distance from a point $S$ to a line $L$}.\\[8pt]
  \begin{minipage}[t]{4in}
	Keys to a formula:
	\be
	\item
	  Our distance is $|\overrightarrow{PS}|\sin\theta$, where\\
	  $P$ is any point on line $L$.
	\item
	  For two vectors $\mbu$ and $\mbv$,
	  $\quad |\mbu\vcrossprod\mbv| = |\mbu||\mbv|\sin\theta$.
%	  \[ |\mbu\vcrossprod\mbv| = |\mbu||\mbv|\sin\theta. \]
	\ee
	From these we get
	\[ |\overrightarrow{PS}|\sin\theta = \frac{|\overrightarrow{PS}
	  \vcrossprod \mbv|}{|\mbv|}, \]
	where $\mbv$ is any vector parallel to line $L$.
  \end{minipage}
% \hspace{0.5in}
  \begin{minipage}[t]{2in}
	\mbox{}

	\setlength{\unitlength}{2mm}
	\begin{picture}(35,10)
	  \put(0,7){\line(5,-1){35}}
	  \put(25,2){\line(1,5){2.32}}
	  \put(5,6){\circle*{1}}
	  \put(27.3,13.4){\circle*{1}}
	  \put(28,-1.5){$L$}
	  \put(4.1,3.5){$P$}
	  \put(27,15.2){$S$}
	  \qbezier(8.5,6.7)(9.3,6.5)(9,4.9)
	  \put(10.5,5.6){$\theta$}
%	  \put(17,1){$\mbv$}
	  \thicklines
	  \put(5,6){\vector(3,1){22.2}}
%	  \put(15,3){\vector(5,-1){8}}    % vector directions are too limited
	\end{picture}
  \end{minipage}

\np
\item
  {\bf Distance from a point $S$ to a plane} containing the point $P$
  with normal vector $\mbn$. \\[8pt]
  Keys to a formula:
  \be
  \item
	Our distance is $|\overrightarrow{PS}||\cos\theta|$, where
	$\theta$ is the angle between $\overrightarrow{PS}$ and $\mbn$.
  \item
	If $\theta$ is the angle between vectors $\mbu$ and $\mbv$, then
	$\quad\ds{ \cos\theta \;=\; \frac{\mbu\vdotprod\mbv}{|\mbu||\mbv|} }$.
  \ee
  Thus, we get
  $$ |\overrightarrow{PS}||\cos\theta| \;=\; \frac{|\overrightarrow{PS}
	\vdotprod \mbn|}{|\mbn|}. $$

\item
  {\bf Angle between two planes}. \\[8pt]
  \sdefn{The {\em angle between planes} is taken to
	be the angle $\theta \in [0,\pi/2]$ between normal
	vectors to the planes.}

  By this definition, if $\mbn_1$ and $\mbn_2$ are normal vectors
  to the two planes, then the angle between the planes is
  $$ \theta \;=\; \left\{\begin{array}{ll}
	\ds{\arccos\left(\frac{\mbn_1\vdotprod\mbn_2}
	  {|\mbn_1||\mbn_2|}\right),}
	  & \mbox{if}\; \mbn_1\vdotprod\mbn_2 \ge 0, \\[14pt]
	\ds{\pi - \arccos\left(\frac{\mbn_1\vdotprod\mbn_2}
	  {|\mbn_1||\mbn_2|}\right),}
	  & \mbox{if}\; \mbn_1\vdotprod\mbn_2 < 0.
	\end{array}\right. $$

\item
  {\bf Line of intersection between two non-parallel planes}. \\[10pt]
  It should not be difficult to find a point on the desired line.
  If the two planes have equations $a_1 x + b_1 y + c_1 z = d_1$
  and $a_2 x + b_2 y + c_2 z = d_2$, then it is quite likely the
  line of intersection will eventually pass through a point $P$ where
  the $x$-coordinate is zero.  Assuming this is so, we may do the
  usually steps of solving the simultaneous equations in 2 unknowns
  $$ \begin{array}{l} b_1 y + c_1 z = d_1 \\ b_2 y + c_2 z = d_2 \end{array} $$
  for the corresponding $y$ and $z$ coordinates of this point.
  (If the solution process fails to yield corresponding $y$ and $z$
  coordinates, one can instead look for the point $P$ for which the
  $y$ or, alternatively, the $z$-coordinate is zero.)

  Once a point $P$ on our line of intersection is found, we next need
  a vector that is parallel to our line.  Such a vector would be
  perpendicular to normal vectors to both planes, and so could be
  any multiple of
  $$ (a_1 \mbi + b_1 \mbj + c_1 \mbk) \vcrossprod
	(a_2 \mbi + b_2 \mbj + c_2 \mbk)  \;=\; \left|\begin{array}{ccc}
	\mbi & \mbj & \mbk \\ a_1 & b_1 & c_1 \\ a_2 & b_2 & c_2
  \end{array}\right|. $$
\ei

\end{document}

