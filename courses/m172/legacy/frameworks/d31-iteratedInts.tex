\documentclass[12pt,fleqn]{article}

\newif\ifcomplete
\completetrue
%\completefalse
\def\topic{Double and Iterated Integrals, Rectangular Regions}
\def\lecdate{Wed., Apr.~11}

\oddsidemargin=0in
\textwidth=6.5in
\topmargin=-0.5in
\textheight=9.0in

\input mathMacros.tex
\usepackage{amsmath}
\usepackage{ascii}
%\usepackage{wasysym}
\usepackage{fancyhdr}
\usepackage{/Users/scofield/Library/texmf/latex/timestamp}
\usepackage[pdftex]{hyperref}
\hypersetup{pdfpagemode=None,colorlinks=true}
\usepackage{graphicx}
\usepackage[usenames,dvipsnames]{color}
\def\mbunsc{\blank{0.07in}}
\def\pref#1{(\ref{#1})}
%\def\RE{$\mathrm{Re}\,$}
%\def\IM{$\mathrm{Im}\,$}
\def\RE{$\mbox{Re}\,$}
\def\IM{$\mbox{Im}\,$}
\def\fb#1{\framebox{\parbox[t]{6.5in}{#1}}}
\def\sfb#1{\framebox{\parbox[t]{6.0in}{#1}}}
\def\defn#1{\fb{{\bf Definition}: #1}}
\def\sdefn#1{\sfb{{\bf Definition}: #1}}
\def\thm#1{\fb{{\bf Theorem}: #1}}
\def\sthm#1{\sfb{{\bf Theorem}: #1}}
\def\eg#1{{\bf Example}: #1}
\def\egs#1{{\bf Examples}: #1}
\def\newt{\vspace{0.2in}\ni}
\setlength{\fboxsep}{5pt}
%\def\vdotprod{{\ascii\BEL}}
\def\vdotprod{\,\mbox{{\Large $\cdot$}}\,}
\def\vcrossprod{\times}


\begin{document}

\pagestyle{fancy}
\fancyhf{}
\lhead{Framework for \lecdate}
\chead{}
\rhead{\topic}
\lfoot{}
\cfoot{\thepage}
\rfoot{}
\renewcommand{\headrulewidth}{0.4pt}
\renewcommand{\footrulewidth}{0.4pt}

\thispagestyle{empty}

\begin{center}
  \Large{MATH 162: Calculus II} \\
  \large{Framework for \lecdate} \\
  \large{\topic}
\end{center}

\vs{0.2in}
\ni
{\bf Today's Goal}:
To understand the meaning of double integrals over
bounded rectangular regions $R$ of the plane, and to be
able to evaluate such integrals.

\vspace{0.15in}
\ni
{\bf Important Note}: In conjunction with this framework,
you should look over Section 13.1 of your text.

\vspace{0.15in}
\subsection*{Riemann Sums}
We assume that $f(x,y)$ is a function of 2 variables, and
that $R = \{(x,y)\in\mathR^2 \,|\, a\le x\le b, \; c\le y\le d\}$
(i.e., $R$ is some bounded rectangular region of the plane whose sides
are parallel to the coordinate axes).  Supppose we
\bi
\item
  divide $R$ up into $n$ smaller rectangles, labeling them
  $R_1$, $R_2$, \ldots, $R_n$,
\item
  choose, from each rectangle $R_k$, some point $(x_k,y_k)$, and
\item
  use the symbol $\Delta A_k$ to denote the area of rectangle $R_k$.
\ei
Then the sum
$$ \sum_{k=1}^n f(x_k,y_k)\Delta A_k $$
is called a {\em Riemann sum} of $f$ over the region $R$.\\[10pt]
\defn{
The collection of smaller rectangles $R_1$, \ldots, $R_n$ is
called a {\em partition} $P$ of $R$.  The maximum, taken over all
lengths and widths of these rectangles, is called the {\em norm}
of the partition $P$, and is denoted by $\|P\|$.\\[10pt]
The function $f$ is said to be {\em integrable over} $R$ if the limit
$$ \lim_{\|P\|\rightarrow 0} \sum_{k=1}^n f(x_k,y_k)\Delta A_k, $$
taken over all partitions $P$ of $R$, exists.  The value of this
limit, called the {\em double integral of} $f$ over $R$, is denoted by
$$ \iint\limits_R f(x,y)\,dA. $$}\\[10pt]
The following theorem tells us that, for certain functions,
integrability is assured\\[10pt]
\thm{
  Suppose $f(x,y)$ is a continuous function over the closed and
  bounded rectangle $R = \{(x,y\in\mathR^2 \,|\, a\le x\le b, \;
  c\le y\le d\}$ in the plane.  Then $f$ is integrable over $R$.}

\newpage
\ni
Remarks
\bi
\item
  The double integral of $f$ over $R$ is a definite integral,
  and has a numeric value.
\item
  When $f(x,y)$ is a nonnegative function, $\iint_R f(x,y)\,dA$
  may be interpreted as the volume under the surface $z = f(x,y)$
  over the region $R$ in the $xy$-plane.
\item
  When $f(x,y)$ is a constant function (i.e., $f$ has the same
  value, say $c$, for each input point $(x,y)$), then
  $$ \iint\limits_R f(x,y)\,dA \;=\; c \cdot Area(R). $$
\ei

\vspace{0.1in}
\subsection*{Iterated Integrals and Fubini's Theorem}
Except in the very special case of constant functions $f$, the
definition of $\iint_R f(x,y)\,dA$ does not, by itself, provide
us with much help for evaluating a double integral.  (If you are
viewing this document on the web, click
\href{http://www.jstor.org/view/0025570x/di021064/02p0066q/0}{here}
for an alternate point of view, expressed by Peter A.~Lindstrom
of Genesee Community College.)  However, the following theorem
indicates that the double integral may be evaluated as an
{\em iterated integral}.\\[16pt]
\thm{
({\bf{Fubini}}) Suppose $f(x,y)$ is continuous throughout the
rectangle \mbox{$R = \{(x,y)\in\mathR^2 \,|\,
a\le x\le b, \; c\le y\le d\}$}.  Then
$$ \iint\limits_R f(x,y)\,dA \;=\; \int_a^b \int_c^d f(x,y)\,dy\,dx
	\;=\; \int_c^d \int_a^b f(x,y)\,dx\,dy. $$ }\\[16pt]
\egs{}
Evaluate the double integral, with $R$ as specified.
\be
\item
  $\ds{\iint\limits_R (2x + x^2 y)\,dA,}\quad$ where
  $R: -2 \le x \le 2, \; -1 \le y \le 1$.\\[52pt]
\item
  $\ds{\iint\limits_R y\sin(xy)\,dA,}\quad$ where
  $R: 1 \le x \le 2, \; 0 \le y \le \pi$.
\ee

\end{document}

