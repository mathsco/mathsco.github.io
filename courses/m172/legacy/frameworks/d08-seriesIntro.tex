\documentclass[12pt,fleqn]{article}

\newif\ifcomplete
\completetrue
%\completefalse
\def\topic{Introduction to Series}
\def\lecdate{Tues., Feb.~13}

\oddsidemargin=0in
\textwidth=6.5in
\topmargin=-0.5in
\textheight=9.0in

\input mathMacros.tex
\usepackage{amsmath}
\usepackage{fancyhdr}
\usepackage{/Users/scofield/Library/texmf/latex/timestamp}
\usepackage[pdftex]{hyperref}
\hypersetup{pdfpagemode=None,colorlinks=true}
\usepackage{graphicx}
\usepackage[usenames,dvipsnames]{color}
\def\unsc{\blank{0.07in}}
%\def\RE{$\mathrm{Re}\,$}
%\def\IM{$\mathrm{Im}\,$}
\def\RE{$\mbox{Re}\,$}
\def\IM{$\mbox{Im}\,$}


\begin{document}

\pagestyle{fancy}
\fancyhf{}
\lhead{MATH 162---Framework for \lecdate}
\chead{}
\rhead{\topic}
\lfoot{}
\cfoot{\thepage}
\rfoot{}
\renewcommand{\headrulewidth}{0.4pt}
\renewcommand{\footrulewidth}{0.4pt}

\thispagestyle{empty}

\begin{center}
  \Large{MATH 162: Calculus II} \\
  \large{Framework for \lecdate} \\
  \large{\topic}
\end{center}

\vs{0.2in}
\begin{minipage}[t]{3.8in}
  \ni
  {\bf Example}: Application of the direct comparison test

  \vspace{0.15in}
  Suppose $g(x) = x^{-p}$, with $p > 0$ (so $g$ has the general shape
  of the blue curve for $x > 0$), and $f$ is the step function pictured
  in black.
  \bi
  \item
	By the direct comparison test, if $p > 1$ then
	\beqn
	  \int_1^\infty f(x)\,dx & = & 2^{-p} + 3^{-p} + 4^{-p} + \cdots
		+ n^{-p} + \cdots \\
	  & = & \sum_{n=2}^\infty n^{-p}
	\eeqn
	converges.  And, since it is the case that
	$$ \sum_{n=1}^\infty n^{-p} \;=\; 1 + \sum_{n=2}^\infty n^{-p}, $$
	$\sum_{n=1}^\infty n^{-p}$ converges as well when $p > 1$.

  \item
	To conclude $\sum_{n=1}^\infty n^{-p}$ diverges for $p \le 1$,
	we must deal with a function like $f$ that stays above $g$.
	The function $h(x) = f(x-1)$ will do (see at right).
	The improper integral
	\beqn
	  \int_1^\infty h(x)\, dx & = & 1^{-p} + 2^{-p} + 3^{-p} + \cdots \\
	  & = & \sum_{n=1}^\infty n^{-p}
	\eeqn
	diverges since $\int_1^\infty x^{-p}\,dx$ diverges for $p \le 1$.
  \ei
\end{minipage}
\begin{minipage}[t]{3.5in}
  \mbox{}

  \vspace{-0.55in}
  \includegraphics[width=3.5in]{figs/pSeriesFns.png}

  \vspace{-0.2in}
  \includegraphics[width=3.5in]{figs/pSeriesRects.png}

  \vspace{0.5in}
  \includegraphics[width=3.5in]{figs/pSeriesRectsShifted.png}
\end{minipage}

\np
\ni
\section*{Infinite Series}
\bi
\item
  An infinite sum of numbers: \;\;
  $\sum_{n=1}^\infty a_n = a_1 + a_2 + \cdots + a_n + \cdots$.
\item
  Can be thought of as an improper integral:

  \ni
  \begin{minipage}[t]{3.0in}
  Define $\;\; \ds{f(x) := \left\{\begin{array}{ll}
	a_1, & 0 \le x < 1, \\
	a_2, & 1 \le x < 2, \\
	\,\,\vdots & \qquad\vdots \\
	a_n, & n-1 \le x < n, \\
	\,\,\vdots & \qquad\vdots
  \end{array}\right.}$\\[6pt]
  (See graph of step fn.~at right.)

  \vspace{0.3in}
  \ni
  Then our given series may be expressed as an improper
  integral of $f$:
  $$ \sum_{n=1}^\infty a_n \;=\; \int_0^\infty f(x)\,dx. $$
% $$ \int_0^\infty f(x)\,dx \;=\; \sum_{n=1}^\infty a_n. $$
  \end{minipage}
  \begin{minipage}[t]{3.5in}
	\mbox{}

	\vs{-0.9in}
	\includegraphics[width=3.5in]{figs/numsToFn.png} \\

	\vs{-0.2in}
	\includegraphics[width=3.5in]{figs/numsToRects.png}
  \end{minipage}
\item
  Will be said to {\em converge} or {\em diverge}.  As with
  other improper integrals, convergence requires the existence
  of a limit of ``proper sums'' (actually called {\em partial sums}).
  Define
  $$ \begin{array}{l}
	  s_1 \;:=\; a_1, \\
	  s_2 \;:=\; a_1 + a_2, \\
	  s_3 \;:=\; a_1 + a_2 + a_3, \\
	  \;\vdots \qquad\qquad\qquad \vdots \\
	  s_n \;:=\; a_1 + a_2 + \cdots + a_n, \\
	  \;\vdots \qquad\qquad\qquad \vdots
	\end{array} $$
  The series $\sum_{n=1}^\infty a_n$ converges precisely when
  the {\em sequence} $s_n$ converges to some real number limit.
\ei

\ni
{\bf Examples}:
\bi
\item[]
  $\ds{1 - 1 + 1 - 1 + 1 - 1 + \cdots \;=\; \sum_{n=1}^\infty (-1)^{n-1}}
  \qquad \mbox{diverges}.$
\item[]
  $\ds{\sum_{n=3}^\infty \frac{1}{n(n+1)} = \sum_{n=1}^\infty \left(
  \frac{1}{n} - \frac{1}{n+1}\right)} \qquad \mbox{converges}.$
\ei

\end{document}

