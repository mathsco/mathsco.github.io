\documentclass[12pt,fleqn]{article}

\newif\ifcomplete
\completetrue
%\completefalse
\def\topic{Differentiability}
\def\lecdate{Tues., Mar.~6}

\oddsidemargin=0in
\textwidth=6.5in
\topmargin=-0.5in
\textheight=9.0in

\input mathMacros.tex
\usepackage{amsmath}
\usepackage{fancyhdr}
\usepackage{/Users/scofield/Library/texmf/latex/timestamp}
\usepackage[pdftex]{hyperref}
\hypersetup{pdfpagemode=None,colorlinks=true}
\usepackage{graphicx}
\usepackage[usenames,dvipsnames]{color}
\def\unsc{\blank{0.07in}}
%\def\RE{$\mathrm{Re}\,$}
%\def\IM{$\mathrm{Im}\,$}
\def\RE{$\mbox{Re}\,$}
\def\IM{$\mbox{Im}\,$}
\def\fb#1{\framebox{\parbox[t]{6.5in}{#1}}}
\def\sfb#1{\framebox{\parbox[t]{6.0in}{#1}}}
\def\defn#1{\fb{{\bf Definition}: #1}}
\def\sdefn#1{\sfb{{\bf Definition}: #1}}
\def\thm#1{\fb{{\bf Theorem}: #1}}
\def\sthm#1{\sfb{{\bf Theorem}: #1}}
\def\eg#1{{\bf Example}: #1}
\def\egs#1{{\bf Examples}: #1}
\setlength{\fboxsep}{5pt}


\begin{document}

\pagestyle{fancy}
\fancyhf{}
\lhead{MATH 162---Framework for \lecdate}
\chead{}
\rhead{\topic}
\lfoot{}
\cfoot{\thepage}
\rfoot{}
\renewcommand{\headrulewidth}{0.4pt}
\renewcommand{\footrulewidth}{0.4pt}

\thispagestyle{empty}

\begin{center}
  \Large{MATH 162: Calculus II} \\
  \large{Framework for \lecdate} \\
  \large{\topic}
\end{center}

\vs{0.2in}
\ni
{\bf Today's Goal}: To understand the relationship between partial
derivatives and continuity.

\section*{The Mixed Partial Derivatives}

We have learned that the partial derivative $f_x$ at $(x_0,y_0)$
may be interpreted geometrically as providing the slope at the
point $(x_0,y_0,f(x_0,y_0))$ along the curve that results from
slicing the surface $z = f(x,y)$ with the plane $y=y_0$.  If one
thinks of the $x$-axis as ``facing east'', then what we are talking
about is akin to standing on a patch of (possibly) hilly ground and
asking what slope you would immediately experience heading eastward
from your current position.  Now imagine moving northward (i.e., in the
direction of the positive $y$-axis), but still determining eastward
slopes.  The rate at which those eastward slopes changed as you
moved northward is precisely what $f_{xy} = \partial/\partial y
(f_x)$ provides.

One might ask the following question: Suppose I mark a particular
spot on this hypothetical terrain.  Then I cross over the mark
twice.  The first time, I do so heading northward, noting the
rate of change of eastward-facing slopes as I cross (that is,
$f_{xy}$).  The second time, I do so heading eastward, noting
the rate at which northward-facing slopes change as I cross (i.e.,
$f_{yx}$).  Should these two rates of change be equal?  There
does not seem to be a particular reason why they should be, but
experimenting with various formulas $f(x,y)$ we find, nevertheless,
that they often are.

\vspace{0.2in}
\ni
\eg{} $f(x,y) = \cos(x^2y)$

\vspace{0.2in}
\ni
This phenomenon has much to do with our natural inclination to
choose ``nice'' functions.  In general, $f_{xy}$ and $f_{yx}$
are not equal.  But, under the conditions of the following theorem,
they are.

\vspace{0.2in}
\ni
\thm{({\bf The Mixed Derivative Theorem}, p.~26)
If $f(x,y)$ and its partial derivatives $f_x$, $f_y$, $f_{xy}$ and
$f_{yx}$ are defined throughout an open region of the plane
containing the point $(x_0,y_0)$, and are all continuous at
$(x_0,y_0)$, then
$$ f_{xy}(x_0,y_0) = f_{yx}(x_0,y_0). $$}

\section*{Differentiability and Continuity}

In MATH 161, we learn
\bi
\item
  how to differentiate a function of a single variable
\item
  at points of differentiability, the function
  \bi
  \item is also continuous.
  \item looks (locally) like a straight line.
  \ei
\ei

\ni
For functions of multiple variables, we have learned how to take
partial derivatives, and what these partial derivatives represent.
Unfortunately, existence of partial derivatives does not, by itself,
imply continuity.

\vspace{0.2in}
\ni
\eg{}
For the function
$$ f(x,y) \;:=\; \left\{\begin{array}{ll}
	1, & \mbox{if}\; xy = 0, \\
	0, & \mbox{if}\; xy \ne 0,
  \end{array}\right. $$
the partial derivatives exist at $(0,0)$.  However, $f$
is not continuous at $(0,0)$.  (The graph of this function
is given on p.~725 of your text.)

\vspace{0.2in}
\ni
We would like functions of multiple variables, like their
single-variable counterparts, to be continuous whenever they are
differentiable.  In light of the previous example, we will require
more of such a function than just ``its partial derivatives exist''
before we call it {\em differentiable}.

\vspace{0.2in}
\ni
\defn{
A function $z = f(x,y)$ is said to be {\em differentiable at}
$(x_0,y_0)$, a point in the domain of $f$, if $f_x(x_0,y_0)$
and $f_y(x_0,y_0)$ both exist, and $\Delta z := f(x,y) - f(x_0,y_0)$
satisfies the equation
\begin{equation}
  \Delta z \;=\; f_x(x_0,y_0) \Delta x + f_y(x_0,y_0) \Delta y
	+ \epsilon_1 \Delta x + \epsilon_2 \Delta y, \label{differentiableEqn}
\end{equation}
where $$ \Delta x \;:=\; x - x_0, \qquad \Delta y \;:=\; y - y_0, $$ and
$\epsilon_1$, $\epsilon_2 \rightarrow 0$ as $(x,y)\rightarrow (x_0,y_0)$.
}

\vspace{0.2in}
\ni
If we drop the terms in equation (\ref{differentiableEqn}) that
become more and more negligible as $(x,y)\rightarrow (x_0,y_0)$
(the ones involving $\epsilon_1$ and $\epsilon_2$), then we obtain
the approximation
\begin{equation}
  \Delta z \;\approx\; f_x(x_0,y_0) \Delta x + f_y(x_0,y_0) \Delta y,
	\label{linearApprox}
\end{equation}
or
$$ f(x,y) - f(x_0,y_0) \;\approx\; f_x(x_0,y_0)(x-x_0) + f_y(x_0,y_0)(y-y_0), $$
or
$$ f(x,y) \;\approx\; f_x(x_0,y_0)(x-x_0) + f_y(x_0,y_0)(y-y_0) + f(x_0,y_0). $$
The right-hand side of this last version of the approximation is in the
form
$$ Ax + By + C. $$
Later in the course, we shall see that this is one form of the equation
of a plane.  Thus, the definition says that $z = f(x,y)$ is
differentiable at $(x_0,y_0)$ if, locally speaking, the surface at
the point looks like (is well-approximated by) the plane
$$ f(x_0,y_0) + f_x(x_0,y_0)(x-x_0) + f_y(x_0,y_0)(y-y_0). $$
We know (from the example above) that existence of partial derivatives
at the point $(x_0,y_0)$ alone is not sufficient to guarantee that a
function is differentiable there.  However, the following theorem
provides a stronger condition that guarantees it.

\vspace{0.2in}
\ni
\thm{Let $f$ be be a function of 2 variables whose partial derivatives
$f_x$ and $f_y$ are continuous throughout an open region $R$ of the
plane.  Then $f$ is differentiable at each point of $R$.}

\vspace{0.2in}
\ni
Given our notion of differentiability, we may prove this analog
to the theorem from MATH 161 relating differentiability and continuity.

\vspace{0.2in}
\ni
\thm{If a function $f$ of two variables is differentiable at
$(x_0,y_0)$, then $f$ is continuous there.}

\section*{Differential Notation and Linear Approximation}

For functions of one variable $y = f(x)$, we sometimes write
$dy = f'(x) dx$.  What does this mean?
\bi
\item
  $dx$ is an independent variable (think of it like $\Delta x$)
\item
  $dy$ is a dependent variable, a function of both $x$ and $dx$.
\item
  This ``differential notation'' is another way of writing the
  linear approximation to $f$.
\ei
Now, for the function $z = f(x,y)$, we may analogously write
$$ dz = f_x(x,y) dx + f_y(x,y) dy. $$
Compare this to equation (\ref{linearApprox}).

\vspace{0.2in}
\ni
\eg{}The volume of a right circular cylinder is given by
$v(r,h) = \pi r^2 h$.  Thus
$$ dv \;=\; 2\pi rh\, dr + \pi r^2\, dh. $$
Thus, if a cylinder of radius $2$ in.~and height $5$ in.~is
deformed to a different cylinder, now of radius $1.98$
in.~and height $5.03$ in., then the approximate change in
volume is
$$ 2\pi (2)(5)(-0.02) + \pi(2)^2(0.03) \;=\; -0.8797 $$
cubic inches.  (The actual change is -0.8809 cubic inches.)

\end{document}

