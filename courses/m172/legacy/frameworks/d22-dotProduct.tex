\documentclass[12pt,fleqn]{article}

\newif\ifcomplete
\completetrue
%\completefalse
\def\topic{Dot Products and Projections}
\def\lecdate{Mon., Mar.~12}

\oddsidemargin=0in
\textwidth=6.5in
\topmargin=-0.5in
\textheight=9.0in

\input mathMacros.tex
\usepackage{amsmath}
\usepackage{ascii}
\usepackage{fancyhdr}
\usepackage{/Users/scofield/Library/texmf/latex/timestamp}
\usepackage[pdftex]{hyperref}
\hypersetup{pdfpagemode=None,colorlinks=true}
\usepackage{graphicx}
\usepackage[usenames,dvipsnames]{color}
\def\mbunsc{\blank{0.07in}}
%\def\RE{$\mathrm{Re}\,$}
%\def\IM{$\mathrm{Im}\,$}
\def\RE{$\mbox{Re}\,$}
\def\IM{$\mbox{Im}\,$}
\def\fb#1{\framebox{\parbox[t]{6.5in}{#1}}}
\def\sfb#1{\framebox{\parbox[t]{6.0in}{#1}}}
\def\defn#1{\fb{{\bf Definition}: #1}}
\def\sdefn#1{\sfb{{\bf Definition}: #1}}
\def\thm#1{\fb{{\bf Theorem}: #1}}
\def\sthm#1{\sfb{{\bf Theorem}: #1}}
\def\eg#1{{\bf Example}: #1}
\def\egs#1{{\bf Examples}: #1}
\def\newt{\vspace{0.2in}\ni}
\setlength{\fboxsep}{5pt}
%\def\vdotprod{{\ascii\BEL}}
\def\vdotprod{\,\mbox{{\Large $\cdot$}}\,}


\begin{document}

\pagestyle{fancy}
\fancyhf{}
\lhead{MATH 162---Framework for \lecdate}
\chead{}
\rhead{\topic}
\lfoot{}
\cfoot{\thepage}
\rfoot{}
\renewcommand{\headrulewidth}{0.4pt}
\renewcommand{\footrulewidth}{0.4pt}

\thispagestyle{empty}

\begin{center}
  \Large{MATH 162: Calculus II} \\
  \large{Framework for \lecdate} \\
  \large{\topic}
\end{center}

\vs{0.2in}
\ni
{\bf Today's Goal}: To define the dot product and learn of some of its
properties and uses

\vspace{0.1in}
\subsection*{The Dot Product}

\defn{For vectors $\mbu = \left<u_1,u_2,u_3\right>$ and
$\mbv = \left<v_1,v_2,v_3\right>$, we define the {\em dot
product} of $\mbu$ and $\mbv$ to be
$$ \mbu \vdotprod \mbv \;:=\; u_1 v_1 + u_2 v_2 + u_3 v_3. $$}

\newt
Notes:
\bi
\item
  The dot product $\mbu\vdotprod\mbv$ is a scalar (number), not another vector.
\item
  Properties
  \be
  \item[1.] $\mbu\vdotprod\mbv \;=\; \mbv\vdotprod\mbu$
  \item[2.] $c(\mbu\vdotprod\mbv) \;=\; (c\mbu)\vdotprod\mbv
				\;=\; \mbu\vdotprod (c\mbv)$.
  \item[3.] $\mbu\vdotprod(\mbv + \mbw) \;=\; \mbu\vdotprod\mbv
				+ \mbu\vdotprod\mbw$
  \item[4.] $\0\vdotprod\mbv \;=\; 0$
  \item[5.] $\mbv\vdotprod\mbv \;=\; |\mbv|^2$
  \ee
\ei

\newt
\thm{If $\mbu$ and $\mbv$ are nonzero vectors, then the angle $\theta$
between them satisfies
$$ \cos\theta \;=\; \frac{\mbu\vdotprod\mbv}{|\mbu||\mbv|}. $$}\\[16pt]
Note that, when $\theta = \pi/2$, the numerator on the right-hand
side must be zero.  This motivates the following definition.

\newt
\subsection*{Orthogonality}

\vspace{0.1in}
\defn{Two vectors $\mbu$ and $\mbv$ are said to be {\em orthogonal}
(or {\em perpendicular}) if $\mbu\vdotprod\mbv = 0$.}

\newt
\eg{}The zero vector $\0$ is orthogonal to every other vector.
In 2D, the vectors $\left<a,b\right>$ and $\left<-b,a\right>$
are orthogonal, since
$$ \left<a,b\right>\vdotprod \left<-b,a\right> \;=\; (a)(-b)
	+ (b)(a) \;=\; 0. $$

\np
\ni
\eg{}
Find an equation for the plane containing the point $(1, 1, 2)$
and perpendicular to the vector $\left<A, B, C\right>$.

\newt
\subsection*{Projections}

{\bf Scalar component of $\mbu$ in the direction of $\mbv$}:
$\quad\ds{ |\mbu|\cos\theta \;=\; \frac{\mbu\vdotprod\mbv}{|\mbv|} }$.\\[20pt]
{\bf Vector projection of $\mbu$ onto $\mbv$}:
$$ \proj_{\mbv} \mbu \;:=\;
	\left(\begin{array}{c}
	  \mbox{scalar component of $\mbu$} \\
	  \mbox{in direction of $\mbv$}
	\end{array}\right)
	\left(\begin{array}{c}
	  \mbox{direction}\\
	  \mbox{of $\mbu$}
	\end{array}\right) \;=\; \left(\frac{\mbu\vdotprod\mbv}{|\mbv|}\right)
	\left(\frac{\mbv}{|\mbv|}\right) \;=\; \frac{\mbu\vdotprod\mbv}
	{|\mbv|^2}\,\mbv. $$

\newt
\subsection*{Work}

The {\em work} done by a constant force $\mbF$ acting through a
displacement vector $\mbD = \overrightarrow{PQ}$ is given by
$$ W \;=\; \mbF\vdotprod\mbD \;=\; |\mbF||\mbD|\cos\theta, $$
where $\theta$ is the angle between $\mbF$ and $\mbD$.

\end{document}

