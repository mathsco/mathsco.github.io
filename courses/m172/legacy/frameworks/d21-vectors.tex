\documentclass[12pt,fleqn]{article}

\newif\ifcomplete
\completetrue
%\completefalse
\def\topic{Vectors}
\def\lecdate{Fri., Mar.~9}

\oddsidemargin=0in
\textwidth=6.5in
\topmargin=-0.5in
\textheight=9.0in

\input mathMacros.tex
\usepackage{amsmath}
\usepackage{fancyhdr}
\usepackage{/Users/scofield/Library/texmf/latex/timestamp}
\usepackage[pdftex]{hyperref}
\hypersetup{pdfpagemode=None,colorlinks=true}
\usepackage{graphicx}
\usepackage[usenames,dvipsnames]{color}
\def\mbunsc{\blank{0.07in}}
%\def\RE{$\mathrm{Re}\,$}
%\def\IM{$\mathrm{Im}\,$}
\def\RE{$\mbox{Re}\,$}
\def\IM{$\mbox{Im}\,$}
\def\fb#1{\framebox{\parbox[t]{6.5in}{#1}}}
\def\sfb#1{\framebox{\parbox[t]{6.0in}{#1}}}
\def\defn#1{\fb{{\bf Definition}: #1}}
\def\sdefn#1{\sfb{{\bf Definition}: #1}}
\def\thm#1{\fb{{\bf Theorem}: #1}}
\def\sthm#1{\sfb{{\bf Theorem}: #1}}
\def\eg#1{{\bf Example}: #1}
\def\egs#1{{\bf Examples}: #1}
\def\newt{\vspace{0.2in}\ni}
\setlength{\fboxsep}{5pt}


\begin{document}

\pagestyle{fancy}
\fancyhf{}
\lhead{MATH 162---Framework for \lecdate}
\chead{}
\rhead{\topic}
\lfoot{}
\cfoot{\thepage}
\rfoot{}
\renewcommand{\headrulewidth}{0.4pt}
\renewcommand{\footrulewidth}{0.4pt}

\thispagestyle{empty}

\begin{center}
  \Large{MATH 162: Calculus II} \\
  \large{Framework for \lecdate} \\
  \large{\topic}
\end{center}

\vs{0.2in}
\ni
{\bf Today's Goal}: To understand vectors and be able to manipulate them.

\subsection*{Vectors in 2 and 3 Dimensions}
\defn{A {\em vector} is a directed line segment.  If $P$ and $Q$
are points in $\mathR^2$ or $\mathR^3$, then the directed line
segment from the {\em initial} point $P$ to the {\em terminal}
point $Q$ is denoted $\overrightarrow{PQ}$.}

\bi
\item
  {\bf Vector names}: bold-faced letters (usually lower-case) $\mbv$,
  or letters with arrows $\vec{v}$
\item
  There are two things that distinguish a vector $\mbv$:
  its {\em length} and its {\em direction}.
  Thus, two directed line segments which are parallel, have
  the same length, and are oriented in the same direction
  (arrow pointing the same way) are considered to be the
  same (equal) even if their initial and terminal points are
  different.
\item
  {\bf Component form}: Given what was said above, any vector
  $\mbv$ may be moved rigidly so as to make its initial point be
  the origin.  Writing $(v_1,v_2,v_3)$ for the resulting
  terminal point, we then say that $\mbv = \left<v_1, v_2, v_3\right>$.
  This is called the {\em component form} of $\mbv$.  The numbers
  $v_1$, $v_2$ and $v_3$ are the {\em components} of $\mbv$.
\item
  {\bf Equality of vectors}: Two vectors are considered equal
  when they are equal in each component.
\item
  If $\mbv = \left<v_1, v_2, v_3\right>$ then the {\em length}
  (or {\em magnitude}) of $\mbv$, denoted $|\mbv|$, is given by
  $$ |\mbv| \;:=\; \sqrt{v_1^2 + v_2^2 + v_3^2}. $$
  The only vector whose length is zero is the one whose
  components are all zero.  We call this the {\em zero vector},
  denoting it by $\0$.  The other vectors (the ones with nonzero
  length) are collectively referred to as {\em nonzero} vectors.
\ei

\subsection*{Vector Operations}

{\bf Vector Addition}: If $\mbu = \left<u_1, u_2, u_3\right>$
and $\mbv = \left<v_1, v_2, v_3\right>$, we define
$$ \mbu + \mbv \;:=\; \left<u_1 + v_1, u_2 + v_2, u_3 + v_3\right>. $$\\
{\bf Scalar Multiplication}: If $\mbv = \left<v_1, v_2, v_3\right>$
and $c$ is a real number (a {\em scalar}), then we define
$$ c\mbv \;:=\; \left<cv_1, cv_2, cv_3\right>. $$

\np
\ni
Note that:
\bi
\item
  Our definitions for vector addition and scalar multiplication
  are enough to give us the notion of {\em vector subtraction} as
  well, since we may think of $\mbu - \mbv$ as
  $$ \mbu + (-1)\mbv \;=\; \left<u_1, u_2, u_3\right> +
	\left<-v_1, -v_2, -v_3\right> \;=\; \left<u_1 - v_1,
	u_2 - v_2, u_3 - v_3\right>. $$
\item
  We make sense of an expression like $\mbv/c$ (i.e., dividing
  a vector by a scalar) by thinking of it as $(1/c)\mbv$
  (i.e., the reciprocal of $c$ multiplied by $\mbv$).  For
  any nonzero vector $\mbv$, $\mbv/|\mbv|$ is a vector whose
  length is 1, called the {\em direction} of $\mbv$.
\item
  No attempt has been made to define any type of multiplication
  (not yet) nor division (never!) between two vectors.
\ei

\subsection*{Unit Vectors}

Any vector whose length is 1 is called a {\em unit vector}.

\newt
\eg{}
For each $\mbv \ne \0$, the direction $\ds{\frac{\mbv}{|\mbv|}}$
of $\mbv$ is a unit vector.  Thus, in 2D, the vector
$\mbv = \left<-2, 5\right>$ has direction
$$ \mbd \;=\; \frac{\left<-2,5\right>}{\sqrt{(-2)^2 + 5^2}} \;=\;
	\left<\frac{-2}{\sqrt{29}}, \frac{5}{\sqrt{29}}\right>. $$
We may then write $\mbv$ as a product of its magnitude times its
direction
$$ \mbv \;=\; \sqrt{29} \mbd. $$

\newt
{\bf Standard unit vectors} (the ones parallel to the
coordinate axes): $\mbi := \left<1,0,0\right>$,
$\mbj := \left<0,1,0\right>$, and $\mbk := \left<0,0,1\right>$.

\newt
Notice that, for $\mbv = \left<v_1, v_2, v_3\right>$, it is the
case that
$$ \mbv \;=\; v_1 \mbi + v_2 \mbj + v_3 \mbk. $$

\end{document}

