\documentclass[12pt,fleqn]{article}

\newif\ifcomplete
\completetrue
%\completefalse
\def\topic{Chain Rules}
\def\lecdate{Thurs., Mar.~8}

\oddsidemargin=0in
\textwidth=6.5in
\topmargin=-0.5in
\textheight=9.0in

\input mathMacros.tex
\usepackage{amsmath}
\usepackage{fancyhdr}
\usepackage{/Users/scofield/Library/texmf/latex/timestamp}
\usepackage[pdftex]{hyperref}
\hypersetup{pdfpagemode=None,colorlinks=true}
\usepackage{graphicx}
\usepackage[usenames,dvipsnames]{color}
\def\unsc{\blank{0.07in}}
%\def\RE{$\mathrm{Re}\,$}
%\def\IM{$\mathrm{Im}\,$}
\def\RE{$\mbox{Re}\,$}
\def\IM{$\mbox{Im}\,$}
\def\fb#1{\framebox{\parbox[t]{6.5in}{#1}}}
\def\sfb#1{\framebox{\parbox[t]{6.0in}{#1}}}
\def\defn#1{\fb{{\bf Definition}: #1}}
\def\sdefn#1{\sfb{{\bf Definition}: #1}}
\def\thm#1{\fb{{\bf Theorem}: #1}}
\def\sthm#1{\sfb{{\bf Theorem}: #1}}
\def\eg#1{{\bf Example}: #1}
\def\egs#1{{\bf Examples}: #1}
\def\newt{\vspace{0.2in}\ni}
\setlength{\fboxsep}{5pt}


\begin{document}

\pagestyle{fancy}
\fancyhf{}
\lhead{MATH 162---Framework for \lecdate}
\chead{}
\rhead{\topic}
\lfoot{}
\cfoot{\thepage}
\rfoot{}
\renewcommand{\headrulewidth}{0.4pt}
\renewcommand{\footrulewidth}{0.4pt}

\thispagestyle{empty}

\begin{center}
  \Large{MATH 162: Calculus II} \\
  \large{Framework for \lecdate} \\
  \large{\topic}
\end{center}

\vs{0.2in}
\ni
{\bf Today's Goal}: To extend the chain rule to functions of multiple variables.

\vspace{0.3in}
\ni
{\bf\large Chain rule, single (independent) variable case}

\vspace{0.15in}
\ni
\begin{minipage}[t]{4in}
{\bf Setting}: $y$ is a function of $x$, while $x$ is a function of $t$.
More explicitly, $y = y(x)$, and $x = x(t)$ (so $y = y(x(t))$). \\[16pt]
{\bf Chain Rule}:
$\ds{ \quad\frac{dy}{dt} \;=\; \frac{dy}{dx}\frac{dx}{dt}}$\\[20pt]
Note here that
\bi
\item $y$ is the (final) dependent variable.
\item $t$ is the independent variable.
\item $x$ is an intermediate variable.
\ei
\end{minipage}\hspace{1.0in}
\setlength{\unitlength}{2mm}
\begin{minipage}[t]{2in}
  \mbox{}

  \begin{picture}(10,20)
	\put(5,0){\circle*{1}}
	\put(5,10){\circle*{1}}
	\put(5,20){\circle*{1}}
	\put(3,0){$t$}
	\put(7,5){$\ds{\frac{dx}{dt}}$}
	\put(3,10){$x$}
	\put(7,15){$\ds{\frac{dy}{dx}}$}
	\put(3,20){$y$}
	\thicklines
	\put(5,0){\vector(0,1){6}}
	\put(5,6){\vector(0,1){10}}
	\put(5,16){\line(0,1){4}}
  \end{picture}
\end{minipage}

\vspace{0.3in}
\ni
{\bf\large Many Multivariate Chain Rules}

\vspace{0.15in}
\ni
\begin{minipage}[t]{5in}
  {\bf Setting 1}: $\; z = f(x,y)$, with $x = x(t)$, $y = y(t)$ \\[16pt]
  {\bf Chain Rule}:
  $\ds{ \quad\frac{dz}{dt} \;=\; \frac{\partial z}{\partial x}\frac{dx}{dt}}
	+ \frac{\partial z}{\partial y}\frac{dy}{dt}$

  \vspace{0.7in}
  \ni
  {\bf Setting 2}: $\; w = f(x,y,z)$, with $x = x(t)$, $y = y(t)$ \\[16pt]
  {\bf Chain Rule}:
  $\ds{ \quad\frac{dw}{dt} \;=\; \frac{\partial w}{\partial x}\frac{dx}{dt}
	+ \frac{\partial w}{\partial y}\frac{dy}{dt}
	+ \frac{\partial w}{\partial z}\frac{dz}{dt}}$

  \vspace{0.6in}
  \ni
  {\bf Setting 3}: $\; z = f(x,y)$, with $x = x(u,v)$, $y = y(u,v)$,
	\\[16pt]
  {\bf Chain Rules}:
  $ \quad \ds{
	\left\{\begin{array}{ll}
	  \ds{\frac{\partial z}{\partial u} \;=\; \frac{\partial z}{\partial x}
		\frac{\partial x}{\partial u} + \frac{\partial z}{\partial y}
		\frac{\partial y}{\partial u}} \\[16pt]
	  \ds{\frac{\partial z}{\partial v} \;=\; \frac{\partial z}{\partial x}
		\frac{\partial x}{\partial v} + \frac{\partial z}{\partial y}
		\frac{\partial y}{\partial v}}
	\end{array}\right.} $
\end{minipage}
\begin{minipage}[t]{1.5in}
  \mbox{}

  \vspace{-0.3in}
  \begin{picture}(20,18)
	\put(0,8){$t$}
	\put(20,8){$z$}
	\put(9,17){$y$}
	\put(3,14){$\ds{\frac{dy}{dt}}$}
	\put(14,14){$\ds{\frac{\partial z}{\partial y}}$}
	\put(9,1){$x$}
	\put(3,3){$\ds{\frac{dx}{dt}}$}
	\put(14,3){$\ds{\frac{\partial z}{\partial x}}$}
	\put(2,9){\circle*{1}}
	\put(10,15){\circle*{1}}
	\put(10,3){\circle*{1}}
	\put(18,9){\circle*{1}}
	\thicklines
	\put(2,9){\vector(4,3){5}}
	\put(6,12){\line(4,3){4}}
	\put(2,9){\vector(4,-3){5}}
	\put(6,6){\line(4,-3){4}}
	\put(10,15){\vector(4,-3){5}}
	\put(14,12){\line(4,-3){4}}
	\put(10,3){\vector(4,3){5}}
	\put(14,6){\line(4,3){4}}
  \end{picture}
\end{minipage}

\section*{Another Look at Implicit Differentiation}

Many problems from MATH 161 in which implicit differentation was used
involved equations which could be put in the form $F(x,y) = 0$.
Assuming that this equation defines $y$ implicitly as a function of
$x$ (an assumption that is generally true), then by the chain rule
$$ \frac{dF}{dx} \;=\; \frac{\partial F}{\partial x}\frac{dx}{dx}
	+ \frac{\partial F}{\partial y}\frac{dy}{dx}
	\;=\; F_x + F_y\,\frac{dy}{dx}. $$
This is the $x$-derivative of one side of the equation $F(x,y) = 0$.
The $x$-derivative of the other side is, naturally, 0.  Thus, we have
$$ F_x + F_y\,\frac{dy}{dx} \;=\; 0 \qquad\Rarrow\qquad
	\frac{dy}{dx} \;=\; -\frac{F_x}{F_y}. $$

\end{document}

