\documentclass[12pt,fleqn]{article}

\newif\ifcomplete
\completetrue
%\completefalse
\def\topic{Taylor Series}
\def\lecdate{Fri., Feb.~23}

\oddsidemargin=0in
\textwidth=6.5in
\topmargin=-0.5in
\textheight=9.0in

\input mathMacros.tex
\usepackage{amsmath}
\usepackage{fancyhdr}
\usepackage{/Users/scofield/Library/texmf/latex/timestamp}
\usepackage[pdftex]{hyperref}
\hypersetup{pdfpagemode=None,colorlinks=true}
\usepackage{graphicx}
\usepackage[usenames,dvipsnames]{color}
\def\unsc{\blank{0.07in}}
%\def\RE{$\mathrm{Re}\,$}
%\def\IM{$\mathrm{Im}\,$}
\def\RE{$\mbox{Re}\,$}
\def\IM{$\mbox{Im}\,$}


\begin{document}

\pagestyle{fancy}
\fancyhf{}
\lhead{MATH 162---Framework for \lecdate}
\chead{}
\rhead{\topic}
\lfoot{}
\cfoot{\thepage}
\rfoot{}
\renewcommand{\headrulewidth}{0.4pt}
\renewcommand{\footrulewidth}{0.4pt}

\thispagestyle{empty}

\begin{center}
  \Large{MATH 162: Calculus II} \\
  \large{Framework for \lecdate} \\
  \large{\topic}
\end{center}

\vs{0.2in}
\ni
In the section on power series, we seemed to be
\bi
\item
  interested in finding power series expressions for
  various functions $f$,

\item
  but able to find such series only when the function $f$,
  or some order derivative/antiderivative of $f$,
  looked enough like $(1-x)^{-1}$ to make this feasible.
\ei
Our goal today is to find series expressions for important
functions that are not so closely linked to $(1-x)^{-1}$.
First, a definition:

\vspace{0.2in}
\ni
{\bf Definition}:
Suppose $f$ is a function which has derivatives of all orders
at $x=a$.  The {\em Taylor series for f at x = a} is
$$ \sum_{n=0}^\infty \frac{f^{(n)}(a)}{n!}\,(x - a)^n. $$

\vspace{0.2in}
\ni
What this definition says is that, for appropriate $f$, we may
construct a power series about $x=a$ employing the values of
derivatives $f^{(n)}(a)$ in the coefficients.  As of yet, no
assertion that this power series actually equals $f$ has been made.
(See note 2 below.)

\vspace{0.2in}
\ni
{\bf Some important notes}:
\be
\item
  The Taylor series, like any power series, has a radius of
  convergence $R$, which may be zero.
\item
  Even if the radius of convergence $R > 0$, the function defined
  by the Taylor series of $f$ might not equal $f$ except at the
  single location $x=a$.
\item
  But, if $R > 0$, then for many ``nice'' functions $f$, the Taylor
  series for $f$ equals $f$ on its entire interval of convergence.
\item
  If we stop the sum at the term containing $(x-a)^n$ (i.e., consider
  the partial sum of the series that includes as its last term the
  one with $(x-a)$ to the $n^{\mbox{th}}$ power), we get a polynomial
  of $n^{\mbox{th}}$ degree.  This polynomial is called the
  {\em Taylor polynomial of order n for f at x = a}.
\item
  If $a = 0$, then the Taylor series is called the {\em MacLaurin
  series of f}.
\item
  If $f$ equals any power series about $x=a$ at all, then
  that series must be the Taylor series.
\ee

\np
\ni
{\bf Some favorite Taylor series} (all of these are MacLaurin series)

\beqn
  \frac{1}{1-x} & = & \sum_{n=0}^\infty x^n \\
  & = & 1 + x + x^2 + \cdots + x^n + \cdots, \qquad -1 < x < 1 \\[10pt]
  e^x & = & \sum_{n=0}^\infty \frac{x^n}{n!} \\
  & = & 1 + x + \frac{x^2}{2} + \frac{x^3}{3!} + \cdots + \frac{x^n}{n!}
	+ \cdots, \qquad -\infty < x < \infty \\[10pt]
  \sin x & = & \sum_{n=0}^\infty (-1)^n \,\frac{x^{2n+1}}{(2n + 1)!} \\
  & = & x - \frac{x^3}{3!} + \frac{x^5}{5!} + \cdots + (-1)^n \frac{x^{2n+1}}
	{(2n+1)!} + \cdots, \qquad -\infty < x < \infty \\[10pt]
  \cos x & = & \sum_{n=0}^\infty (-1)^n \,\frac{x^{2n}}{(2n)!} \\
  & = & 1 - \frac{x^2}{2!} + \frac{x^4}{4!} + \cdots + (-1)^n \frac{x^{2n}}
	{(2n)!} + \cdots, \qquad -\infty < x < \infty \\[10pt]
  \arctan x & = & \sum_{n=0}^\infty (-1)^n \frac{x^{2n+1}}{(2n+1)} \\
  & = & x - \frac{x^3}{3} + \frac{x^5}{5} + \cdots
	+ (-1)^n \frac{x^{2n+1}}{2n+1} + \cdots, \qquad -1 \le x \le 1 \\[10pt]
  \ln(1 + x) & = & \sum_{n=1}^\infty (-1)^{n+1}\frac{x^n}{n} \\
  & = & x - \frac{x^2}{2} + \frac{x^3}{3} + \cdots
	+ (-1)^{n+1}\frac{x^n}{n} + \cdots, \qquad -1 < x \le 1
%	\\[10pt]
% \ln x & = & \sum_{n=1}^\infty (-1)^{n+1}\frac{(x-1)^n}{n} \\
% & = & (x - 1) - \frac{(x-1)^2}{2} + \frac{(x-1)^3}{3} + \cdots
%	+ (-1)^{n+1}\frac{(x-1)^n}{n} + \cdots, \qquad 0 < x \le 2
\eeqn

\vspace{0.2in}
\ni
As with expressions that were similar to $(1-x)^{-1}$, we may
substitute into these power series to get power series expressions
for other, related functions.

\vspace{0.2in}
\ni
{\bf Example}:
The MacLaurin series converging to $exp(-x^2)$ is
$$ exp(-x^2) \;=\; 1 - x^2 + \frac{x^4}{2} - \frac{x^6}{3!} + \frac{x^8}{4!}
	+ \cdots + (-1)^n \frac{x^{2n}}{n!} + \cdots. $$

\end{document}

