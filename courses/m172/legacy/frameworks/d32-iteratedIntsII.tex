\documentclass[12pt,fleqn]{article}

\newif\ifcomplete
\completetrue
%\completefalse
\def\topic{Double Integrals, General Regions}
\def\lecdate{Fri., Apr.~13}

\oddsidemargin=0in
\textwidth=6.5in
\topmargin=-0.5in
\textheight=9.0in

\input mathMacros.tex
\usepackage{amsmath}
\usepackage{ascii}
%\usepackage{wasysym}
\usepackage{fancyhdr}
\usepackage{/Users/scofield/Library/texmf/latex/timestamp}
\usepackage[pdftex]{hyperref}
\hypersetup{pdfpagemode=None,colorlinks=true}
\usepackage{graphicx}
\usepackage[usenames,dvipsnames]{color}
\def\mbunsc{\blank{0.07in}}
\def\pref#1{(\ref{#1})}
%\def\RE{$\mathrm{Re}\,$}
%\def\IM{$\mathrm{Im}\,$}
\def\RE{$\mbox{Re}\,$}
\def\IM{$\mbox{Im}\,$}
\def\fb#1{\framebox{\parbox[t]{6.5in}{#1}}}
\def\sfb#1{\framebox{\parbox[t]{6.0in}{#1}}}
\def\defn#1{\fb{{\bf Definition}: #1}}
\def\sdefn#1{\sfb{{\bf Definition}: #1}}
\def\thm#1{\fb{{\bf Theorem}: #1}}
\def\sthm#1{\sfb{{\bf Theorem}: #1}}
\def\eg#1{{\bf Example}: #1}
\def\egs#1{{\bf Examples}: #1}
\def\newt{\vspace{0.2in}\ni}
\setlength{\fboxsep}{5pt}
%\def\vdotprod{{\ascii\BEL}}
\def\vdotprod{\,\mbox{{\Large $\cdot$}}\,}
\def\vcrossprod{\times}


\begin{document}

\pagestyle{fancy}
\fancyhf{}
\lhead{Framework for \lecdate}
\chead{}
\rhead{\topic}
\lfoot{}
\cfoot{\thepage}
\rfoot{}
\renewcommand{\headrulewidth}{0.4pt}
\renewcommand{\footrulewidth}{0.4pt}

\thispagestyle{empty}

\begin{center}
  \Large{MATH 162: Calculus II} \\
  \large{Framework for \lecdate} \\
  \large{\topic}
\end{center}

\vs{0.2in}
\ni
{\bf Today's Goal}:
To understand the meaning of double integrals over
more general bounded regions $R$ of the plane, and to be
able to evaluate such integrals.

\vspace{0.15in}
\ni
{\bf Important Note}: In conjunction with this framework,
you should look over Section 13.2 of your text.

\vspace{0.15in}
\subsection*{Double Integrals as Iterated Integrals: General Treatment}

{\bf{Q:}} What if we seek $\iint_R f(x,y)\,dA$ when $R$ is not a
rectangle whose sides are parallel to the coordinate axes?
\bi
\item[{\bf{A1:}}]
  If $R$ is a ``nice enough'' region (and, for our study,
  it will be), we can, once again, define $\iint_R f(x,y)\,dA$ in
  terms of Riemann sums.  The twist here is that, for any given
  partition of $R$, the rectangles will only partially fill up $R$.

  We will not pursue this train of thought further.

\item[{\bf{A2:}}]
  Use a more general form of Fubini's theorem.  You can see
  the formal statement on p.~792 of your text.  It deals with
  2 cases (pictured): \\[16pt]
  \includegraphics[width=6in]{figs/fubiniGeneralRegions.png}
  \bi
  \item[{\bf{Case 1:}}]
	the upper and lower boundaries of $R$ each are functions
	of $x$ on a common interval; that is, a region
	$R:\; a \le x \le b, \; g_1(x) \le y \le g_2(x)$.  Then
	$$ \iint\limits_R f(x,y)\,dA \;=\; \int_a^b \int_{g_1(x)}^{g_2(x)}
		f(x,y)\,dy \,dx. $$

	\eg{}
	Evaluate $\;\iint_R (x + 2y)\,dA\;$ over the region $R$
	that lies between the parabolas $\;y = 2x^2\;$ and
	$\;y = 1 + x^2$.
	\vspace{2.5in}

  \item[{\bf{Case 2:}}]
	the left and right boundaries of $R$ each are functions
	of $y$ on a common interval; that is, a region
	$R:\; c \le y \le d, \; h_1(x) \le x \le h_2(x)$.  Then
	$$ \iint\limits_R f(x,y)\,dA \;=\; \int_c^d \int_{h_1(y)}^{h_2(y)}
		f(x,y)\,dx \,dy. $$

	\eg{}
	Set up an integral for a function $f(x,y)$ over the region
	bounded by the $y$-axis and the curve $x + y^2 = 1$.
  \ei
\ei

\end{document}

