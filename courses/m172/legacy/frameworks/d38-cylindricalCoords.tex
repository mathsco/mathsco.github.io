\documentclass[12pt,fleqn]{article}

\newif\ifcomplete
\completetrue
%\completefalse
\def\topic{Integration in Cylindrical Coordinates}
\def\lecdate{Mon., Apr.~30}

\oddsidemargin=0in
\textwidth=6.5in
\topmargin=-0.5in
\textheight=9.0in

\input mathMacros.tex
\usepackage{amsmath}
\usepackage{ascii}
%\usepackage{wasysym}
\usepackage{fancyhdr}
\usepackage{/Users/scofield/Library/texmf/latex/timestamp}
\usepackage[pdftex]{hyperref}
\hypersetup{pdfpagemode=None,colorlinks=true}
\usepackage{graphicx}
\usepackage[usenames,dvipsnames]{color}
\def\mbunsc{\blank{0.07in}}
\def\pref#1{(\ref{#1})}
%\def\RE{$\mathrm{Re}\,$}
%\def\IM{$\mathrm{Im}\,$}
\def\RE{$\mbox{Re}\,$}
\def\IM{$\mbox{Im}\,$}
\def\fb#1{\framebox{\parbox[t]{6.5in}{#1}}}
\def\sfb#1{\framebox{\parbox[t]{6.0in}{#1}}}
\def\defn#1{\fb{{\bf Definition}: #1}}
\def\sdefn#1{\sfb{{\bf Definition}: #1}}
\def\thm#1{\fb{{\bf Theorem}: #1}}
\def\sthm#1{\sfb{{\bf Theorem}: #1}}
\def\eg#1{{\bf Example}: #1}
\def\egs#1{{\bf Examples}: #1}
\def\egsc#1{{\bf Examples #1:}}
\def\newt{\vspace{0.2in}\ni}
\setlength{\fboxsep}{5pt}
%\def\vdotprod{{\ascii\BEL}}
\def\vdotprod{\,\mbox{{\Large $\cdot$}}\,}
\def\vcrossprod{\times}


\begin{document}

\pagestyle{fancy}
\fancyhf{}
\lhead{Framework for \lecdate}
\chead{}
\rhead{\topic}
\lfoot{}
\cfoot{\thepage}
\rfoot{}
\renewcommand{\headrulewidth}{0.4pt}
\renewcommand{\footrulewidth}{0.4pt}

\thispagestyle{empty}

\begin{center}
  \Large{MATH 162: Calculus II} \\
  \large{Framework for \lecdate} \\
  \large{\topic}
\end{center}

\vs{0.2in}
\ni
{\bf Today's Goal}:
To develop an understanding of cylindrical and spherical
coordinates, and to learn to set up and evaluate triple
integrals in cylindrical coordinates.

\vspace{0.15in}
\ni
{\bf Important Note}: In conjunction with this framework,
you should look over Section 13.7 of your text.

\vspace{0.15in}
\subsection*{Coordinate Systems for 3D Space}

\begin{minipage}[t]{4.5in}
\bi
\item {\bf Rectangular Coordinates}:
  Generally uses letters $(x,y,z)$.  It tells how far to travel
  in directions parallel to three orthogonal coordinate axes
  in order to arrive at the specified point.

\item {\bf Cylindrical Coordinates}
  Generally uses letters $(r,\theta,z)$.  Here $z$ should
  be understood in exactly the same way as it is for
  rectangular coordinates, while $r$ and $\theta$ are
  polar coordinates for the shadow point in the $xy$-plane.
  (See the top figure.)  Each of the three coordinates may take
  any value in $\mathR$.

\item {\bf Spherical Coordinates}
  Generally uses letters $(\rho, \phi, \theta)$.  The meaning
  of $\theta$ is precisely the same as with cylindrical
  coordinates.  If a ray is drawn from the origin to the point
  in question, then $\rho$ is the distance along that ray to
  the point, while $\phi$ is the angle that ray makes with the
  $z$-axis.  It is possible to identify all points of 3D-space
  using values for the three coordinates which satisfy
  $$ \rho \ge 0, \qquad 0 \le \theta \le 2\pi, \qquad
	0 \le \phi \le \pi. $$
\ei
\end{minipage}\hspace{0.3in}
\begin{minipage}[t]{2in}
  \mbox{}

  \vspace{-0.5in}
  \includegraphics[width=2in]{figs/cylindCoords.png}\\[20pt]

  \includegraphics[width=2in]{figs/sphCylCoords.png}
\end{minipage}

\subsection*{Conversions between Coordinates}
The Pythagorean Theorem and trigonometry give
$$ r \;=\; \sqrt{x^2 + y^2} \;=\; \rho\sin\phi, \and
  \rho \;=\; \sqrt{x^2 + y^2 + z^2}. $$
Thus,
$$ \begin{array}{l}
  x \;=\; r\cos\theta \;=\; \rho\sin\phi\cos\theta, \\
  y \;=\; r\sin\theta \;=\; \rho\sin\phi\sin\theta, \\
  z \;=\; \rho\cos\phi \quad\mbox{(by trigonometry)}.
  \end{array} $$

\subsection*{Triple Integrals in Cylindrical Coordinates}

When computing the triple integral $\iiint_D f(x,y,z)\,dV$,
one can choose any of the three coordinate systems discussed
above.  Cylindrical coordinates are attractive when
\bi
\item
  the boundary of the shadow region in the plane may be
  expressed nicely as a combination of polar functions, and/or
\item
  the integrand $f(r\cos\theta, r\sin\theta, z)$ is simple.
\ei
For iterated integrals in cylindrical coordinates, the volume
element is $dV = r \,dz \,dr \,d\theta$, or some permutation
of this.  Thus,
$$ \iiint\limits_D f(x,y,z)\,dV \;=\; \iiint\limits_D
	f(r\cos\theta, r\sin\theta, z)r \,dz\,dr\,d\theta. $$
\egs{}
\be
\item
  Show that the volume of a sphere with radius $a$ is what we
  think it should be.
\item
  Find the volume of a ``cored apple''---a sphere of radius
  $a$ from which a cylindrical region of radius $b$ ($b < a$)
  has been removed.
\item
  Find the volume of a typical cone whose radius at the base
  is $a$ and whose height is $h$.
\ee

\end{document}

