\documentclass[12pt,fleqn]{article}

\newif\ifcomplete
\completetrue
%\completefalse
\def\topic{Limits and Continuity}
\def\lecdate{Wed., Feb.~28}

\oddsidemargin=0in
\textwidth=6.5in
\topmargin=-0.5in
\textheight=9.0in

\input mathMacros.tex
\usepackage{amsmath}
\usepackage{fancyhdr}
\usepackage{/Users/scofield/Library/texmf/latex/timestamp}
\usepackage[pdftex]{hyperref}
\hypersetup{pdfpagemode=None,colorlinks=true}
\usepackage{graphicx}
\usepackage[usenames,dvipsnames]{color}
\def\unsc{\blank{0.07in}}
%\def\RE{$\mathrm{Re}\,$}
%\def\IM{$\mathrm{Im}\,$}
\def\RE{$\mbox{Re}\,$}
\def\IM{$\mbox{Im}\,$}
\def\fb#1{\framebox{\parbox[t]{6.5in}{#1}}}
\def\sfb#1{\framebox{\parbox[t]{6.0in}{#1}}}
\def\defn#1{\fb{{\bf Definition}: #1}}
\def\sdefn#1{\sfb{{\bf Definition}: #1}}
\def\thm#1{\fb{{\bf Theorem}: #1}}
\def\sthm#1{\sfb{{\bf Theorem}: #1}}
\def\eg#1{{\bf Example}: #1}
\def\egs#1{{\bf Examples}: #1}
\setlength{\fboxsep}{5pt}


\begin{document}

\pagestyle{fancy}
\fancyhf{}
\lhead{MATH 162---Framework for \lecdate}
\chead{}
\rhead{\topic}
\lfoot{}
\cfoot{\thepage}
\rfoot{}
\renewcommand{\headrulewidth}{0.4pt}
\renewcommand{\footrulewidth}{0.4pt}

\thispagestyle{empty}

\begin{center}
  \Large{MATH 162: Calculus II} \\
  \large{Framework for \lecdate} \\
  \large{\topic}
\end{center}

\vs{0.2in}
\ni
{\bf Today's Goal}: To understand the meaning of limits and
continuity of functions of 2 and 3 variables.

\section*{Geometry of the Domain Space}

\defn{(Distance in $\mathR^3$): Suppose that $(x_1,y_1,z_1)$
and $(x_2,y_2,z_2)$ are points in $\mathR^3$.  The {\em distance}
between these two points is
$$ \sqrt{(x_1 - x_2)^2 + (y_1 - y_2)^2 + (z_1 - z_2)^2}. $$}

\vspace{0.1in}
\ni
If our two points have the same $z$-value (for instance, if
they both lie in the $xy$-plane), then the distance between
them is just
$$ \sqrt{(x_1 - x_2)^2 + (y_1 - y_2)^2}. $$
We employ these definitions for distance in $\mathR^2$, $\mathR^3$
to define circles, disks, spheres and balls.

\vspace{0.2in}
\ni
\defn{A {\em circle} in $\mathR^2$ centered at $(a,b)$ with radius
$r$ is the set of points $(x,y)$ satisfying
$$ (x - a)^2 + (y - b)^2 \;=\; r^2. $$
Given such a circle $C$, the set of all points on or inside $C$
is a {\em closed disk}.  The set of points inside but not on $C$
is an {\em open disk}.}

\vspace{0.2in}
\ni
\defn{A {\em sphere} in $\mathR^3$ centered at $(a,b,c)$ with radius
$r$ is the set of points $(x,y,z)$ satisfying
$$ (x - a)^2 + (y - b)^2 + (z - c)^2 \;=\; r^2. $$
The inside of a sphere is called a {\em ball}, and can be closed
or open depending on whether the (whole) sphere is included.}

\vspace{0.2in}
\ni
Open intervals in $\mathR$ are sets of the form
$a < x < b$ (also written using interval notation $(a,b)$), where
neither endpoint $a$, $b$ is included in the set.  One observation
about such sets is that, if you take any $x \in (a,b)$, there is a
value of $r > 0$, perhaps quite small, for which the interval
$(x - r, x + r)$ is wholly contained inside $(a, b)$.  We build
on that idea when defining various kinds of subsets of $\mathR^2$.

\vspace{0.2in}
\ni
\defn{Let $R$ be a region of the $xy$-plane and $(x_0, y_0)$ a point
(perhaps in $R$, perhaps not).  We call $(x_0,y_0)$ an {\em interior
point} of $R$ if there is an open disk of positive radius centered
at $(x_0,y_0)$ such that every point in this disk lies inside
$R$.\\[6pt]
We call $(x_0,y_0)$ a {\em boundary point} of $R$ if every disk
with positive radius centered at $(x_0,y_0)$ contains both a
point that is in $R$ and a point that isn't in $R$. \\[6pt]
The region $R$ is said to be {\em open} if all points in $R$
are interior points of $R$. \\[6pt]
The region $R$ is said to be {\em closed} if all boundary points
of $R$ are in $R$. \\[6pt]
The region $R$ is said to be {\em bounded} if it lies entirely
inside a disk of finite radius.}

\vspace{0.2in}
\ni
\egs{}
\be
\item
  An open disk is open.  A closed disk is closed.  For both,
  the boundary points are those found on the enclosing circle.
\item
  The upper half-plane $R$ consisting of points $(x,y)$ for
  which $y > 0$ is an open, unbounded set.  The boundary
  points of $R$ are precisely those points found along the
  $x$-axis, none of which are contained in $R$.
\item
  If $y = f(x)$ is a continuous function on the interval
  $a \le x \le b$, then the graph of $f$ is a closed, bounded
  set made up entirely of boundary points.
\item
  The set of points $(x,y)$ satisfying $x \ge 0$, $y \ge 0$
  and $y < x+1$ is bounded, but neither open nor closed.
\item
  The only nonempty region of the plane which is both open
  and closed is the entire plane $\mathR^2$.
\ee

\section*{Limits and Continuity}

The idea behind the statement
$$ \lim_{(x,y)\rightarrow (x_0,y_0)} f(x,y) \;=\; L $$
for a function $f$ of two variables is that you can make the
value of $f(x,y)$ as close to $L$ as you like by focusing only
on $(x,y)$ in the domain of $f$ which are inside an open disk
centered at $(x_0,y_0)$ of some positive radius.  The official
definition follows.

\vspace{0.2in}
\ni
\defn{The {\em limit} of $f$ as $(x,y)$ approaches $(x_0,y_0)$
is $L$ if for every $\epsilon > 0$ there is a corresponding
$\delta > 0$ such that, for all $(x,y)$ in $\dom(f)$,
$$ \mbox{if} \quad 0 < \sqrt{(x - x_0)^2 + (y - y_0)^2} < \delta
	\quad\mbox{then}\quad |f(x,y) - L| < \epsilon. $$}

\vspace{0.2in}
\ni
Remarks:
\be
\item
  All the limit laws of functions of a single variable---those
  stated in Section 2.2---have analogs for functions of 2 variables.
  For instance, the analog to Rule 5 on p.~65 goes like this:

  \vspace{0.1in}
  \ni
  \sthm{Suppose
  $$ \lim_{(x,y)\rightarrow(x_0,y_0)} f(x,y) = L
	\qquad\mbox{and}\qquad \lim_{(x,y)\rightarrow(x_0,y_0)}
	g(x,y) = M. $$
  (That is, suppose both these limits exist, and call them
  $L$, $M$ respectively.)  If $M \ne 0$, then
  $$ \lim_{(x,y)\rightarrow(x_0,y_0)} \frac{f(x,y)}{g(x,y)}
	= \frac{L}{M}. $$}

\item
  We recall that, for a function $f$ of a single variable and
  a point $x_0$ interior to the domain of $f$, the statement
  $\lim_{x\rightarrow\x_0} f(x) = L$ requires the values of
  $f$ to approach $L$ as the $x$-values approach $x_0$ from
  both the left and the right.  The requirement at interior points
  $(x_0,y_0)$ to the domain of a function $f$ of 2 variables
  is even more strict.  Any path of $(x,y)$-values within the
  domain of $f$ traversed en route to $(x_0,y_0)$ should
  produce function values which approach $L$.

  \vspace{0.1in}
  \ni
  \eg{} The limits $\;\ds{\lim_{(x,y)\rightarrow(0,0)}
  \frac{4xy}{x^2 + y^2}}\;$ and $\;\ds{\lim_{(x,y)\rightarrow(0,0)}
  \frac{4xy^2}{x^2+y^4}}\;$ do not exist.
\vspace{0.1in}
\item
  One defines continuity for functions of two variables in
  an identical fashion as for functions of a single variable.

  \vs{0.1in}
  \sdefn{Suppose $(x_0,y_0)$ is in the domain of $f$.  We say
  that {\em f is continuous at} $(x_0, y_0)$ if
  $$ \lim_{(x,y)\rightarrow(x_0,y_0)} f(x,y) = f(x_0, y_0) $$
  (i.e., if this limit exists and has the same value as
  $f(x_0,y_0)$).  We say {\em f is continuous} (or {\em f is
  continuous throughout its domain}) if $f$ is continuous at
  each point in its domain.
  }

  \vspace{0.1in}
  \ni
  \eg{} The functions $\;\ds{f(x,y) := \frac{4xy}{x^2 + y^2}}\;$
  and $\;\ds{g(x,y) := \frac{4xy^2}{x^2+y^4}}\;$ are continuous
  at all points $(x,y)$ except $(0,0)$.
\vspace{0.1in}
\item
  The notions of {\em distance}, {\em open} and {\em closed balls},
  {\em boundary point}, {\em interior point}, {\em open},
  {\em closed} and {\em bounded sets}, {\em limits} and {\em continuity}
  may all be generalized to functions of $n$ variables, where $n$
  is any integer greater than 1.

\ee

\end{document}

