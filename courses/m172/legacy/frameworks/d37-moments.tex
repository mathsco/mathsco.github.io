\documentclass[12pt,fleqn]{article}

\newif\ifcomplete
\completetrue
%\completefalse
\def\topic{Moments, Centers of Mass}
\def\lecdate{Thurs., Apr.~26--Fri., Apr.~27}

\oddsidemargin=0in
\textwidth=6.5in
\topmargin=-0.5in
\textheight=9.0in

\input mathMacros.tex
\usepackage{amsmath}
\usepackage{ascii}
%\usepackage{wasysym}
\usepackage{fancyhdr}
\usepackage{/Users/scofield/Library/texmf/latex/timestamp}
\usepackage[pdftex]{hyperref}
\hypersetup{pdfpagemode=None,colorlinks=true}
\usepackage{graphicx}
\usepackage[usenames,dvipsnames]{color}
\def\mbunsc{\blank{0.07in}}
\def\pref#1{(\ref{#1})}
%\def\RE{$\mathrm{Re}\,$}
%\def\IM{$\mathrm{Im}\,$}
\def\RE{$\mbox{Re}\,$}
\def\IM{$\mbox{Im}\,$}
\def\fb#1{\framebox{\parbox[t]{6.5in}{#1}}}
\def\sfb#1{\framebox{\parbox[t]{6.0in}{#1}}}
\def\defn#1{\fb{{\bf Definition}: #1}}
\def\sdefn#1{\sfb{{\bf Definition}: #1}}
\def\thm#1{\fb{{\bf Theorem}: #1}}
\def\sthm#1{\sfb{{\bf Theorem}: #1}}
\def\eg#1{{\bf Example}: #1}
\def\egs#1{{\bf Examples}: #1}
\def\egsc#1{{\bf Examples #1:}}
\def\newt{\vspace{0.2in}\ni}
\setlength{\fboxsep}{5pt}
%\def\vdotprod{{\ascii\BEL}}
\def\vdotprod{\,\mbox{{\Large $\cdot$}}\,}
\def\vcrossprod{\times}


\begin{document}

\pagestyle{fancy}
\fancyhf{}
\lhead{Framework for \lecdate}
\chead{}
\rhead{\topic}
\lfoot{}
\cfoot{\thepage}
\rfoot{}
\renewcommand{\headrulewidth}{0.4pt}
\renewcommand{\footrulewidth}{0.4pt}

\thispagestyle{empty}

\begin{center}
  \Large{MATH 162: Calculus II} \\
  \large{Framework for \lecdate} \\
  \large{\topic}
\end{center}

\vs{0.2in}
\ni
{\bf Today's Goal}:
To learn to apply the skill of setting up and evaluating
triple integrals in the context of finding centers of mass.

\vspace{0.15in}
\ni
{\bf Important Note}: In conjunction with this framework,
you should look over Section 13.6 of your text.

\vspace{0.15in}
\subsection*{Point Masses along a Line (1D)}
\bi
\item
  Assume $n$ masses $m_1$, \ldots, $m_n$ sit at
  locations $x_1$, \ldots, $x_n$ on a number line.
\item
  Call the location $\bar x$ about which the total torque
  is zero.  That is, $\bar x$ is the location to place a
  fulcrum so that
  $$ \sum_{k=1}^n m_k (x_k - \bar x) \;=\; 0. $$
  Solving for $\bar x$, we get
  \begin{equation}
	\qquad\qquad\qquad
	\bar x \;=\; \frac{\ds{\sum_{k=1}^n m_k x_k}}{\ds{\sum_{k=1}^n m_k}}
	  \;=:\; \frac{\mbox{1st moment about $x=0$}}{\mbox{total mass}}
		\label{discreteCOM}
  \end{equation}
\ei

\subsection*{Continuous Mass along a Line
 (1D; more instructive than practical)}
When
\bi
\item
  mass is distributed throughout a continuous body (instead of
  being concentrated at finitely-many distinct positions),
\item
  $\rho(x)$ gives the mass density (mass per unit length)
  inside the interval $a \le x \le b$,
\ei
then the total mass is
$$ \int_a^b \rho(x)\, dx. $$
The continuous analog to the numerator of \pref{discreteCOM}
comes from the following definition:\\[8pt]
\defn{
If $\rho(x)$ is the mass density (in mass per unit {\em length})
of a substance contained in a region $a \le x \le b$ of 1D space,
then the {\em first moment} about $x=0$ is given by
$$ \int_a^b x \rho(x)\,dx. $$
}\\
Using this, the corresponding center of mass is
\begin{equation}
  \qquad\qquad\qquad\qquad\qquad\qquad\qquad
  \bar x \;=\; \frac{\int_a^b x \rho(x)\, dx}{\int_q^b \rho(x)\, dx}
	\label{contCOM1D}
\end{equation}

\subsection*{Centers of Mass in 3D}
\defn{
If $\rho(x,y,z)$ is the mass density (in mass per unit volume)
of a substance contained in a region $D$ of 3D space, then the
{\em first moment} about the $yz$-plane ($x=0$) is given by
$$ M_{yz} \;:=\; \iiint\limits_D x \rho(x,y,z)\,dV. $$
Similarly, the first moments about the $xz$ and $xy$-planes are
$$ M_{xz} \;:=\; \iiint\limits_D y \rho(x,y,z)\,dV \and
	M_{xy} \;:=\; \iiint\limits_D z \rho(x,y,z)\,dV $$
respectively.
}\\[8pt]
The corresponding {\em center of mass} will reside at a point
$(\bar x, \bar y, \bar z)$ in space.  If we denote the total
mass in the region $D$ by
$$ M \;:=\; \iiint\limits_D \rho(x,y,z)\,dV, $$
then the {\bf coordinates of the center of mass} are given by
\begin{equation}
  \bar x \;=\; \frac{M_{yz}}{M}, \quad
	\bar y \;=\; \frac{M_{xz}}{M}, \and
	\bar z \;=\; \frac{M_{xy}}{M}. \label{com3d}
\end{equation}
Remarks:
\bi
\item
  One can develop formulas for the position $(\bar x, \bar y)$
  of the center of mass in two dimensions.  In that case it is
  assumed $\rho(x,y)$ gives mass density in mass per unit area,
  and the first moments $M_y = \iint_R x\rho(x,y) \,dA$ and
  $M_x = \iint_R y\rho(x,y) \,dA$ are {\em about the x-axis and
  y-axis} respectively.
\item
  When there is constant mass density $\rho(x,y,z) = \delta$
  throughout the region $D$, then another name for the center
  of mass is the {\em centroid}.  In this case,
  $$ \bar x \;=\; \frac{M_{yz}}{M} \;=\; \frac{\delta \iiint\limits_D 
	x\,dV}{\iiint\limits_D \delta \,dV}
	\;=\; \frac{\iiint\limits_D x\,dV}{\iiint\limits_D dV}
	\;=\; \mbox{avg.~$x$-value in $D$}. $$
  Similar statements may be made about the other coordinates
  $\bar y$ and $\bar z$ of the centroid.
\ei

\end{document}

