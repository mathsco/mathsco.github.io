\documentclass[12pt,fleqn]{article}

\newif\ifcomplete
\completetrue
%\completefalse
\def\topic{The Gradient Vector}
\def\lecdate{Thurs., Mar.~29--Fri. Mar.~30}

\oddsidemargin=0in
\textwidth=6.5in
\topmargin=-0.5in
\textheight=9.0in

\input mathMacros.tex
\usepackage{amsmath}
\usepackage{ascii}
%\usepackage{wasysym}
\usepackage{fancyhdr}
\usepackage{/Users/scofield/Library/texmf/latex/timestamp}
\usepackage[pdftex]{hyperref}
\hypersetup{pdfpagemode=None,colorlinks=true}
\usepackage{graphicx}
\usepackage[usenames,dvipsnames]{color}
\def\mbunsc{\blank{0.07in}}
\def\pref#1{(\ref{#1})}
%\def\RE{$\mathrm{Re}\,$}
%\def\IM{$\mathrm{Im}\,$}
\def\RE{$\mbox{Re}\,$}
\def\IM{$\mbox{Im}\,$}
\def\fb#1{\framebox{\parbox[t]{6.5in}{#1}}}
\def\sfb#1{\framebox{\parbox[t]{6.0in}{#1}}}
\def\defn#1{\fb{{\bf Definition}: #1}}
\def\sdefn#1{\sfb{{\bf Definition}: #1}}
\def\thm#1{\fb{{\bf Theorem}: #1}}
\def\sthm#1{\sfb{{\bf Theorem}: #1}}
\def\eg#1{{\bf Example}: #1}
\def\egs#1{{\bf Examples}: #1}
\def\newt{\vspace{0.2in}\ni}
\setlength{\fboxsep}{5pt}
%\def\vdotprod{{\ascii\BEL}}
\def\vdotprod{\,\mbox{{\Large $\cdot$}}\,}
\def\vcrossprod{\times}


\begin{document}

\pagestyle{fancy}
\fancyhf{}
\lhead{MATH 162---Framework for \lecdate}
\chead{}
\rhead{\topic}
\lfoot{}
\cfoot{\thepage}
\rfoot{}
\renewcommand{\headrulewidth}{0.4pt}
\renewcommand{\footrulewidth}{0.4pt}

\thispagestyle{empty}

\begin{center}
  \Large{MATH 162: Calculus II} \\
  \large{Framework for \lecdate} \\
  \large{\topic}
\end{center}

\vs{0.2in}
\ni
{\bf Today's Goal}:
To learn about the gradient vector $\vec\nabla f$ and
its uses, where $f$ is a function of two or three variables.

\vspace{0.1in}
\subsection*{The Gradient Vector}
Suppose $f$ is a differentiable function of two variables $x$ and
$y$ with domain $R$, an open region of the $xy$-plane.  Suppose
also that
$$ \mbr(t) = x(t)\mbi + y(t)\mbj, \qquad t\in I, $$
(where $I$ is some interval) is a differentiable vector function
(parametrized curve) with $(x(t), y(t))$ being a point in $R$
for each $t\in I$.  Then by the chain rule,
\begin{eqnarray}
  \frac{d}{dt} f(x(t), y(t)) & = & \frac{\partial f}{\partial x}
	\frac{dx}{dt} + \frac{\partial f}{\partial y}\frac{dy}{dt} \nonumber \\
  & = & \left[f_x \mbi + f_y \mbj\right]
	\vdotprod \left[x'(t) \mbi + y'(t) \mbj\right] \nonumber \\
  & = & \left[f_x \mbi + f_y \mbj\right]
	\vdotprod \frac{d\mbr}{dt}. \label{totalD}
\end{eqnarray}
\defn{For a differentiable function $f(x_1, \ldots, x_n)$ of $n$
variables, we define the {\em gradient vector of} $f$ to be
$$ \vec\nabla f \;:=\; \left<\frac{\partial f}{\partial x_1},
  \frac{\partial f}{\partial x_2}, \ldots,
  \frac{\partial f}{\partial x_n}\right>. $$}\\[8pt]
Remarks:
\bi
\item
  Using this definition, the total derivative $df/dt$ calculated
  in \pref{totalD} above may be written as
  $$ \frac{df}{dt} \;=\; \vec\nabla f \vdotprod \mbr'. $$
  In particular, if
  $\mbr(t) = x(t)\mbi + y(t)\mbj + z(t)\mbk$, $t\in (a,b)$ is a
  differentiable vector function, and if $f$ is a function of
  3 variables which is differentiable at the point $(x_0 , y_0, z_0)$,
  where $x_0 = x(t_0)$, $y_0 = y(t_0)$, and $z_0 = z(t_0)$ for some
  $t_0 \in (a,b)$, then
  $$ \left.\frac{df}{dt}\right|_{t=t_0} \;=\;
	\vec\nabla f(x_0, y_0, z_0) \vdotprod \mbr'(t_0). $$

\item
  If $f$ is a function of 2 variables, then $\vec\nabla f$ has
  2 components.  Thus, while the graph of such an $f$ lives in
  3D, $\vec\nabla f$ should be thought of as a vector in
  the plane.

  If $f$ is a function of 3 variables, then $\vec\nabla f$ has
  3 components, and is a vector in 3-space.

  \newpage
  \begin{minipage}[t]{3.3in}
	Speaking more generally, we may say that while a function
	$f(x_1, \ldots, x_n)$ of $n$ variables requires $n$ inputs
	to produce a single (numeric) output, the corresponding
	gradient $\vec\nabla f$ produces from those same $n$ inputs
	a {\em vector} with $n$ components.  Objects which assign
	to each $n$-tuple input an $n$-vector output are known as
	{\em vector fields}.  The gradient is an example of a vector
	field.\\[14pt]
	\eg{}
	For $f(x,y) = y^2 - x^2$, we have $\dom(f) = \mathR^2$ and
	$\vec\nabla f(x,y) = -2x\mbi + 2y\mbj$.  Selecting any point
	$(x,y)$ in the plane, we may choose to draw $\vec\nabla f(x,y)$
	not as a vector in standard position, but rather one with
	initial point $(x,y)$, obtaining the picture at right.
  \end{minipage}\hspace{0.3in}
  \begin{minipage}[t]{3.2in}
	\mbox{}

	\vspace{-0.3in}
	\includegraphics[width=2.8in]{figs/gradientField.pdf}
  \end{minipage}

\item
  {\bf Properties of the gradient operator}:
  If $f$, $g$ are both differentiable functions of $n$ variables
  on an open region $R$, and $c$ is any real number (constant), then
  \be
  \item
	$\vec\nabla(cf) = c\vec\nabla f \qquad$ (constant multiple rule)\\
  \item
	$\vec\nabla(f \pm g) = \vec\nabla f \pm \vec\nabla g \qquad$
	(sum/difference rules)\\
  \item
	$\vec\nabla(f g) = g(\vec\nabla f) + f(\vec\nabla g) \qquad$
	(product rule)\\
  \item
	$\ds{\vec\nabla\left(\frac{f}{g}\right)
	  = \frac{g(\vec\nabla f) - f(\vec\nabla g)}{g^2}} \qquad$
	(quotient rule)
  \ee
\ei

\vspace{0.1in}
\subsection*{Directional Derivatives}
Let $\mbu = u_1\mbi + u_2\mbj$ be a unit vector (i.e., $|\mbu| = 1$),
and let $\mbr(t)$ be the vector function
$$ \mbr(s) = (x_0 + s u_1)\mbi + (y_0 + s u_2)\mbj, $$
which parametrizes the line through $P = (x_0, y_0)$ parallel to
$\mbu$ in such a way that the ``speed'' $|d\mbr/dt| = |\mbu| = 1$.
We make the following definition.\\[8pt]
\defn{
For a function $f$ of two variables that is differentiable at
$(x_0,y_0)$, we define the {\em directional derivative of} $f$
{\em at} $(x_0, y_0)$ {\em in the direction} $\mbu$ to be
$$ D_\mbu f(x_0, y_0) \;:=\; \lim_{x\rightarrow 0}
  \frac{f(x_0 + su_1, y_0 + su_2) - f(x_0, y_0)}{s} \;=\;
  \left(\frac{df}{ds}\right)_{\mbu, P} =
  \vec\nabla f(x_0,y_0)\vdotprod \mbu. $$
} \\[8pt]
Remarks:
\bi
\item
  {\bf Special cases}: the partial derivatives $f_x$, $f_y$ themselves are
  derivatives in the directions $\mbi$, $\mbj$ respectively.\\[6pt]
  \mbox{} $\qquad \ds{D_\mbi f = \vec\nabla f \vdotprod \mbi = f_x},
  \qquad\mbox{and}\qquad
  \ds{D_\mbj f = \vec\nabla f \vdotprod \mbj = f_y}$.

\item
  For $f$ a function of 2 variables, the direction of
  $\vec\nabla f(x,y)$ (namely $\vec\nabla f(x,y)/|\vec\nabla f(x,y)|$)\\[6pt]
  is the direction of maximum increase, while
  $-\vec\nabla f(x,y)/|\vec\nabla f(x,y)|$ is the direction of
  maximum decrease.\\[8pt]
  \begin{minipage}[t]{0.5in}
  Proof:
  \end{minipage}
  \begin{minipage}[t]{5.5in}
  For any unit vector $\mbu$,
  $$ D_\mbu f(x,y) \;=\; \vec\nabla f(x,y)\vdotprod\mbu
	\;=\; |\vec\nabla f(x,y)||\mbu|\cos\theta
	\;=\; |\vec\nabla f(x,y)|\cos\theta, $$
  where $\theta$ is the angle between $\vec\nabla f(x,y)$ and $\mbu$.
  This directional derivative is \\[6pt]
  largest when $\theta = 0$ (i.e., when
  $\mbu = \vec\nabla f/|\vec\nabla f|$) and smallest when $\theta = \pi$.
  \end{minipage}

\item
  The notion of directional derivative extends naturally to
  functions of 3 or more variables.

\ei

\vspace{0.1in}
\subsection*{The Gradient and Level Sets}

\begin{minipage}[t]{4in}
Suppose $f(x,y)$ is differentiable at the point $P = (x_0,y_0)$,
and let $k = f(x_0,y_0)$.  Then the $k$-level curve of $f$
contains $(x_0,y_0)$.  Suppose that we have a parametrization
of a section of this level curve containing the point $(x_0,y_0)$.
That is, let
\bi
\item
  $x(t)$ and $y(t)$ be differentiable functions of $t$ in an open
  interval $I$ containing $t=0$,
\item
  $x_0 = x(0)$ and $y_0 = x(0)$, and
\ei
\end{minipage}
\begin{minipage}[t]{4in}
\end{minipage}\\
\bi
\item
  $f(x(t), y(t)) = k$ for $t\in I$ (that is, the parametrization
  gives at least a small part of the $k$-level curve of $f$---a
  part that contains the point $P$).
\ei
Because we are parametrizing a level curve of $f$, it follows
that $df/dt = 0$ for $t\in I$.  In particular,
$$ 0 \;=\; \left.\frac{df}{dt}\right|_{t=t_0} \;=\;
  \vec\nabla f(x_0, y_0)\vdotprod [x'(t_0)\mbi + y'(t_0)\mbj]. $$
This shows that the gradient vector at $P$ is orthogonal to
the level curve of $f$ (or the tangent line to the level curve)
through $P$.  This is true at all points $P$ where $f$ is
differentiable.  That this result may be generalized to higher
dimensions is motivation for the definition of a tangent plane.

\subsection*{Tangent Planes}

\defn{Let $f(x,y,z)$ be differentiable at a point $P = (x_0, y_0, z_0)$
contained in the level surface $f(x,y,z) = k$.  We define the
{\em tangent plane to this level surface of f at} $P$ to be the plane
containing $P$ normal to $\vec\nabla f(x_0, y_0, z_0)$.
}

\newt
\eg{}
Suppose $f(x,y,z)$ is a differentiable function at the point
$P = (x_0, y_0, z_0)$ lying on the level surface $f(x,y,z) = k$.
Derive a formula for the equation of the tangent plane to this
level surface of $f$ at $P$.  Then use it to write the equation
of the tangent plane to the quadric surface
$$ f(x,y,z) = x^2 + 3y^2 + 2z^2 \;=\; 6 $$
at the point $(1,1,1)$.

\vspace{3in}
\ni
\eg{}
Suppose $z = f(x,y)$ is a differentiable function at the point
$P = (x_0, y_0)$.  Derive a formula for the equation of the tangent
plane to the surface $z = f(x,y)$ at the point
$P = (x_0, y_0, f(x_0,y_0))$.  Use it to get the equation of
the tangent plane to $z = 2x^2 - y^2$ at the point $(1, 3, -7)$.

\end{document}

