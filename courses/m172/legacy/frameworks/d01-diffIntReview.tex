\documentclass[12pt,fleqn]{article}

\newif\ifcomplete
\completetrue
%\completefalse
\def\topic{Review of Differentiation and Integration}
\def\lecdate{Mon., Jan.~29}

\oddsidemargin=0in
\textwidth=6.5in
\topmargin=-0.5in
\textheight=9.0in

\input mathMacros.tex
\usepackage{amsmath}
\usepackage{fancyhdr}
\usepackage{/Users/scofield/Library/texmf/latex/timestamp}
\usepackage[pdftex]{hyperref}
\hypersetup{pdfpagemode=None,colorlinks=true}
\def\unsc{\blank{0.07in}}
%\def\RE{$\mathrm{Re}\,$}
%\def\IM{$\mathrm{Im}\,$}
\def\RE{$\mbox{Re}\,$}
\def\IM{$\mbox{Im}\,$}


\begin{document}

\pagestyle{fancy}
\fancyhf{}
\lhead{MATH 162---Framework for \lecdate}
\chead{}
\rhead{\topic}
\lfoot{}
\cfoot{\thepage}
\rfoot{}
\renewcommand{\headrulewidth}{0.4pt}
\renewcommand{\footrulewidth}{0.4pt}

\thispagestyle{empty}

\begin{center}
  \Large{MATH 162: Calculus II} \\
  \large{Framework for \lecdate} \\
  \large{\topic}
\end{center}

\section*{Differentiation}
\vs{0.2in}
\ni
Definition of derivative $f'(x)$:
$$ \lim_{h\rightarrow 0} \frac{f(x+h) - f(x)}{h} \qquad\mbox{or}\qquad
  \lim_{y\rightarrow x} \frac{f(y) - f(x)}{y-x}. $$

\vs{0.2in}
\ni
Differentiation rules:
\be
\item {\bf Sum/Difference rule}: If $f$, $g$ are differentiable at $x_0$, then
  $$ (f \pm g)'(x_0) \= f'(x_0) \pm g'(x_0). $$
\item {\bf Product rule}: If $f$, $g$ are differentiable at $x_0$, then
  $$ (fg)'(x_0) \= f'(x_0)g(x_0) + f(x_0)g'(x_0). $$
\item {\bf Quotient rule}: If $f$, $g$ are differentiable at $x_0$, and
  $g(x_0) \ne 0$, then
  $$ \left(\frac{f}{g}\right)'(x_0) \= \frac{f'(x_0)g(x_0)
	- f(x_0)g'(x_0)}{[g(x_0)]^2}. $$
\item {\bf Chain rule}: If $g$ is differentiable at $x_0$, and $f$
  is differentiable at $g(x_0)$, then
  $$ (f \circ g)'(x_0) \= f'(g(x_0)) g'(x_0). $$
  This rule may also be expressed as
  $$ \frac{dy}{dx}\Big|_{x=x_0} \= \left(\frac{dy}{du}\Big|_{u=u(x_0)}\right)
	\left(\frac{du}{dx}\Big|_{x=x_0}\right). $$
  Implicit differentiation is a consequence of the chain rule.  For
  instance, if $y$ is really dependent upon $x$ (i.e., $y = y(x)$),
  and if $u = y^3$, then
  $$ \frac{d}{dx} (y^3) \= \frac{du}{dx} \= \frac{du}{dy}\,\frac{dy}{dx}
	\= \frac{d}{dy}(y^3) y'(x) \= 3y^2 y'. $$
  {\em Practice}: Find
  $$ \frac{d}{dx} \left(\frac{x}{y}\right), \qquad
	\frac{d}{dx} (x^2 \sqrt{y}), \qquad\mbox{and}\qquad
	\frac{d}{dx} [y\cos(xy)]. $$

\ee

\section*{Integration}

\ni
The definite integral
\bi
\item the area problem
\item Riemann sums
\item definition
\ei

\ni
{\bf Fundamental Theorem of Calculus}:
\be
\item[I:]
  Suppose $f$ is continuous on $[a,b]$.  Then the function given by
  $F(x) := \int_a^x f(t)\,dt$ is continuous on $[a,b]$ and differentiable
  on $(a,b)$, with derivative
  $$ F'(x) \= \frac{d}{dx} \int_a^x f(t)\,dt \= f(x). $$
\item[II:]
  Suppose that $F(x)$ is continuous on the interval $[a,b]$
  and that $F'(x) = f(x)$ for all $a < x < b$.  Then
  $$ \int_a^b f(x)\, dx \= F(b) - F(a). $$
\ee

\ni
Remarks:
\bi
\item
  Part I says there is always a formal antiderivative on
  $(a,b)$ to continuous $f$.  A vertical shift of one antiderivative
  results in another antiderivative (so, if one exists, infinitely
  many do).  But if an antiderivative is to pass through a particular
  point (an {\em initial value problem}), there is often just one
  satisfying this additional criterion.
\item
  Part II indicates the definite integral is equal to the total
  change in any (and all) antiderivatives.
\ei

\ni
The {\em average value of $f$ over $[a,b]$} is defined to be
$$ \frac{1}{b-a} \int_a^b f(x)\, dx, $$
when this integral exists.

\vs{0.2in}
\ni
Integration by substitution:
\bi
\item
  Counterpart to the chain rule  (Q: What rules for integration
  correspond to the other differentiation rules?)
\item Examples:
  \be
  \item $\ds{\int e^{3x}\,dx}$
  \item $\ds{\int_0^5 \frac{dx}{2x + 1}}$
  \item $\ds{\int_0^{\sqrt \pi/2} 2x\cos(x^2)\,dx}$
  \item $\ds{\int \frac{\ln x}{x}\,dx}$
  \item $\ds{\int \frac{dx}{1 + (x-3)^2}}$
  \item $\ds{\int \frac{dx}{x \sqrt{4x^2-1}}}$
  \item $\ds{\int \cos(3x)\sin(3x)\,dx}$
  \item $\ds{\int \frac{\arctan(2x)}{1 + 4x^2}\,dx}$
  \item $\ds{\int \tan^m x \sec^2 x \, dx}$
  \item $\ds{\int \tan x \, dx}\quad$ (worth extra practice)
  \item $\ds{\int \sec x \, dx}\quad$ (worth extra practice)
  \ee
\ei

\end{document}

