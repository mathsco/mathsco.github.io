\documentclass[12pt,fleqn]{article}

\newif\ifcomplete
\completetrue
%\completefalse
\def\topic{The Ratio Test}
\def\lecdate{Mon., Feb.~19}

\oddsidemargin=0in
\textwidth=6.5in
\topmargin=-0.5in
\textheight=9.0in

\input mathMacros.tex
\usepackage{amsmath}
\usepackage{fancyhdr}
\usepackage{/Users/scofield/Library/texmf/latex/timestamp}
\usepackage[pdftex]{hyperref}
\hypersetup{pdfpagemode=None,colorlinks=true}
\usepackage{graphicx}
\usepackage[usenames,dvipsnames]{color}
\def\unsc{\blank{0.07in}}
%\def\RE{$\mathrm{Re}\,$}
%\def\IM{$\mathrm{Im}\,$}
\def\RE{$\mbox{Re}\,$}
\def\IM{$\mbox{Im}\,$}


\begin{document}

\pagestyle{fancy}
\fancyhf{}
\lhead{MATH 162---Framework for \lecdate}
\chead{}
\rhead{\topic}
\lfoot{}
\cfoot{\thepage}
\rfoot{}
\renewcommand{\headrulewidth}{0.4pt}
\renewcommand{\footrulewidth}{0.4pt}

\thispagestyle{empty}

\begin{center}
  \Large{MATH 162: Calculus II} \\
  \large{Framework for \lecdate} \\
  \large{\topic}
\end{center}

\vs{0.2in}
\ni

\section*{Several Ideas Coming Together}

\bi
\item
  {\bf Power fns.}: ones of form $x^{-p}$.
  \bi
  \item {\bf Identification}.
	The base $x$ varies, while the exponent $(-p)$ remains fixed.
	These fns.~grow without bound as $x\rightarrow\infty$ when $p < 0$,
	and approach zero as $x\rightarrow \infty$ when $p > 0$.

  \item {\bf Improper integrals}.
	Only in the case $p > 0$ might we {\em hope} $\int_1^\infty x^{-p}\,dx$
	converges.  Our finding is that, in actuality, $\int_1^\infty x^{-p}\,dx$
	converges only when $p > 1$.  That is, only when $p>1$ do such
	fns.~approach 0 {\em quickly enough} so that the stated integral
	converges.  Our corresponding finding for series: $\sum_{n=1}^\infty
	n^{-p}$ converges for $p>1$ and diverges for $p\le 1$.
  \ei

\item
  {\bf Exponential fns.}: ones of form $b^x$
  \bi
  \item {\bf Identification}.
	The base $b$ remains fixed, and the exponent $x$ varies.
	Such fns.~grow without bound as
	$x\rightarrow\infty$ when $b > 1$; they approach zero
	as $x\rightarrow\infty$ when $0 < b < 1$.

  \item {\bf Improper integrals}.
	When $0 < b < 1$, the exponential $b^x$ decays to zero faster
	than any power fn.~$x^{p}$ with $p>0$.
%	(The proof for $p$ a positive integer: look at the ratio
%	$b^x/x^{-p} = x^p/(b^{-1})^x$, take the limit as $x\rightarrow\infty$,
%	and apply L'H\^{o}pital's rule $p$ times.)
	Not surprisingly, then,
	$$ \int_a^\infty b^x \,dx $$
	($a$ any real number) converges for any value of $b$ in $(0,1)$.
	The corresponding types of series are geometric series
	$$ \sum_{n=0}^\infty r^n $$
	which we know converge for $-1 < r < 1$, and diverge for $|r| \ge 1$.
  \ei

\item
  {\bf It's the tail that matters}.  The ``fate'' of a series
  $\sum a_n$ (in terms of whether or not it converges) does not depend
  on its first few, say 100 trillion or so, terms, but on the {\em tail}
  (the remaining infinitely many terms).

\ei

\np
\section*{The Ratio Test}

\ni
The previous facts point to the following conjecture concerning series:
\bq
  Given some series $\sum a_n$, if it may be determined that,
  eventually, the terms shrink in magnitude at least as quickly as
  an exponential decay function, then the series should converge.
\eq
There might be several ways to formulate this conjecture into a
mathematical statement.  One way, which has been proved (so it is
a theorem), is:

\vs{0.2in}
\ni
{\bf Theorem (Ratio test}, p.~533{\bf{)}}:
Let $\sum a_n$ be a series whose terms $a_n > 0$ (all positive),
and suppose $\lim_{n\rightarrow\infty} a_{n+1}/a_n = \rho$ (i.e.,
suppose that this limit exists; we'll call it $\rho$).
\bi
\item[(i)]
  If $\rho < 1$, the series $\sum a_n$ converges.
\item[(ii)]
  If $\rho > 1$, the series $\sum a_n$ diverges.
\ei

\vs{0.2in}
\ni
Note: If $\rho = 1$, this test does not allow us to draw
a conclusion either way.

\vs{0.2in}
\ni
{\bf Examples}:
\bi
\item[] $\ds{\sum_{n=1}^\infty \frac{2^n}{n^2 3^n}}$
\item[] $\ds{\sum_{n=1}^\infty \frac{1}{n}}$
\item[] $\ds{\sum_{n=1}^\infty \frac{1}{n^2}}$
\ei
When a series $\sum a_n$ has negative terms, we may apply the
ratio test to $\sum |a_n|$.  If the test reveals that $\sum |a_n|$
converges, then so does $\sum a_n$ by the absolute convergence test.

\vs{0.2in}
\ni
{\bf Example}: Determine those $x$-values for which the series converges.
\bi
\item[(a)] $\ds{\sum_{n=0}^\infty \frac{(3x-2)^n}{n 3^n}}$
\item[(b)] $\ds{\sum_{n=0}^\infty \frac{x^n}{n!}}$
\item[(c)] $\ds{\sum_{n=0}^\infty n! x^n}$
\ei

\end{document}

