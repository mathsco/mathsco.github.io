\documentclass[12pt,fleqn]{article}

\newif\ifcomplete
\completetrue
%\completefalse
\def\topic{Introduction to Power Series}
\def\lecdate{Tues., Feb.~20}

\oddsidemargin=0in
\textwidth=6.5in
\topmargin=-0.5in
\textheight=9.0in

\input mathMacros.tex
\usepackage{amsmath}
\usepackage{fancyhdr}
\usepackage{/Users/scofield/Library/texmf/latex/timestamp}
\usepackage[pdftex]{hyperref}
\hypersetup{pdfpagemode=None,colorlinks=true}
\usepackage{graphicx}
\usepackage[usenames,dvipsnames]{color}
\def\unsc{\blank{0.07in}}
%\def\RE{$\mathrm{Re}\,$}
%\def\IM{$\mathrm{Im}\,$}
\def\RE{$\mbox{Re}\,$}
\def\IM{$\mbox{Im}\,$}


\begin{document}

\pagestyle{fancy}
\fancyhf{}
\lhead{MATH 162---Framework for \lecdate}
\chead{}
\rhead{\topic}
\lfoot{}
\cfoot{\thepage}
\rfoot{}
\renewcommand{\headrulewidth}{0.4pt}
\renewcommand{\footrulewidth}{0.4pt}

\thispagestyle{empty}

\begin{center}
  \Large{MATH 162: Calculus II} \\
  \large{Framework for \lecdate} \\
  \large{\topic}
\end{center}

\vs{0.2in}
\ni
{\bf Definition}:
A function of $x$ which takes the series form
\begin{equation}
  \sum_{n=0}^\infty c_n (x-a)^n \;=\; c_0 + c_1 (x-a) + c_2 (x-a)^2
	+ c_3 (x-a)^3 + \cdots \label{powerSeries}
\end{equation}
is called a {\em power series about} $x=a$.  The number $a$ is called
the {\em center}, and the coefficients $c_0$, $c_1$, $c_2$, \ldots
are constants.

\vspace{0.3in}
\ni
Remarks:
\bi
\item
  As for any function, a power series has a domain.  The acceptable
  inputs $x$ to a power function are those $x$ for which the series
  converges.

\item
  If it were not for the coefficients $c_j$ (if, say, each $c_j = 1$),
  a power series would look geometric.  Indeed, the geometric series
  \begin{equation}
	\sum_{n=0}^\infty x^n \;=\; 1 + x + x^2 + x^3 + \cdots
	\label{geometricPowerSeries}
  \end{equation}
  is a power series about $x=0$, that is known to converge to
  $(1-x)^{-1}$ when $-1 < x < 1$ and to diverge when $|x| \ge 1$.
  So, the domain of power series (\ref{geometricPowerSeries}) is
  $(-1,1)$. \\[8pt]
  More generally, for a special type of power series about $x=a$
  with coefficients $c_j = \beta^j$ (i.e., whose coefficients are
  ascending powers of
  some fixed number $\beta$)
  \begin{equation}
	\sum_{n=0}^\infty \beta^n (x-a)^n \;=\; 1 + \beta(x-a) + \beta^2(x-a)^2
	  + \beta^3(x-a)^3 + \cdots, \label{modifiedGeometricPowerSeries}
  \end{equation}
  we have that this series
  \bq
	converges to $\ds{\frac{1}{1 - \beta(x-a)}}$ when
	$|\beta(x-a)| < 1$, that is, for
	$\ds{a - \frac{1}{|\beta|} < x < a + \frac{1}{|\beta|}}$, \\[6pt]
	and diverges when $|(x-a)| \ge \ds{\frac{1}{|\beta|}}$.
  \eq
  Thus, the domain of series (\ref{modifiedGeometricPowerSeries}) is
  $(a - 1/|\beta|, a + 1/|\beta|)$.

\item
  In the most general case, where the coefficients $c_j$ in
  (\ref{powerSeries}) do not, in general, equal $\beta^j$ for some
  number $\beta$, the determination of the domain usually requires
  \bi
  \item[1.]
	the use of the ratio test on the series
	$\sum_{n=0}^\infty |c_n (x-a)^n|$.  That is, one looks at
	$$ \lim_{n\rightarrow\infty} \frac{|c_{n+1}| |x - a|^{n+1}}
	  {|c_n| |x-a|^n} \;=\; \left(\lim_{n\rightarrow\infty} \frac{|c_{n+1}|}
	  {|c_n|}\right) |x-a|. $$
	If $\lim_{n\rightarrow\infty} |c_{n+1}|/|c_n|$ exists, and if we let
	$$ \rho \;=\; \left(\lim_{n\rightarrow\infty} \frac{|c_{n+1}|}
	  {|c_n|}\right) |x-a|, $$
	then part (i) of the ratio test imposes constraints on what values
	$x$ may take.  Specifically, we generally wind up with a number
	$R \ge 0$, called the {\em radius of convergence}, for which
	the series (\ref{powerSeries})
	\bq
	  converges if $x$ is inside the open interval $(a-R, a+R)$, and \\[6pt]
	  diverges if $|x - a| > R$ (i.e., if $x$ is outside the closed
	  interval $[a-R, a+R]$).
	\eq
	In those cases where $\lim_{n\rightarrow\infty} |c_{n+1}|/|c_n| = 0$,
	the value of $R := +\infty$, and when this happens there is no need
	to proceed to step 2.

  \item[2.]
	the determination (by some other means than the ratio test) of
	whether the series converges when $|x-a| = R$ (i.e., at the points
	$x = a \pm R$).
  \ei
  The upshot is that the domain of a power series whose radius of
  convergence $R$ is nonzero is always an interval, an interval
  that has $x=a$ at its center and, in the case $R \ne +\infty$,
  may include one or both of its endpoints $x = a \pm R$.  For this
  reason the domain of a power series is usually called its
  {\em interval of convergence}.
\ei

\vs{0.2in}
\ni
{\bf Example}: Determine the interval of convergence for
\be
\item[(a)] $\ds{\sum_{n=0}^\infty \frac{(3x-2)^n}{n3^n}}$
\item[(b)] $\ds{\sum_{n=0}^\infty \frac{x^n}{n!}}$
\item[(c)] $\ds{\sum_{n=0}^\infty n! x^n}$
\item[(d)] $\ds{\sum_{n=0}^\infty \frac{x^n 3^n}{n^{3/2}}}$
\ee

\np
\ni
{\bf Power Series Expressions for Some Fns.~(building new series
from known ones)}

\vs{0.1in}
\ni
We know that, for $|x| < 1$, the fn.~$f(x) = 1/(1-x)$ may be
expressed as a power series:
$$ \frac{1}{1-x} \;=\; \sum_{n=0}^\infty x^n, \qquad |x| < 1. $$
Thus, we may express similar-looking fns.~as power series:
$$ \begin{array}{ll}
	\ds{\frac{1}{1 - 3x} \;=\; \sum_{n=0}^\infty (3x)^n
	  \;=\; \sum_{n=0}^\infty 3^n x^n}, & |3x| < 1 \;\;\Rightarrow\;\;
	  \ds{-\frac{1}{3} < x < \frac{1}{3}} \\[20pt]
	\ds{\frac{1}{2-x} \;=\; \frac{1}{2}\cdot\frac{1}{1-(x/2)}
	  \;=\; \frac{1}{2}\,\sum_{n=0}^\infty \left(\frac{x}{2}\right)x^n
	  \;=\; \sum_{n=0}^\infty \frac{x^n}{2^{n+1}}}, \quad\mbox{} &
	  \ds{\left|\frac{x}{2}\right| < 1 \;\;\Rightarrow\;\; -2 < x < 2}. \\[20pt]
	\ds{\frac{x^3}{1-x} \;=\; x^3\cdot\frac{1}{1-x}
	  \;=\; x^3\,\sum_{n=0}^\infty x^n
	  \;=\; \sum_{n=0}^\infty x^{n+3}}, & -1 < x < 1. \\[20pt]
	\ds{\frac{1}{1+x} \;=\; \frac{1}{1-(-x)} \;=\; \sum_{n=0}^\infty [(-1)x]^n
	  \;=\; \sum_{n=0}^\infty (-1)^n x^n}, & |-x| < 1 \;\;\Rightarrow\;\;
	  -1 < x < 1. \\[20pt]
	\ds{\frac{1}{1+x^2} \;=\; \sum_{n=0}^\infty (-1)^n (x^2)^n
	  \;=\; \sum_{n=0}^\infty (-1)^n x^{2n}}, & |x^2| < 1 \;\;\Rightarrow\;\;
	  -1 < x < 1.
  \end{array} $$
All of the above are power series about $x=0$.  We show how the
2nd one (in the above list of 5) could also be written as a power
series centered around $x=1$:
$$ \frac{1}{2-x} \;=\; \frac{1}{1 - (x-1)}
	\;=\; \sum_{n=0}^\infty (x-1)^n, \qquad |x - 1| < 1 \;\;\Rightarrow\;\;
	0 < x < 2. $$

\end{document}

