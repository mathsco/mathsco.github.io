\documentclass[12pt,fleqn]{article}

\newif\ifcomplete
\completetrue
%\completefalse
\def\topic{Differentiation and Integration of Power Series}
\def\lecdate{Wed., Feb.~21}

\oddsidemargin=0in
\textwidth=6.5in
\topmargin=-0.5in
\textheight=9.0in

\input mathMacros.tex
\usepackage{amsmath}
\usepackage{fancyhdr}
\usepackage{/Users/scofield/Library/texmf/latex/timestamp}
\usepackage[pdftex]{hyperref}
\hypersetup{pdfpagemode=None,colorlinks=true}
\usepackage{graphicx}
\usepackage[usenames,dvipsnames]{color}
\def\unsc{\blank{0.07in}}
%\def\RE{$\mathrm{Re}\,$}
%\def\IM{$\mathrm{Im}\,$}
\def\RE{$\mbox{Re}\,$}
\def\IM{$\mbox{Im}\,$}


\begin{document}

\pagestyle{fancy}
\fancyhf{}
\lhead{MATH 162---Framework for \lecdate}
\chead{}
\rhead{\topic}
\lfoot{}
\cfoot{\thepage}
\rfoot{}
\renewcommand{\headrulewidth}{0.4pt}
\renewcommand{\footrulewidth}{0.4pt}

\thispagestyle{empty}

\begin{center}
  \Large{MATH 162: Calculus II} \\
  \large{Framework for \lecdate} \\
  \large{\topic}
\end{center}

\vs{0.2in}
\ni
While power series are allowed to have nonzero numbers as centers, for today's
results we will assume all power series we discuss are centered
about $x=0$; that is, are of the form
\begin{equation}
  \sum_{n=0}^\infty c_n x^n \;=\; c_0 + c_1 x + c_2 x^2
	+ c_3 x^3 + \cdots. \label{powerSeriesWithCenterZero}
\end{equation}
We will also assume each has a positive radius of convergence $R > 0$,
so that the series converges at least for those $x$ satisfying
$-R < x < R$.

\section*{Differentiation of Power Series about $x=0$}

{\bf Theorem (Term-by-Term Differentiation}, p.~549{\bf{)}}:
Let $f(x)$ take the form of the power series in
(\ref{powerSeriesWithCenterZero}), with radius of
convergence $R > 0$.  Then the series
$$ \sum_{n=1}^\infty n c_n x^{n-1} $$
converges for all $x$ satisfying $-R < x < R$, and
$$ f'(x) \;=\; \sum_{n=1}^\infty n c_n x^{n-1},
	\qquad\mbox{for all $x$ satisfying}\; -R < x < R. $$

\vspace{0.3in}
\ni
Remarks:
\bi
\item
  Since the hypotheses of this theorem now apply to $f'(x)$, we
  can continue to differentiate the series to find derivatives of $f$
  of all orders, convergent at least on the interval $-R < x < R$.
  For instance,
  \beqn
	f''(x) & = & \sum_{n=2}^\infty n(n-1)c_n x^{n-2}
	  \;\;=\;\; (2\cdot 1) c_2 + (3 \cdot 2) c_3 x + (4 \cdot 3) c_4 x^2
	  + \cdots \\[10pt]
	f'''(x) & = & \sum_{n=3}^\infty n(n-1)(n-2)c_n x^{n-3}
	  \;\;=\;\; (3\cdot 2\cdot 1) c_3 + (4\cdot 3 \cdot 2) c_4 x
	  + (5\cdot 4 \cdot 3) c_5 x^2 + \cdots
	  \\[10pt]
	\vdots\phantom{X} & & \phantom{XXXXXXX}\vdots \\[10pt]
	f^{(j)}(x) & = & \sum_{n=j}^\infty n(n-1)(n-2)\cdots(n-j+1) c_n x^{n-j} \\
	& = & j!\,c_j + [(j+1)j\cdots 2]\, c_{j+1}x + [(j+2)(j+1)\cdots 3]\,
	  c_{j+2}x^2 + \cdots.
  \eeqn

\item
  The easiest place to evaluate a power series is at its center.  In
  particular, if $f$ has the form (\ref{powerSeriesWithCenterZero}),
  we may evaluate $f$ and all of its derivatives at zero to get:
  $$ \begin{array}{lcl}
	  f(0) \;=\; c_0, \\[4pt]
	  f'(0) \;=\; c_1, \\
	  f''(0) \;=\; (2\cdot 1)\,c_2
		& \Rightarrow & c_2 = \ds{\frac{1}{2}} f''(0), \\[12pt]
	  f'''(0) \;=\; (3\cdot 2\cdot 1)\, c_3
		& \Rightarrow & c_3 = \ds{\frac{1}{3!}} f'''(0), \\
	\end{array} $$
  and so on.  So, we arrive at the following relationship
  between the coefficients $c_j$ and the derivatives of $f$
  at the center:

  \bq
	{\bf Corollary}: Let $f(x)$ be defined by the power series
	(\ref{powerSeriesWithCenterZero}).  Then
	$$ c_n \;=\; \frac{f^{(n)}(0)}{n!}, \qquad\mbox{for all}\;
		n = 0, 1, 2, \ldots. $$
  \eq
\ei

\vs{0.2in}
\ni
{\bf Example:}
We know that $f(x) = (1-x)^{-1}$ has the power series representation
$\sum_{n=0}^\infty x^n$ for $x$ in the interval $(-1,1)$.
By the term-by-term differentiation theorem, $f'(x) = (1-x)^{-2}$
has the power series representation
\beqn
  \frac{1}{(1 - x)^2} & = & \frac{d}{dx}\left(\sum_{n=0}^\infty
	x^n\right) \;\;=\;\; \frac{d}{dx}\left(1 + x + x^2 + x^3 + \cdots\right)\\
  & = & \frac{d}{dx}(1) + \frac{d}{dx}(x) + \frac{d}{dx}(x^2)
	+ \frac{d}{dx}(x^3) + \cdots
	\qquad\mbox{(the theorem justifies this step)} \\
  & = & 0 + 1 + 2x + 3x^2 + 4x^3 + \cdots \\
  & = & \sum_{n=1}^\infty nx^{n-1},
\eeqn
with this series representation holding at least for $-1 < x < 1$.

\section*{Integration of Power Series about $x=0$}

\ni
If we can differentiate a series expression for $f$ term-by-term
in order to arrive at a series expression for $f'$, it may not be
surprising that we may integrate a series term-by-term as well.

\vspace{0.3in}
\ni
{\bf Theorem (Term-by-Term Integration}, p.~550{\bf{)}}:
Suppose that $f(x)$ is defined by the power series
(\ref{powerSeriesWithCenterZero}) and the radius of convergence
$R > 0$.  Then
$$ \int_0^x f(t)\,dt \;=\; \sum_{n=0}^\infty \frac{c_n}{n+1}\,x^{n+1},
\qquad\mbox{for all $x$ satisfying}\; -R < x < R. $$

\vs{0.2in}
\ni
{\bf Example:}
We know that $f(x) = (1+x)^{-1}$ has the power series representation
$\sum_{n=0}^\infty (-1)^n x^n$ for $x$ in the interval $(-1,1)$.
Moreover,
$$ \int_0^x \frac{dt}{1+t} \;=\; \Big[\ln |1+t|\Big]_0^x
	\;=\; \ln|1 + x|. $$
By the term-by-term integration theorem, for $-1 < x < 1$ we also have
\beqn
  \int_0^x \frac{dt}{1+t} & = & \int_0^x \left(\sum_{n=0}^\infty
	(-1)^n t^n\right)\,dt
	\;\;=\;\; \int_0^x (1 - t + t^2 - t^3 + t^4 - t^5 + \cdots)\,dt
	\\[8pt]
  & = & \int_0^x dt - \int_0^x t\,dt + \int_0^x t^2\,dt - \int_0^x t^3\,dt
	+ \cdots \qquad\mbox{(the theorem justifies this step)} \\[8pt]
  & = & \sum_{n=0}^\infty (-1)^n\,\left(\int_0^x t^n\,dt\right) \\[8pt]
  & = & \sum_{n=0}^\infty \frac{(-1)^n}{n+1} \Big[t^{n+1}\Big]_0^x \\[8pt]
  & = & \sum_{n=0}^\infty \frac{(-1)^n}{n+1} x^{n+1} \\[8pt]
  & = & x - \frac{1}{2} x^2 + \frac{1}{3} x^3 - \frac{1}{4} x^4 + \cdots.
\eeqn
That is,
$$ \sum_{n=0}^\infty \frac{(-1)^n}{n+1} x^{n+1} \;=\;  \ln (1 + x), $$
at least for all $x$ in the interval $-1 < x < 1$.  In fact, though the
theorem does not go so far as to guarantee convergence at the value
$x=1$, since the series on the left converges at $x=1$ (Why?), one
might suspect that
$$ 1 - \frac{1}{2} + \frac{1}{3} - \frac{1}{4} + \cdots \;=\; \ln 2. $$
This, indeed, is the case.

\vs{0.2in}
\ni
{\bf Example}: Use the power series representation for $(1+x^2)^{-1}$
about zero to get a power series representation for $\arctan x$.

\end{document}

