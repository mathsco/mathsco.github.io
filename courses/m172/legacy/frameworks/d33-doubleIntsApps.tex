\documentclass[12pt,fleqn]{article}

\newif\ifcomplete
\completetrue
%\completefalse
\def\topic{Applications of Double Integrals}
\def\lecdate{Mon., Apr.~16}

\oddsidemargin=0in
\textwidth=6.5in
\topmargin=-0.5in
\textheight=9.0in

\input mathMacros.tex
\usepackage{amsmath}
\usepackage{ascii}
%\usepackage{wasysym}
\usepackage{fancyhdr}
\usepackage{/Users/scofield/Library/texmf/latex/timestamp}
\usepackage[pdftex]{hyperref}
\hypersetup{pdfpagemode=None,colorlinks=true}
\usepackage{graphicx}
\usepackage[usenames,dvipsnames]{color}
\def\mbunsc{\blank{0.07in}}
\def\pref#1{(\ref{#1})}
%\def\RE{$\mathrm{Re}\,$}
%\def\IM{$\mathrm{Im}\,$}
\def\RE{$\mbox{Re}\,$}
\def\IM{$\mbox{Im}\,$}
\def\fb#1{\framebox{\parbox[t]{6.5in}{#1}}}
\def\sfb#1{\framebox{\parbox[t]{6.0in}{#1}}}
\def\defn#1{\fb{{\bf Definition}: #1}}
\def\sdefn#1{\sfb{{\bf Definition}: #1}}
\def\thm#1{\fb{{\bf Theorem}: #1}}
\def\sthm#1{\sfb{{\bf Theorem}: #1}}
\def\eg#1{{\bf Example}: #1}
\def\egs#1{{\bf Examples}: #1}
\def\newt{\vspace{0.2in}\ni}
\setlength{\fboxsep}{5pt}
%\def\vdotprod{{\ascii\BEL}}
\def\vdotprod{\,\mbox{{\Large $\cdot$}}\,}
\def\vcrossprod{\times}


\begin{document}

\pagestyle{fancy}
\fancyhf{}
\lhead{Framework for \lecdate}
\chead{}
\rhead{\topic}
\lfoot{}
\cfoot{\thepage}
\rfoot{}
\renewcommand{\headrulewidth}{0.4pt}
\renewcommand{\footrulewidth}{0.4pt}

\thispagestyle{empty}

\begin{center}
  \Large{MATH 162: Calculus II} \\
  \large{Framework for \lecdate} \\
  \large{\topic}
\end{center}

\vs{0.2in}
\ni
{\bf Today's Goal}:
To use double integrals meaningfully in solving problems.

\vspace{0.15in}
\ni
{\bf Important Note}: In conjunction with this framework,
you should look over Section 13.3 of your text.

\vspace{0.15in}
\subsection*{Area}

If $R$ is a bounded region of the plane, then $\iint_R dA$
gives the area of $R$.  (This is because the volume under
the curve $z = 1$ over the region $R$, while it has different
units, is the same as the area of $R$.)

\vspace{0.15in}
\subsection*{Average Value of a Function}
For $y = f(x)$ (a function of one variable), we defined
the average value of $f$ over the interval $[a,b]$ to be
$$ \frac{1}{b-a}\,\int_a^b f(x)\,dx \;=\;
	\frac{1}{\mbox{length of interval $[a,b]$}}\,\int_a^b f(x)\,dx. $$
Similarly, \\[10pt]
\defn{If $f(x,y)$ is integrable over a region $R$ of the plane,
then the {\em average value} of $f$ over $R$ is defined to be
$$ \frac{1}{\mbox{Area}(R)}\,\iint\limits_R f(x,y)\,dA. $$}

\vspace{0.15in}
\subsection*{Integral of a Density}
Some functions give the amount of something per unit of
measurement, as in
\bi
\item
  $f(x)$ is the number of grams per unit length in an
  (idealized) 1-dimensional string.
\item
  $f(x,y)$ is the number of grams per unit area in an
  (idealized) 2-dimensional plate.
\item
  $f(x,y,z)$ is the number of molecules per unit volume
  of a certain gas.
\ei
Such functions are collectively known as {\em densities}.
The examples above are 1, 2 and 3-dimensional densities
respectively.\\[10pt]
If $f(x,y)$ gives the density (2-dimensional) of something
at each point $(x,y)$ in the region $R$, then
$ \iint_R f(x,y) \,dA $ is the total of that substance found in $R$.

\end{document}

