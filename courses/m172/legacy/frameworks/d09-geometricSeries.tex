\documentclass[12pt,fleqn]{article}

\newif\ifcomplete
\completetrue
%\completefalse
\def\topic{Geometric Series and Series Introduction}
\def\lecdate{Thurs., Feb.~15}

\oddsidemargin=0in
\textwidth=6.5in
\topmargin=-0.5in
\textheight=9.0in

\input mathMacros.tex
\usepackage{amsmath}
\usepackage{fancyhdr}
\usepackage{/Users/scofield/Library/texmf/latex/timestamp}
\usepackage[pdftex]{hyperref}
\hypersetup{pdfpagemode=None,colorlinks=true}
\usepackage{graphicx}
\usepackage[usenames,dvipsnames]{color}
\def\unsc{\blank{0.07in}}
%\def\RE{$\mathrm{Re}\,$}
%\def\IM{$\mathrm{Im}\,$}
\def\RE{$\mbox{Re}\,$}
\def\IM{$\mbox{Im}\,$}


\begin{document}

\pagestyle{fancy}
\fancyhf{}
\lhead{MATH 162---Framework for \lecdate}
\chead{}
\rhead{\topic}
\lfoot{}
\cfoot{\thepage}
\rfoot{}
\renewcommand{\headrulewidth}{0.4pt}
\renewcommand{\footrulewidth}{0.4pt}

\thispagestyle{empty}

\begin{center}
  \Large{MATH 162: Calculus II} \\
  \large{Framework for \lecdate} \\
  \large{\topic}
\end{center}

\vs{0.2in}
\ni
\section*{Geometric Series}
\bi
\item Form of series under this classification
  $$ a + ar + ar^2 + \cdots + ar^n + \cdots \;=\; \sum_{n=0}^\infty ar^n, $$
  $a$, $r$ nonzero constants

\item Zeno's paradox about crossing a room
  \bi
  \item
	If $L$ is length of room, then he is looking at adding up distances
	$$ L\cdot \left(\frac{1}{2}\right) + L\cdot \left(\frac{1}{2}\right)^2
	  L\cdot \left(\frac{1}{2}\right)^3 + \cdots
	  \;=\; \sum_{n=0}^\infty ar^n, $$
	with $a = L/2$, $r = 1/2$.
  \item
	Evidence that (some) geometric series converge
  \ei

\item
  Partial sums $s_n$
  \bi
  \item Define in customary way:
	$$ s_1 = a, \; s_2 = a + ar, \; s_3 = a + ar + ar^2, \quad \mbox{etc.}$$

  \item
	$n$th partial sum has nice closed-form formula:
	$$ s_n \;=\; \left\{\begin{array}{ll}
		\ds{\frac{a(1 - r^n)}{1-r}}, & \mbox{when} \; r \ne 1, \\[10pt]
		na, & \mbox{when} \; r = 1.
	  \end{array}\right. $$

  \item
	{\bf Main Result}: Geometric series $\ds{\sum_{n=0}^\infty ar^n}$
	converges to $\ds{\frac{a}{1-r}}$ when $|r| < 1$, and diverges otherwise.

  \ei

\item
  Note the {\em divergence} when $|r| = 1$: \\
  $r = 1: \;\; \ds{\sum_{n=0}^\infty a} = a + a + \cdots + a + \cdots
	\qquad$(divergent) \\[8pt]
  $r = -1: \;\; \ds{\sum_{n=0}^\infty a} = a - a + a - a + a - a + \cdots
	\qquad$(divergent) \\
\ei

\np
\ni
Remarks concerning infinite series (general, not just geometric ones)
$\ds{\sum_{n=1}^\infty a_n}$:
\be
\item
  Convergence relies on the partial sums $s_n := a_1 + \cdots + a_n$
  approaching a limit as $n\rightarrow\infty$
\item
  Assessing the limit of partial sums directly requires a nice
  closed-form expression for $s_n$.  Such an expression exists only
  in rare cases, such as the following examples we've already done
  \bi
  \item[] Geometric series: $\;\ds{\sum_{n=0}^\infty ar^n}$
  \item[]
	``Telescoping series'': $\;\ds{\sum_{n=1}^\infty \left(\frac{1}{n}
	- \frac{1}{n+1}\right)}$
  \ei
\item
  When no closed-form expression for $s_n$ is available, determining
  if limit exists is usually more difficult. \\[8pt]
  {\bf Example}: $\ds{\sum_{n=1}^\infty \frac{(-1)^{n-1}}{n^p}}$,
  $\; p \ge 0$
\item
  Systematic tests can help
  \bi
  \item Some of the tests that have been developed ('*'
	indicate ones we will study)
	\bi
	\item *{\bf $n$th-term test for divergence} (p.~519)
	\item
	  {\bf integral test} (p.~525): formalization of the approach
	  we used to determine which $p$-series $\ds{\sum_{n=1}^\infty n^{-p}}$
	  converge/diverge
	\item
	  {\bf direct comparison test} (p.~529): practically a restatement
	  of the one of the same name for improper integrals
	\item
	  {\bf limit comparison test} (p.~530): did not do comparable
	  result for improper integrals
	\item *{\bf ratio test} (p.~533)
	\item {\bf root test} (p.~535)
	\item
	  {\bf alternating series test} (p.~538): formalization of
	  the approach we used to show
	  $\ds{\sum_{n=1}^\infty \frac{(-1)^{n-1}}{n^p}}$ converges
	  for $p \ge 0$
	\item *{\bf absolute convergence test} (p.~540)
	\ei
  \item Must be cognizant of
	\bi
	\item
	  the situations in which a test may be applied
	\item
	  what conclusions may and may not be drawn from such tests
	\ei
  \ei
\ee

\end{document}

